\chapter{Verwandte Arbeiten}\label{sec:chapter9}

In diesem Kapitel werden verwandte Arbeiten vorgestellt. Es werden Arbeiten zu den Themen Modellierung von Softwareentwicklungsprozessen, Verständlichkeit von Prozessmodellierungssprachen und Vergleich von Prozessmodellierungssprachen beschrieben und gegenüber der Thematik der vorliegenden Arbeit abgegrenzt.

\section{Modellierung von Softwareentwicklungsprozessen}

Es gibt schon einige Arbeiten, in welchen es um die Modellierung von Software Engineering Prozessmodellen in BPMN geht.
\cite{Menhorn2014} beschäftigt sich mit der Analyse und der Überführung von Softwareentwicklungsprozessen in die Prozessmodellierungssprache BPMN und der Erweiterung von BPMN bei eventuellen Grenzen der Darstellbarkeit.\newline
Weiterhin gibt es drei Arbeiten bzw. Blogeinträge, bei welchen es um die Modellierung von Scrum \cite{software}, Open UP \cite{brunner2007fallstudie} und V-Modell XT \cite{Bregenzer2014} in BPMN geht. Hier werden jeweils Teile der drei Software-Engineering Prozessmodelle analysiert und anschließend in BPMN modelliert.\newline
In \cite{sabrina734, sabrina758, sabrina795} wurden Softwareentwicklungsprozesse in einem System umgesetzt. Das Ziel war es, dass die Softwareentwicklungsprozesse dort komplett ausführbar sind. Die Softwareentwicklungsprozesse wurden imperativ umgesetzt. Eine deklarative Umsetzung erfolgte für dynamische Abläufe, welche im laufenden Tagesgeschehen außerhalb der Modelle abgearbeitet wurden.
 Im Gegensatz zur vorliegenden Arbeit wurde in den Arbeiten \cite{software}, \cite{brunner2007fallstudie}, \cite{Bregenzer2014} jeweils nur ein Softwareentwicklungsprozess modelliert und auch nur in der imperativen Prozessmodellierungssprache BPMN. In \cite{Menhorn2014} wurden ebenfalls Teile von Scrum, Open UP und V-Modell XT modelliert, jedoch nur in der imperativen Prozessmodellierungssprache BPMN. Der Fokus der Arbeit \cite{Menhorn2014} lag auf der Erweiterung von BPMN um zusätzliche Notationselemente.\newline
 In der vorliegenden Arbeit hingegen werden Teile von allen drei Softwareentwicklungsprozessmodellen in der imperativen Prozessmodellierungssprache BPMN und der deklarativen Prozessmodellierungssprache ConDec modelliert. Der Fokus der vorliegenden Arbeit liegt zudem auf der Beurteilung der Eignung zur Modellierung der beiden Prozessmodellierungssprachen und auf deren Vergleich und nicht auf der Modellierung der Softwareentwicklungsprozesse an sich oder der Erweiterung der Notationsumfänge der beiden Prozessmodellierungssprachen.\newline
 In \cite{sabrina795, sabrina734, sabrina758} wurden ebenfalls Softwareentwicklungsprozesse imperativ und deklarativ umgesetzt. Jedoch erfolgte dort im Gegensatz zur vorliegenden Arbeit kein Vergleich zwischen imperativen und deklarativen Modellierungsansätzen, sondern es wurde je nachdem, ob es sich um einen statischen oder dynamischen Prozess handelt, ein imperativer oder deklarativer Ansatz gewählt.\newline
Keine der hier vorgestellten Arbeiten beschäftigt sich mit dem Vergleich der Anwendbarkeit von imperativen und deklarativen Prozessmodellierungsansätzen im Kontext von Softwareentwicklungsprozessen.


\section{Verständlichkeit von Prozessmodellierungssprachen}

Verwandte Arbeiten zur Thematik Verständlichkeit von Prozessmodellierungssprachen werden im Folgenden vorgestellt.

Es existieren bereits einige Arbeiten, welche die Verständlichkeit von Prozessmodellen untersuchen. \cite{bpm07} z.B. beleuchtet die Verständlichkeit von Ereignisgesteuerten Prozessketten (EPK), während sich \cite{gruhn2006complexity} der Komplexität und Verständlichkeit von BPMN und UML-Diagrammen widmet. \cite{reijers2011study} untersucht die Verständlichkeit von EPK und BPMN im Hinblick auf die Komplexität der XOR-, OR- und UND-Verzweigungen und in \cite{gruhn2006adopting} wird die Verständlichkeit von BPMN durch Zuordnung von kognitiven Werten zu den einzelnen Notationselementen bei BPMN beurteilt. \newline
In \cite{pinggera2012tracing, forster2012collaborative, pinggera2010investigating} wird der Prozess der Prozessmodellierung aus individueller und kollaborativer Sicht untersucht. Es wird beleuchtet, wie Prozessmodellierer beim Erstellen von Prozessmodellen vorgehen. Zum Erforschen dieses Vorgehens wird auch ein System entwickelt, welches den Prozess des Prozessmodellierens des Anwenders analysieren kann. \newline
In \cite{sabrina942} wird die Verständlichkeit von deklarativen Prozessmodellen im Hinblick auf die Verwendung von hierarchischen Unterprozessen untersucht. Eine weitere Arbeit, welche sich mit der Verständlichkeit von deklarativen Prozessmodellen beschäftigt, ist \cite{haisjackl2014understanding}. Hier wird das Vorgehen von Systemanalysten beim Verstehen von deklarativen Modellen untersucht. In \cite{sabrina933} wird ebenfalls das Verständnis von deklarativen Prozessmodellen im Hinblick auf das Lesen von deklarativen Modellen, die Kombination von Constraints und Unterschiede zwischen flachen und hierarchischen Prozessmodellen beleuchtet. Diese Arbeiten stützen sich bei ihrer Untersuchung hauptsächlich auf empirische Daten, welche durch das Durchführen von Studien zustande gekommen sind. Die vorliegende Arbeit verwendet bei der Untersuchung der Verständlichkeit der deklarativen Prozessmodelle in erster Linie theoretische Erkenntnisse. Diese werden anschließend durch empirische Daten gestützt. \newline
In der vorliegenden Arbeit werden einige Erkenntnisse der vorgestellten Arbeiten beim Modellieren und beim Vergleich der Prozessmodelle beachtet. Beispielsweise wurden Unterprozesse beim  Modellieren verwendet um komplexe Prozessmodelle übersichtlicher darzustellen. Diese wurden aber beim Modellieren von sowohl BPMN als auch ConDec bei den gleichen Sachverhalten angewendet.\newline
Außerdem wurden die Ergebnisse von \cite{gruhn2006adopting} zum schwierigeren Verständnis von Patterns in BPMN bei der Durchführung des Vergleiches herangezogen.\newline
\cite{pinggera2012tracing, forster2012collaborative, pinggera2010investigating} untersuchen im Gegensatz zur vorliegenden Arbeit den Prozess des Prozessmodellierens an sich, aber sie vergleichen keine Prozessmodellierungssprachen im Hinblick auf eine geeignetere Anwendbarkeit bei der Modellierung für den Modellierer und den späteren Leser der Modelle.\newline
Im Gegensatz zu den hier vorgestellten Arbeiten liegt der Fokus des Vergleiches in dieser Arbeit nicht vor allem auf der Eignung zur Modellierung der beiden Prozessmodellierungssprachen. Die hier vorgestellten Arbeiten untersuchen alle die Verständlichkeit von Prozessmodellierungssprachen für den Leser der Prozessmodelle. Keiner dieser Arbeiten untersucht die Anwendbarkeit der Prozessmodellierungssprache bei der Modellierung. \newline


\section{Vergleich von Prozessmodellierungssprachen}

Dieser Abschnitt widmet sich Arbeiten, welche sich mit dem Vergleich von Prozessmodellierungssprachen beschäftigen.\newline
Die Arbeit \cite{recker2007does} untersucht Unterschiede in der Verständlichkeit zwischen Ereignisgesteuerten Prozessketten (EPK) und BPMN.\newline
Der Artikel \cite{fahland2010} beschäftigt sich mit dem Unterschied zwischen imperativen und deklarativen Prozessmodellierungssprachen und arbeitet deren Stärken und Schwächen heraus. Der Vergleich baut auf den Unterschieden von imperativen und deklarativen Programmiersprachen auf. \newline
\cite{pichler2012} untersucht aufbauend auf den Erkenntnissen von \cite{fahland2010} die Verständlichkeit von imperativen und deklarativen Prozessmodellierungssprachen anhand einer Studie. Als imperative Prozessmodellierungssprache dient in diesem Artikel ebenfalls BPMN und als deklarative Prozessmodellierungssprache ebenso ConDec. \newline 
\cite{fahland2010} und \cite{pichler2012} untersuchen, bei welchen abzubildenden Informationen entweder imperative oder deklarative Prozessmodellierungssprachen vorzuziehen sind. In der vorliegenden Arbeit wird beim Vergleich der Anwendbarkeit der beiden Prozessmodellierungssprachen nicht unter bestimmten abzubildenden Informationen unterschieden. Der Vergleich wird im Gegensatz zu \cite{fahland2010} und \cite{pichler2012}  ganz allgemein für alle möglichen abzubildenden Informationen durchgeführt. \newline
In der durchgeführten Studie in \cite{pichler2012} wurden die Teilnehmer vorher sowohl in der imperativen als auch in der deklarativen Prozessmodellierung geschult. In der Studie der vorliegenden Arbeit wurde darauf geachtet, Probanden mit unterschiedlichem Hintergrundwissen zu imperativen und deklarativen Prozessmodellen zu befragen. Das Wissen der Teilnehmer der Studie reichte hier von sehr großem Wissen bis überhaupt kein Wissen.\newline
Die vorgestellten Arbeiten vergleichen alle Prozessmodellierungssprachen anhand der Verständlichkeit aus Sicht des Lesers der Prozessmodelle. Keine dieser Arbeiten vergleicht Prozessmodellierungssprachen aus Sicht des Modellierers, wie es in dieser Arbeit ebenfalls der Fall ist. Zudem stützen sich die Vergleiche der hier vorgestellten Arbeiten größtenteils auf empirische Daten, während sich der Vergleich der vorliegenden Arbeit sowohl auf theoretische als auch empirische Daten stützt.



