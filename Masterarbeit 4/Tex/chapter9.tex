\chapter{Related Work}\label{sec:chapter9}

\section{Modellierung von Software-Engineering Prozessmodellen}

\section{Verständlichkeit von Prozessmodellierungssprachen}

\subsection{Investigating expressiveness and understandability of hierarchy
in declarative business process models}

In \ref{sabrina942} wird die Verständlichkeit von deklarativen Prozessmodellen im Hinblick auf die Verwendung von hierarchischen Unterprozessen untersucht. Es wird gezeigt, dass der Einsatz von Hierarchie in deklarativen Prozessmodellen zwar deren Ausdrucksfähigkeit erhöht, aber nicht beliebig überall im Modell eingesetzt werden kann. Eine durchgeführte Studie zum Validieren der Erkenntnisse kam zu dem Ergebnis, dass Hierarchie beim Verstehen der Prozessmodelle unterstützt, jedoch waren die Ergebnisse nicht schlüssig.\newline
Die Arbeit  \ref{sabrina942} fokussiert auf den Einsatz von Unterprozessen als Unterstützung bei der Verständlichkeit von deklarativen Prozessmodellen. In der vorliegenden Arbeit wurden Unterprozesse beim Modellieren verwendet, um komplexe Prozessmodelle verständlicher zu machen. Diese wurden  aber beim Modellieren von sowohl BPMN, als auch ConDec bei den gleichen Sachverhalten verwendet. Die vorliegende Arbeit auf den Vergleich von deklarativen und imperativen Prozessmodellen im Allgemeinen und nicht im Hinblick auf die Verwendung von Unterpozessen. \newline


\section{Vergleich von Prozessmodellierungssprachen}

\subsection{Imperative versus Declarative Process Modeling Languages: An Empirical Investigation}

\subsection{Declarative vesus Imperative Process Modeling Languages: An Empirical Investigation}
