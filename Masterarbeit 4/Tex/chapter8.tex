\chapter{Zusammenfassung und Ausblick}\label{sec:chapter8}
 In diesem Kapitel werden die erarbeiteten Ergebnisse in Kapitel 8.1 zusammengefasst. ein Fazit der Ergebnisse der vorliegenden Arbeit folgt in Kapitel 8.2. Weiterhin wird in Kapitel 8.3 ein Ausblick auf weitere zukünftig mögliche Forschungsthemen im Bereich der Anwendbarkeit von imperativen und deklarativen Prozessmodellierungssprachen gegeben.

\section{Zusammenfassung}

Im Verlauf dieser Arbeit wurden zunächst in Kapitel 2 grundlegende Begriffe im Bereich Prozessmodelle erläutert. Hierfür wurde zunächst in Kapitel 2.1 der Begriff Software Engineering erklärt sowie auf dessen Ziele, dessen Prozess und dessen Prinzipien eingegangen.\newline
Anschließend wurden in Kapitel 2.2 Softwareprozessmodelle vorgestellt. Weiterhin wurden Software-Projekttypen sowie Leichtgewichtige und Schwergewichtige Prozessmodelle eingeführt.\newline

In Kapitel 3 wurden ausführlich die Grundlagen der Modellierung dargelegt. Hierfür erfolgte zunächst eine Einführung in die Prozessmodellierung und es wurden deren Ziele sowie die Grundsätze ordnungsgemäßer Modellierung vorgestellt.\newline

Prozessmodellierungssprachen wurden sodann in Kapitel 3.2 erläutert. Hier wurden auch die in dieser Arbeit verwendeten imperativen und deklarativen Prozessmodellierungssprachen eingeführt.\newline
Ebeneso wurden in Kapitel 3.3 Modellierungswerkzeuge allgemein erklärt und es wurden die in der vorliegenden Arbeit eingesetzten Modellierungstools vorgestellt.\newline

Die Anforderungserhebung erfolgte in Kapitel 4. Hier wurden die Vergleichskriterien für den in Kapitel 5 folgenden Vergleich  festgelegt.\newline

Die Modellierung der imperativen und deklarativen Modell der drei Softwareprozessmodelle Scrum, Open UP und V-Modell XT erfolgte in Kapitel 5. Nach einer kurzen Einführung und Analyse der Modelle wurde die imperative und deklarative Modellierung der selbigen dargelegt. Anschließend erfolgte jeweils der Vergleich der Modellierung. In Kapitel 6.4 wurde abschließend ein Vergleich zwischen allen modellierten Prozessmodellen anhand der Anforderungskriterien aus Kapitel 4 durchgeführt. \newline

In Kapitel 6 wurden die herausgearbeiteten Ergebnisse aus Kapitel 5 mit Hilfe einer Studie validiert. Zunächst wurden der Aufbau und der Ablauf der Studie genau erläutert. Weiterhin wurden die Ergebnisse der Studie präsentiert und ein Fazit gezogen. \newline

Verwandte Arbeiten wurden abschließend in Kapitel 7 gegenüber der vorliegenden Arbeit abgegrenzt. Hierfür wurden jeweils Arbeiten aus den Bereichen Entwicklung von Softwareprozessmodellen, Verständlichkeit von Prozessmodellierungssprachen und Vergleich von Prozessmodellierungssprachen vorgestellt und es wurde erläutert in wie weit sich die vorliegende Arbeit von diesen Arbeiten abgrenzt.

\section{Fazit}

Die Ergebnisse der vorliegenden Arbeit zeigen, dass der imperative Prozessmodellierungsansatz BPMN bei vielen Punkten einen geeigneteren Ansatz darstellt, als ConDec. Dies hat der direkte Vergleich der verschiedenen Modelle in Kapitel 5 ergeben. Weiterhin wurden diese Ergebnisse auch in der Studie in Kapitel 6 nochmals weitgehend bestätigt.\newline
Im Gegensatz zu BPMN gibt es bei ConDec momentan noch einige Grenzen in der Darstellbarkeit. Dies bezieht sich auf Rollen und Artefakte. Diese können zwar im Modellierungstool Declare angelegt und zugewiesen werden jedoch sind diese im Prozessmodell selbst nicht visualisierbar.\newline
Zwar entstehen bei der Modellierung mit BPMN oftmals Modelle mit deutlich mehr Elementen insgesamt, als bei der Modellierung mit ConDec, jedoch sind im Gegensatz dazu die mit ConDec erstellten Modelle oftmals deutlich komplexer als die BPMN Modelle. Dies wirkt sich sowohl negativ auf das Verständnis der Leser des Modelles aus, als auch auf den Aufwand des Modellierers beim Erstellen der Modelle mit ConDec. Je mehr geistiger Aufwand beim Lesen und modellieren für Prozessmodelle notwendig ist desto mehr Fehler entstehen auch beim interpretieren und modellieren der Modelle.\newline
Somit ist BPMN im Moment die geeignetere Modellierungssprache. Bei ConDec müsste noch nachgebessert werden um die Eignung zur Modellierung noch zu steigern.\newline


\section{Ausblick}

Im Laufe der Arbeit haben sich Ideen für weiter Forschungen im Bereich der Anwendbarkeit der imperativen und deklarativen Prozessmodellierungssprachen ergebn, welche jedoch in der vorliegenden Arbeit nicht fortgeführt werden konnten, da sie den Rahmen der Arbeit gesprengt hätten.\newline

\textbf{Cognitive Werte für ConDec Constraints}

Um den Vergleich zwischen den in BPMN und ConDec erstellten Modellen zu optimieren sollten ebenfalls genaue Werte über den geistigen Aufwand zum Verstehen der Constraints in ConDec erhoben werden. Wie in Kapitel 4 erwähnt existieren solche Werte für die Elemente der BPMN Notation (z.B. eine geistige Gewichtung von 1 für ein Sequenzflusselement, eine geistige Gewichtung von 4 für ein paralleles Gateway). Da keine genauen geistigen Gewichte für die Constraints in ConDec vorliegen, war der Vergleich zwischen den Modellen in der vorliegenden Arbeit nur grob möglich. Ein exakter Vergleich wäre mit der genauen Gegenüberstellung der Summen der geistigen Gewichte der jeweiligen Elemente in den BPMN und ConDec Modellen möglich.\newline

\textbf{Cognitive Werte für verschiedene Elemente in Prozessmodellen}

Genau wie geistige Werte für Constraints an sich fehlen, fehlen auch genaue werte, in wie stark die Komplexität eines Modelles wächst, pro zusätzlichem unterschiedlichem Element. Z.B. wäre es interessant für genauere Vergleiche zwischen Prozessmodellierungssprachen, in wie weit sich die Komplexität eines Prozessmodelles erhöht, wenn sich darin statt drei unterschiedlicher Elemente fünf unterschiedliche Elemente befinden. Gerade bei den ConDec Modellen gab es oftmals mehr unterschiedliche Constraints in den Modellen, als unterschiedliche Gateways in den BPMN Modellen. Daher wäre es für einen exakten Vergleich der Modelle nützlich, wenn man einen genauen Faktor hätte, um den sich die Komplexität in den ConDec Modellen erhöht, in Abhängigkeit der Anzahl der unterschiedlichen Constraints.  


\textbf{Erweiterung/Anpassung ConDec}

Die Prozessmodellierungssprache ConDec hat großes Potenzial. Jedoch weist sie derzeit noch einige Mängel auf. Hier können die fehlenden Visualisierungsmöglichkeiten von Rollen und Artefakten im Prozessmodell selbst genannt werden. Hier sollte ConDec noch um weitere Notationselemente erweitert werden, welche Rollen und Artefakte darstellen.\newline
Da auch vielen unterschiedlichen Constraints bei ConDec ein Problem darstellen, könnte hier in weiteren Forschungen versucht werden, die Constraints eventuell anzupassen/umzustellen, um ein kleineres Spektrum an Constraints zu haben.\newline
Weiterhin sollte es auch möglich gemacht werden, mehrere Aktivitäten mit dem Init-Constraint auszustatten, um diese als mögliche Startaktivitäten zu markieren. Damit würde etwas Komplexität aus den Modellen heraus genommen, da der Leser des Modelles sich auf einen Blick orientieren könnte, bei welchen Aktivitäten gestartet werden kann.\newline

\textbf{Weitere Studien mit breiterem Spektrum an Teilnehmern}

Wie bereits in Kapitel 7 erwähnt, weist die in dieser Arbeit durchgeführte Studie dahingehend Mängel auf, dass ein großer Teil der Teilnehmer bereits über Wissen in imperativen Prozessmodellierungssprachen verfügte. Hier sollten zukünftig noch weitere Studien durchgeführt werden, bei denen mehr Personen teilnehmen, die entweder keine Erfahrungen in beiden Prozessmodellierungssprachen aufweisen oder gleich viel Wissen in beiden Prozessmodellierungssprachen aufweisen.\newline




