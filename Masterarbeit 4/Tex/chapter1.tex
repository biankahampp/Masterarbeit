\chapter{Einleitung}\label{sec:chapter1}

In der heutigen Zeit werden die meisten Systeme von Software kontrolliert \cite{Puntambekar2007}. Diese weisen bei deren Entwicklung oftmals ein hohes Maß an Komplexität und Umfang auf. Diese Systeme müssen nicht nur eine hohe Qualität aufweisen, sondern auch in einer vorgegebenen Zeit zu vorgegebenen Kosten fertig gestellt werden \cite{Grechenig2010}. Aus diesem Grund ist es wichtig, bei der Entwicklung eines Systems einem effizienten Softwareprozessmodell zu folgen, da diese den Entwicklungsprozess strukturieren und dadurch beherrschbar machen, indem sie eine Menge von Aktivitäten vorgeben, welche zur Fertigstellung der Software notwendig sind \cite{richling2011autonomie}.\newline

Inzwischen existiert eine Reihe verschiedener Softwareprozessmodelle. Diese unterscheiden sich in leichtgewichtige (weniger formal, kaum Dokumentation) und schwergewichtige (sehr formale, dokumentenlastige Vorgehensweise) Prozessmodelle. Scrum ist ein Beispiel für ein leichtgewichtiges Prozessmodell, beim V-Modell XT handelt es sich um ein schwergewichtiges Prozessmodell und der Open Rational Unified Process (Open UP) befindet sich an einer Schnittstelle zwischen schwergewichtigen und leichtgewichtigen Prozessmodellen \cite{Hanser2010}.\newline

Um Softwareprozessmodelle richtig anzuwenden, müssen diese auch verstanden werden. Eine rein textuelle Beschreibung derselbigen ist oftmals sehr umfangreich (z.B. ist die Dokumentation des V-Modell XT mehr als 900 Seiten lang). Daher sollten diese in einer vereinfachten Art dargestellt werden. Hierfür bieten sich Prozessmodellierungssprachen an, da diese einerseits eine gewisse formale Exaktheit aufweisen und andererseits oftmals auch intuitiv verständlich sind \cite{thomas2009,kircher2006}. \newline

\section{Motivation}



Heutzutage existiert eine Vielzahl unterschiedlicher Prozessmodellierungssprachen. Über deren Vor- und Nachteile wird intensiv diskutiert. Hierbei werden auch sehr häufig die Vorzüge und Nachteile von imperativen und deklarativen Prozessmodellierungssprachen diskutiert \cite{fahland2009}. \newline

Es existieren bereits Arbeiten und Studien, welche sich mit dem Vergleich von imperativen und deklarativen Prozessmodellierungssprachen beschäftigen. Jedoch gibt es noch kaum Arbeiten, welche sich intensiv mit dem Vergleich der Anwendbarkeit der beiden Prozessmodellierungssprachen bei der Modellierung beschäftigen.\newline

Aus diesem Grund wird die Anwendbarkeit von imperativen und deklarativen Prozessmodellierungsansätzen in dieser Arbeit im Kontext von Softwareprozessmodellen eingehend untersucht. Hierfür werden Teile der Softwareprozessmodelle Scrum, Open UP und V-Modell XT sowohl in imperativer, als auch in deklarativer Prozessmodellierungssprache modelliert und anschließend wird deren Anwendbarkeit in diesem Kontext eingehend analysiert und diskutiert.\newline

.


\section{Zielstellung}
Die vorliegende Arbeit verfolgt mehrere Ziele. Einerseits soll dem Leser der vorliegenden Arbeit ein grundlegendes Wissen über Software Engineering und Softwareentwicklungsmodelle vermittelt werden. Weiterhin sollen ihm auch grundlegende Kenntnisse in den Bereichen Prozessmodellierung und Prozessmodellierungssprachen, insbesondere imperative und deklarative Prozessmodellierungssprachen beigebracht werden. \newline

Zudem sollen dem Leser die Softwareprozessmodelle Scrum, Open UP und V-Modell XT vorgestellt werden. Ein weiteres Ziel ist es hier, diese drei Modelle zu analysieren und somit für die nachfolgende Modellierung auf zu bereiten.\newline

Das nächste Ziel dieser Arbeit ist es sodann, Teile der Softwareprozessmodelle Scrum, Open UP und V-Modell XT in der imperativen Prozessmodellierungssprache BPMN und in der deklarativen Prozessmodellierungssprache ConDec zu modellieren. Hierdurch wird der Grundstein für den nachfolgenden Vergleich dieser Modelle gelegt.\newline

Das Hauptziel dieser Arbeit ist sodann der Vergleich der Anwendbarkeit der deklarativen und imperativen Prozessmodellierungssprachen. Hierfür werden die zuvor erstellten Modelle genauestens analysiert und die beiden Prozessmodellierungssprachen BPMN und ConDec werden dann bezüglich ihrer Eignung zum Modellieren verglichen.\newline

Ein weiteres Ziel ist die Validierung der Ergebnisse des durchgeführten Vergleichs. Die Validierung wird mit Hilfe einer Studie durchgeführt, bei welcher mehrere Studierende und Doktoranden der Informatik befragt werden.\newline





\section{Aufbau der Arbeit}

Eine Übersicht über den Aufbau dieser Arbeit gibt Abbildung.
Zunächst werden in Kapitel 2 und 3 grundlegende Begriffe erläutert, welche für das Verständnis der vorliegenden Arbeit notwendig sind.\newline

\textbf{Kapitel 2} liefert zunächst einen Einblick in Prozessmodelle. In Kapitel 2.1 wird der Begriff Software Engineering eingeführt und es werden die Ziele, der Prozess sowie die Prinzipien des Software Engineering vorgestellt. Anschließend werden in Kapitel 2.2 Softwareprozesse erläutert. Hierfür werden Software-Projekttypen sowie Schwergewichtige und Leichtgewichtige Prozessmodelle definiert.\newline

In \textbf{Kapitel 3} erfolgt eine Einführung in die Grundlagen der Modellierung. Zum Einen werden in Kapitel 3.1 Prozessmodellierung, die Ziele der Prozessmodellierung sowie die Grundsätze ordnungsgemäßer Modellierung vorgestellt. Zum Anderen erfolgt in Kapitel 3.2 eine allgemeine Einführung in Prozessmodellierungssprachen und vor allem werden die in dieser Arbeit verwendete imperative und deklarative Modellierung erklärt. Des Weiteren werden noch in Kapitel 3.3 die in der vorliegenden Arbeit verwendeten Modellierungswerkzeuge vorgestellt.\newline

Die Anforderungserhebung erfolgt in \textbf{Kapitel 4}. Hier werden in Kapitel 4.1 die Vergleichskriterien für die beiden Prozessmodellierungssprachen erläutert.\newline

Die imperative und deklarative Modellierung für Software Engineering Prozessmodelle erfolgt in \textbf{Kapitel 5}. Hier wird das Software Engineering Prozessmodell Scrum in Kapitel 5.1 zunächst eingeführt, anschließend analysiert, imperativ und deklarativ modelliert und die beiden Modellierungen werden miteinander verglichen. In Kapitel 5.2 wird zunächst der Open Unified Process (Open UP) vorgestellt, es folgt eine Analyse desselbigen und es werden imperative und deklarative Modelle des Open UP erstellt und miteinander verglichen. Das V-Modell XT wird in Kapitel 5.3 erläutert, analysiert, imperativ und deklarativ modelliert und die jeweiligen Modelle werden miteinander verglichen. In Kapitel 5.4 erfolgt ein insgesamter Vergleich der imperativen und deklarativen Modelle.\newline

Die Validierung der Ergebnisse aus Kapitel 5 wird in \textbf{Kapitel 6} durchgeführt. \newline

\textbf {Kapitel 7} widmet sich verwandten Arbeiten zur vorliegenden Arbeit. Zunächst werden in Kapitel 7.1 verwandte Arbeiten zur Modellierung von Software Engineering Prozessmodellen gegenüber der vorliegenden Arbeit abgegrenzt. Weiterhin werden in Kapitel 7.2 Arbeiten über die Verständlichkeit von Prozessmodellierungssprachen und in Kapitel 7.3 Arbeiten über den Vergleich von Prozessmodellierungssprachen dargelegt und der Thematik dieser Arbeit gegenüber gestellt.\newline

\textbf{Kapitel 8} fasst die gesamt Arbeit nochmals zusammen und gibt weiterhin einen Ausblick auf zukünftige Forschung in dieser Thematik.


