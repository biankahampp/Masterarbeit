\chapter{Einleitung}\label{sec:chapter1}



\section{Motivation}
Heutzutage existiert eine Vielzahl unterschiedlicher Prozessmodellierungssprachen.


\section{Zielstellung}

Die vorliegende Arbeit verfolgt mehrere Ziele. Einerseits soll dem Leser der vorliegenden Arbeit ein grundlegendes Wissen über Software Engineering und Softwareprozessmodelle vermittelt werden. Weiterhin soll ihm auch grundlegende Kenntnisse in den Bereichen Prozessmodellierung und Prozessmodellierungssprachen, insbesondere imperative und deklarative Prozessmodellierungssprachen beigebracht werden. \newline

Zudem sollen dem Leser die Softwareprozessmodelle Scrum, Open UP und V-Modell XT vorgestellt werden. Ein weiteres Ziel ist es hier, diese drei Modelle zu analysieren und somit für die nachfolgende Modellierung auf zu bereiten.\newline

Das nächste Ziel dieser Arbeit ist es sodann, Teile der Softwareprozessmodelle Scrum, Open UP und V-Modell XT in der imperativen Prozessmodellierungssprache BPMN und in der deklarativen Prozessmodellierungssprache ConDec zu modellieren. Hierdurch wird der Grundstein für den nachfolgenden Vergleich dieser Modelle gelegt.\newline

Das der Hauptziele dieser Arbeit ist sodann der Vergleich der Anwendbarkeit der deklarativen und imperativen Prozessmodellierungssprachen. Hierfür werden die zuvor erstellten Modelle genauestens analysiert und die beiden Proessmodellierungssprachen BPMN und ConDec werden dann bezüglich ihrer Eignung zum Modellieren verglichen.\newline

Ein weiteres Ziel ist die Validierung der Ergebnisse des durchgeführten Vergleichs. Die Validierung wird mit Hilfe einer Studie durchgeführt.\newline





\section{Aufbau der Arbeit}

Eine Übersicht über den Aufbau dieser Arbeit gibt Abbildung.
Zunächst werden in Kapitel 2 und 3 grundlegende Begriffe erläutert, welche für das Verständnis der vorliegenden Arbeit notwendig sind.\newline

Kapitel 2 liefert zunächst einen Einblick in Prozessmodelle. In Kapitel 2.1 wird der Begriff Software Engineering eingeführt und es werden die Ziele, der Prozess sowie die Prinzipien des Software Engineering vorgestellt. Anschließend werden in Kapitel 2.2 Softwareprozesse erläutert. Hierfür werden Software-Projekttypen sowie Schwergewichtige und Leichtgewichtige Prozessmodelle definiert.\newline

In Kapitel 3 erfolgt eine Einführung in die Grundlagen der Modellierung. Zum Einen werden Prozessmodellierung, die Ziele der Prozessmodellierung sowie die Grundsätze ordnungsgemäßer Modellierung vorgestellt. Zum Anderen erfolgt eine allgemeine Einführung in Prozessmodellierungssprachen und vor allem werden die in dieser Arbeit verwendete Imperative und Deklarative Modellierung erklärt. Des Weiteren werden noch die in der vorliegenden Arbeit verwendeten Modellierungswerkzeuge vorgestellt.\newline

Die Anforderungserhebung erfolgt in Kapitel 4. Hier werden in Kapitel 4.1 die Vergleichskriterien für die beiden Prozessmodellierungssprachen erläutert.\newline

Die imperative und deklarative Modellierung für Software Engineering Prozessmodelle erfolgt in Kapitel 5. Hier wird das Software Engineering Prozessmodell Scrum in Kapitel 5.1 zunächst eingeführt, anschließend analysiert, imperativ und deklarativ modelliert und die beiden Modellierungen werden miteinander verglichen. In Kapitel 5.2 wird zunächst der Open Unified Process (Open UP) vorgestellt, es folgt eine Analyse desselbigen und es werden imperative und deklarative Modelle des Open UP erstellt und miteinander verglichen. Das V-Modell XT wird in Kapitel 5.3 erläutert, analysiert, imperativ und deklarativ modelliert und die jeweiligen Modelle werden miteinander verglichen. In Kapitel 5.4 erfolgt ein insgesamter Vergleich der imperativen und deklarativen Modelle.\newline

Die Validierung der Ergebnisse aus Kapitel 5 wird in Kapitel 6 durchgeführt.
In Kapitel 7 widmet sich verwandten Arbeiten zur vorliegenden Arbeit. Zunächst werden in Kapitel 7.1 verwandte Arbeiten zur Modellierung von Software Engineering Prozessmodellen gegenüber der vorliegenden Arbeit abgegrenzt. Weiterhin werden in Kapitel 7.2 Arbeiten über die Verständlichkeit von Prozessmodellierungssprachen und in Kapitel 7.3 Arbeiten über den Vergleich von Prozessmodellierungssprachen dargelegt und der Thematik dieser Arbeit gegenüber gestellt.\newline

Kapitel 8 fasst die gesamt Arbeit nochmals zusammen und gibt weiterhin einen Ausblick auf zukünftige Forschung in dieser Thematik.


