
\chapter{Modellierung}\label{sec:chapter3}
Kapitel 3 liefert einen Überblick über die Grundlagen der Modellierung. Zunächst werden in Kapitel 3.1 die Grundlagen der Prozessmodellierung erläutert. Hierbei wird auf die Grundsätze ordnungsgemäßer Modellierung eingegangen. Anschließend werden in Kapitel 3.2 Prozessmodellierungssprachen diskutiert. Einerseits werden imperative Modellierungssprachen erklärt und es wird ein kurzer Einblick in die Prozessmodellierungssprache BPMN gegeben. Andererseits werden deklarative Prozessmodellierungssprachen beschrieben und es erfolgt ein Einblick in die deklarative Prozessmodellierungssprache ConDec. In Kapitel 3.3 werden die in dieser Arbeit für die imperative und deklarative Modellierung verwendeten Modellierungswerkzeuge vorgestellt.

\section{Prozessmodellierung}\label{sec:chapter3:Prozessmodellierung}

Prozessmodellierung hat den Zweck, Prozesse zu beschreiben \cite{fahland2010}. Ein Prozessmodell ist eine vereinfachte Darstellung eines Prozesses und besteht aus einer Abfolge von Tätigkeiten, welche chronologisch-sachlogisch angeordnet sind. Der Umfang und Detaillierungsgrad der Prozessmodelle kann sich je nach Zweck und Zielsetzung unterscheiden \cite{koch2011}.

\subsection{Ziele der Prozessmodellierung}
Mit der Modellierung von Prozessen werden verschiedene Ziele verfolgt. Eine erste Übersicht über die Ziele der Prozessmodellierung gibt Abbildung \ref{fig:ZieleProzess} \cite{koch2011}.
\begin{figure}[htp]
\begin{center}
  \includegraphics[width=\textwidth]{ZieleProzess} %pdf, jpg, png...
  \caption{Ziele der Prozessmodellierung nach \cite{koch2011}}
  \label{fig:ZieleProzess}
\end{center}
\end{figure}

Bei der \textit{Transparenz} geht es darum, dass alle Beteiligten am Prozess einsehen können, von wem welche Aufgaben durchgeführt werden. Weiterhin verfolgt die Prozessmodellierung das Ziel, durch \textit{Fehlervermeidung} die Qualität, Termintreue und Kundenzufriedenheit zu erhöhen. Durch die Modellierung eines Prozesses kann dieser genau analysiert werden und hierdurch können Einsparungspotenziale von \textit{Kosten} aufgedeckt werden. Indem die Abläufe in einem Unternehmen als Prozesse dargestellt werden, ist es möglich, eine \textit{personenunabhängige Verfügbarkeit des Wissens} zu erreichen, da das Wissen hierdurch allen Personen zugänglich gemacht wird, unabhängig davon, ob sie am Prozess beteiligt sind oder nicht. Die Prozessmodellierung führt zu einer \textit{erleichterten Einarbeitung neuer Mitarbeiter}. Durch die Darstellung der Tätigkeiten der einzelnen Mitarbeiter in Prozessmodellen wird ihnen ihr Beitrag zum Erfolg des Unternehmens vor Augen geführt was eine \textit{erhöhte Mitarbeitermotivation} zur Folge hat. Nach deren Erstellung gibt es verschiedene \textit{Auswertungsmöglichkeiten} für die Prozessmodelle. Durch die Modellierung von Prozessen werden etwaige Schwachstellen, wie z.B. Doppelarbeiten und Prozessverzögerungen offengelegt, wodurch eine \textit{Prozessoptimierung} möglich ist. Mit Hilfe von \textit{Simulationen} der Prozessmodelle lassen sich eventuelle Engpässe rechtzeitig erkennen. Die Voraussetzung für die \textit{Zertifizierung} nach DIN EN ISO 9000:1000 sind Prozessmodelle als Dokumentation. Basis für die Entwicklung von Softwaresystemen bilden Prozessmodelle, weshalb sie als \textit{Basis für die informationstechnische Unterstützung} dienen \cite{koch2011}.

\subsection{Grundsätze ordnungsgemäßer Modellierung}

Bei der Gestaltung eines Modelles sollten grundlegende Prinzipien beachtet werden, um die Qualität eines Modelles zu sichern. Hierfür gibt es die Grundsätze ordnungsgemäßer Modellierung, über deren Prinzipien Abbildung \ref{fig:Prinzipien} einen Überblick gibt  \cite{freund2007}.

\begin{figure}[htp]
\begin{center}
  \includegraphics[scale=0.6]{Prinzipien} %pdf, jpg, png...
  \caption{Grundsatz ordnungsgemäßer Modellierung nach \cite{journals95}}
  \label{fig:Prinzipien}
\end{center}
\end{figure}

Der \textit{Grundsatz der Richtigkeit} besitzt zwei verschiedene Ausprägungen: Eine syntaktische und eine semantische. Die syntaktische \textit{Richtigkeit} eines Modelles wird durch die Einhaltung der Notationsregeln der dem Modell zugrunde liegenden Prozessmodellierungssprache erreicht \cite{journals95, becker2012prozessmanagement}. \newline

Ein Modell wird als semantisch korrekt, oder auch formal korrekt bezeichnet, wenn es dem ihm zugrunde liegenden Metamodell gegenüber vollständig und konsistent ist, d.h. es gibt den abzubildenden Sachverhalt korrekt wieder. Hierbei muss einerseits auf die korrekte Abbildung der Struktur des Metamodelles, als auch des dort beschriebenen Verhaltens geachtet werden \cite{journals95, becker2012prozessmanagement}. \newline


Modelle werden üblicherweise in getrennten Sichten modelliert, um die Komplexität so gering wie möglich zu halten. Beispielsweise werden Prozesse in einem Prozessmodell, die Daten aber in einem Datenmodell modelliert. Werden bei einer Modellierung mehrere Sichten (z.B. Organisationssicht, Datensicht, Funktionssicht) modelliert, müssen diese auch ineinander integriert werden. Beim \textit{Grundsatz des systematischen Aufbau} geht es darum, bei der Modellierung auch auf die anderen Sichten zu achten, um eine spätere konsistente Integration der verschiedenen Sichten zu gewährleisten. Insbesondere ist zu vermeiden, dass die gleichen Informationsobjekte mehrmals mit jeweils verschiedenen Begriffen verwendet werden. Weiterhin sollten die Eingabedaten eines Prozessmodells einen Verweis auf bestehende Datenmodelle enthalten \cite{journals95, freund2007, becker2012prozessmanagement,koch2011}.\newline

Der \textit{Grundsatz der Relevanz} besagt, dass alle Elemente und Verknüpfungen eines Modells, ohne die der Nutzen des Modells sinken würde, für die Modellierung relevant sind \cite{journals95, reinshagen2009}. Auf der anderen Seite sollten aber auch nur diejenigen Teile der Realität in das Modell aufgenommen werden, die wirklich notwendig sind. Es sollte somit darauf geachtet werden, nur so viele Informationen ins Modell zu bringen wie minimal benötigt werden \cite{journals95, freund2007,reinshagen2009}.\newline


Durch den \textit{Grundsatz der Klarheit} soll sichergestellt werden, dass das Modell für den Adressaten verständlich ist. Es muss also bei der Modellierung auf Strukturiertheit, Verständlichkeit und Anschaulichkeit geachtet werden. Insbesondere sollte das Modell ohne besondere methodische Kenntnisse verständlich sein. Somit sollte die Modellierung entweder von links nach rechts oder von oben nach unten verlaufen, wobei darauf zu achten ist, dass sich Flusslinien und Kanten hierbei so wenig wie möglich überkreuzen. Weiterhin sollte die Anzahl der Elemente auf das Nötigste reduziert werden. Vor allem die Anzahl an Verzweigungen innerhalb eines Prozessmodells wirkt sich negativ auf die Verständlichkeit von Prozessmodellen aus. Ebenso hat eine hohe Anzahl von Verbindungen zwischen Aktivitäten einen negativen Einfluss auf das Verständnis \cite{leimeister2012,journals95, freund2007,reinshagen2009, becker2012prozessmanagement,koch2011,bpm07,thesis_maja}.\newline

Der \textit{Grundsatz der Wirtschaftlichkeit} sagt aus, dass die Modellierung kosteneffektiv durchzuführen ist \cite{leimeister2012}. Es gilt also abzuwägen, ob der Aufwand, der für die Modellierung notwendig ist, auch einen entsprechenden Nutzen bringt \cite{freund2007, journals95}.\newline

Wird in unterschiedlichen Modellen der gleiche Sachverhalt abgebildet, so sollten letztendlich auch vergleichbare Modelle entstehen, unabhängig von der verwendeten Modellierungssprache. Dies besagt der \textit{Grundsatz der Vergleichbarkeit}. Insbesondere ist auf einen einheitlichen Abstraktionsgrad der Prozessmodelle zu achten \cite{leimeister2012, journals95, freund2007,reinshagen2009}.\newline


\section{Prozessmodellierungssprachen}\label{sec:chapter3:Prozessmodellierungssprachen}

Die Modellierung eines Prozesses mit natürlicher Sprache bringt einige Nachteile mit sich, wie z.B. fehlende Eindeutigkeit, schwer zu überprüfende Vollständigkeit und teilweise Widersprüche. Mögliche Folgen davon können unterschiedliche Interpretationen, Missverständnisse und falsche Schlussfolgerungen sein. Eine reine Beschreibung der Prozessmodelle mit mathematischen Modellen und Formalismen führt jedoch oftmals zu einer Verminderung der intuitiven Verständlichkeit der Prozessmodelle. Aus diesem Grund ist es sinnvoll den Prozess graphisch als Diagramm mit einer Prozessmodellierungssprache darzustellen, da diese eine Schnittstelle zwischen formaler Exaktheit und intuitiver Verständlichkeit bilden \cite{thomas2009,kircher2006}.  \newline
Hierfür existieren eine Reihe verschiedener Prozessmodellierungssprachen, deren Vor- und Nachteile intensiv diskutiert werden. Ein viel erörterter Unterschied ist der zwischen imperativen und deklarativen Prozessmodellierungssprachen \cite{fahland2010}. \newline
Die ursprüngliche Unterscheidung zwischen imperativen und deklarativen Sprachen stammt aus der Programmierung. Während imperative Programmierung angibt, "Wie etwas zu tun ist", folgt deklarative Programmierung dem Ansatz \grqq Sag was benötigt wird und lass das System herausfinden, wie es erreicht werden kann\grqq \ \cite{pichler2012}.

\subsection{Imperative Modellierung}
Imperative Programmierung wird als zustandsbehaftete Programmierung bezeichnet, da das  Ergebnis einer Komponente nicht nur von ihren Argumenten abhängt, sondern auch von internen Parametern, was auch als ihr \grqq Zustand\grqq \  bezeichnet wird \cite{fahland2010}.  \newline
Ähnlich wie die imperative Programmierung, folgt auch die imperative Modellierung einem \grqq Inside-Out-Ansatz \grqq. Alle Ausführungsalternativen eines Prozesses sind somit in diesem spezifiziert und alle weiteren Ausführungsalternativen müssen explizit hinzugefügt werden. Bei der imperativen Modellierung werden Prozesse mit Operatoren und elementaren Aktivitäten modelliert. Hierbei können Sequenz, Parallelität und Synchronisation beschrieben werden \cite{kaschek1998}. Bei einer imperativen Modellierungssprache liegt der Fokus auf den ständigen Veränderungen der Prozess-Objekte.

\subsubsection {BPMN}

Die \textit{Business Process Modelling Notation (BPMN)} wurde von der \textit{Business Process Management Initiative} entwickelt und 2004 veröffentlicht. Seit 2005 wird sie von der \textit{Object Management Group} standardisiert und weiterentwickelt \cite{krallmann2013}.
Die BPMN Notationselemente lassen sich anhand der fünf Kategorien \textit{Swimlanes}, \textit{Flussobjekte}, \textit{verbindende Objekte}, \textit{Daten} und \textit{Artefakte} einteilen. Abbildung \ref{fig:BPMN} zeigt die Einteilung und die wichtigsten Prozess-Elemente von BPMN, welche nachfolgend genauer erläutert werden \cite{gpfert2012}. \newline

\begin{figure}[htp]
\begin{center}
  \includegraphics[scale=0.6]{BPMN} %pdf, jpg, png...
  \caption{BPMN-Elemente Übersicht nach \cite{gpfert2012}}
  \label{fig:BPMN}
\end{center}
\end{figure}

In der Kategorie \textbf{Swimlanes} befinden sich \textit{Pools} und \textit{Lanes}. \textit{Pools} stellen eine Art Container für den Prozess dar. Ein \textit{Pool} ist ein Prozessteilnehmer. Ein Prozessteilnehmer ist z.B. eine Organisationseinheit oder eine selbstständige Geschäftseinheit. Werden in einem Prozessmodell mehrere \textit{Pools} verwendet, so können hiermit Kollaborationen zwischen verschiedenen Prozessteilnehmern dargestellt werden. Ein \textit{Pool} kann in mehrere \textit{Lanes} unterteilt werden. \textit{Lanes} können untergeordnete Organisationseinheiten, Partnerrollen (z.B. Vertrieb, Projektleitung, Marketing) oder auch verschiedene Bestandteile eines Systems sein \cite{gpfert2012, pitschke2010, allweyer2013}. \newline
 \textit{Ereignisse}, \textit{Aktivitäten} und \textit{Gateways} befinden sich in der Kategorie \textbf{Flussobjekte}.
Start und Ende von Prozessen werden in BPMN durch \textit{Ereignisse} beschrieben. Diese werden in  \textit{Startereignisse} und \textit{Endereignisse} unterschieden und geben somit den Beginn und das Ende eines Prozesses an. Weiterhin gibt es auch noch \textit{Zwischenereignisse}. Hierdurch können beispielsweise Pausen im Prozess modelliert werden. Der Prozess stoppt in diesem Fall solange, bis ein bestimmtes Ereignis eintritt \cite{allweyer2013}. \newline
\textit{Aktivitäten} stellen Arbeitseinheiten dar und sind ein Oberbegriff für Aufgaben, Unterprozesse und Aufruf-Aktivitäten. Aufgaben sind Tätigkeiten, welche nicht weiter unterteilt werden können, während ein Unterprozess eine Aufgabe darstellt, welche in weitere Tätigkeiten unterteilt werden kann. Beschriftet werden sie mit einer Objekt-Verb-Verbindung (z.B. Lieferung überprüfen) \cite{gpfert2012}. \newline
Mit Hilfe von \textit{Gateways} lässt sich der Prozessablauf kontrollieren und steuern, da durch diese Verzweigungen und Zusammenführungen von Sequenzflüssen dargestellt werden. \cite{gpfert2012, allweyer2013}. Hierbei werden \textit{Exklusive Gateways} zur Modellierung alternativer Pfade, \textit{Parallele Gateways} zur Modellierung parallel ablaufender Pfade, \textit{Inklusive Gateways} zur Modellierung der Auswahl eines oder mehrerer Pfade und \textit{Komplexe Gateways} zur Modellierung komplexer Regeln bei Verzweigungen und Zusammenführungen, unterschieden \cite{allweyer2013}.\newline 

\begin{figure}[H]
\begin{center}
  \includegraphics[scale=0.5]{gates} %pdf, jpg, png...
  \caption{BPMN-Gateways}
  \label{fig:gates}
\end{center}
\end{figure} 


\textit{Nachrichtenfluss}, \textit{Assoziation}, \textit{Datenassoziation} und \textit{Sequenzfluss} bilden zusammen die Kategorie \textbf{Verbindende Objekte}. Ein \textit{Nachrichtenfluss} wird dazu verwendet, den Nachrichtenfluss zwischen zwei getrennten Prozessteilnehmern, z.B. aus zwei verschiedenen Unternehmen, darzustellen. Mit Hilfe einer \textit{Assoziation} können Daten, Text und andere Artefakte mit Flussobjekten verknüpft werden. Hiermit werden die Ein- und Ausgabe von Aktivitäten aufgezeigt. Ein \textit{Sequenzfluss} dient dazu, die Reihenfolge der Aktivitäten im Prozess festzulegen \cite{white2004introduction}. \newline
In der Kategorie \textbf{Daten} gibt es \textit{Datenobjekte}, \textit{Dateninput}, \textit{Datenoutput} und \textit{Datenspeicher}. \textit{Datenobjekte} geben hierbei an, welche Daten von den Aktivitäten benötigt werden, bzw. von diesen erzeugt werden \cite{white2004introduction}. Sie stellen somit Informationen dar, welche durch den Prozess fließen. Bei einem  \textit{Dateninput} handelt es sich um einen externen Input für den ganzen Prozess, der von einer Aktivität gelesen wird. Ein \textit{Datenoutput} hingegen wird als Ergebnis eines ganzen Prozesses erzeugt. Somit handelt es sich bei \textit{Dateninput}, bzw. \textit{Datenoutput} um Eingangs-, bzw. Ausgangsprozessschnittstellen \cite{bpmnposter}. Ein \textit{Datenspeicher} kann für den indirekten Austausch von Daten zwischen zwei verschiedenen Prozessteilnehmern verwendet werden. Hierfür ist es notwendig, dass beide Prozessteilnehmer Zugriff auf den \textit{Datenspeicher} haben \cite{allweyer2013}.\newline
Die Kategorie \textbf{Artefakte} beinhaltet \textit{Gruppierung}, \textit{Anmerkung} und \textit{Assoziation}. Diese ergänzen den Prozess um zusätzliche Informationen, haben jedoch keinerlei Einfluss auf diesen \cite{gpfert2012}. Eine \textit{Gruppierung} kann hierbei zur Dokumentation oder für Analysezwecke benutzt werden. Durch \textit{Anmerkungen} können dem Leser zusätzliche Informationen in Textform bereitgestellt werden \cite{white2004introduction}. Mit Hilfe einer \textit{Assoziation} lassen sich Datenobjekte mit Aktivitäten und Prozessen verknüpfen \cite{bpmnposter}. \newline
 Eine Übersicht über alle Elemente der BPMN Notation kann Anhang A entnommen werden.\newline





\subsection{Deklarative Modellierung}

 Die deklarative Modellierung folgt im Gegensatz zur imperativen Modellierung einem \grqq Outside-In-Ansatz\grqq \ \cite{lichtenegger2012}. Das heißt, deklarative Sprachen legen den Ablauf nicht im Vorhinein fest \cite{pichler2012} und sie sind somit sehr flexibel \cite{reichert2012}. Zu Beginn befinden sich nur die Aktivitäten im Prozessmodell und erlauben jegliches Ausführungsverhalten. Erst wenn Constraints zum Modell hinzugefügt werden, werden schrittweise Ausführungsalternativen verworfen \cite{pichler2012}. Constraints lassen sich hierbei in die beiden verschiedenen Kategorien \textbf{Ausführungsconstraints} und \textbf{Terminierungsconstraints} einteilen. Die Ausführungsconstraints geben Einschränkungen für die Ausführung von Aktivitäten an. Hierbei kann es sich z.B. um die Anzahl möglicher Ausführungen für eine Aktivität oder eine Mindestzeitverzögerung zwischen zwei Aktivitäten handeln. Terminierungsconstraints hingegen führen auf, wann eine korrekte Terminierung (Beendigung) des Prozesses möglich ist. Es kann hier z.B. vorgeschrieben werden, dass eine Aktivität mindestens einmal ausgeführt werden muss oder dass der Aktivität A Aktivität B folgen muss. Bevor dies nicht geschehen ist, kann der Prozess nicht korrekt enden \cite{reichert2012}.  Abbildung \ref{fig:Dec} zeigt ein Beispiel für ein deklaratives Prozessmodell. Es besteht aus den drei Aktivitäten A, B und C sowie aus zwei Constraints:  Das Constraint zwischen A und B legt fest, dass der Aktivität B die Aktivität A vorausgehen muss und das Constraint bei Aktivität C legt fest, dass diese mindestens einmal ausgeführt werden muss, aber beliebig oft durchgeführt werden kann. Abgesehen von diesen Bedingungen, können die Aktivitäten sowohl beliebig oft als auch in beliebiger Reihenfolge ausgeführt werden. Es wäre z.B. [A,B,C,C,A,B,C] eine korrekte Ausführungsreihenfolge. Die Ausführungsreihenfolgen [C,B,C,A] oder [A,B,A,B] wären jedoch inkorrekt, da bei der ersten Ausführungsreihenfolge B vor A ausgeführt wird und somit das Constraint zwischen diesen beiden Aktivitäten verletzt würde. Bei der zweiten Ausführungsreihenfolge wird Aktivität C nicht bearbeitet, wodurch das Constraint bei Aktivität C verletzt wird \cite{pesic2006}. \newline

\begin{figure}[H]
\begin{center}
  \includegraphics[scale=0.8]{DecProcessModel} %pdf, jpg, png...
  \caption{Deklarativer Beispiel-Prozess \cite{pesic2006}}
  \label{fig:Dec}
\end{center}
\end{figure} 

\subsubsection{ConDec}

Die deklarative Modellierungssprache ConDec wurde erstmals unter dem Namen DecSerFlow veröffentlicht \cite{fahland2010}. Mit ConDec lassen sich einerseits sehr strenge Modelle erstellen, welche den gesamten Prozess im Detail vorgeben und andererseits sehr leichtgewichtige Modelle, welche zwar angeben, welche Arbeit getan werden muss, aber nicht wie sie ausgeführt werden muss \cite{pesic2006}. \newline
In ConDec gibt es die vier verschiedenen Arten von Contraints: \textit{Existenz}, \textit{Choice}, \textit{Relation} und \textit{Negation}. Tabelle \ref{tab:tab3} zeigt die Bedeutung der verschiedenen Constraints. \newline

\begin{table}
\begin{tabular}{|p{0.5\textwidth}|p{0.5\textwidth}|}
\hline
\textbf{Constraint} & \textbf{Erläuterung}\\
\hline
Existenz Constraints & Ein-stellige Kardinalitäts-Constraints. Sie geben an, wie oft eine Aktivität ausgeführt werden kann bzw. muss.\\
\hline
Choice Constraints & N-stellige Constraints. Sie geben die Notwendigkeit der Ausführung von Aktivitäten an, die zu einer Reihe möglicher Alternativen gehören, unabhängig von anderen Constraints. \\
\hline
Relation Constraints & Zwei-stellige Constraints. Sie geben vor, dass eine gewisse Aktivität ausgeführt werden muss falls eine andere Aktivität ausgeführt wird. Es können auch qualitative zeitliche Constraints zwischen diesen beiden Aktivitäten verlangt werden.\\
\hline
Negation Constraints & Stellt die negative Version der Relation Constraints dar. Sie verbieten explizit die Ausführung einer gewissen Aktivität, wenn eine andere Aktivität ausgeführt wird.\\
\hline
 \end{tabular}
  \caption{Constraints ConDec \cite{pesic2006}}
\label{tab:tab3}
 \end{table}
 Eine Übersicht über die genaue Notation von ConDec ist in Anhang B verfügbar.


\section{Modellierungswerkzeuge}\label{sec:chapter3:Modellierungswerkzeuge}
Ein Modellierungswerkzeug ist ein Softwaresystem, mit dessen Hilfe sich Prozessmodelle erstellen  lassen. Teilweise bietet ein Modellierungswerkzeug noch weitere Funktionen wie z.B. das Ausführen und Monitoring der Prozesse, Simulationen und die Analyse von Prozessmodellen an. Die Ausführung der Prozessschritte kann hierbei durch die jeweilige Person, welche für die Aktivität zuständig ist, ausgeführt werden. Für die Prozessmodellierung in der vorliegenden Arbeit kommt das Modellierungswerkzeug Signavio für die imperative Modellierung mit BPMN und Declare für die deklarative Modellierung mit ConDec zum Einsatz. Diese beiden Modellierungswerkzeuge werden nachfolgend vorgestellt \cite{gadatsch2012}.

\subsection{Signavio}

Bei Signavio handelt es sich um ein webbasiertes Prozessmodellierungstool, welches auch das kollaborative Modellieren von Prozessen mit den Modellierungsstandards BPMN und EPC zulässt. Ein großer Vorteil von Signavio besteht darin, dass es nicht auf dem Rechner installiert werden muss, sondern direkt im Web-Browser ausgeführt werden kann. Die Prozessmodelle werden in einem zentralen Repository gespeichert und sind für die Benutzer entsprechend ihren Zugriffsrechten aufrufbar. Prozessmodelle besitzen alle eine eigene eindeutige URL und können über diese im Web-Browser aufgerufen werden. Hierbei wird auch gleich die Modellierungsumgebung mitgeladen und kann somit im Web-Browser ausgeführt werden \cite{quteprints}. Abbildung \ref{fig:Signavio} zeigt den \textit{Signavio Process Editor}.

\begin{figure}[H]
\begin{center}
  \includegraphics[scale=0.6]{Signavio} %pdf, jpg, png...
  \caption{Siganvio Process Editor (Screenshot Siganvio)}
  \label{fig:Signavio}
\end{center}
\end{figure} 

Links in Abbildung \ref{fig:Signavio} ist die BPMN Palette zu sehen. Die einzelnen Elemente können per \textit{Drag and Drop} in das Arbeitsdokument gezogen werden. Signavio verfügt über eine automatische Überprüfung von Modellierungskonventionen. Durch diese ist es möglich, das Modell auf die Einhaltung von Modellierungsrichtlinien, wie z.B. Notationsumfang, Benennung, Prozessstruktur und Diagrammlayout zu überprüfen. Die Modelle können alle als PDF exportiert werden. \newline
In Abbildung \ref{fig:Simulation} ist die Simulations-Sicht von Signavio zu sehen. Hier kann der Benutzer den Prozessablauf simulieren. Dies kann einerseits mit Benutzerinteraktion Schritt für Schritt erfolgen oder auch im Ganzen durch den Simulator gesteuert werden, wobei XOR-Verzweigungen nach wie vor vom Benutzer ausgewählt werden müssen.
\begin{figure}[H]
\begin{center}
  \includegraphics[scale=0.6]{Simulation} %pdf, jpg, png...
  \caption{Siganvio Simulation (Screenshot Signavio)}
  \label{fig:Simulation}
\end{center}
\end{figure} 



\subsection{Declare}

Declare wurde als Constraint-basiertes Workflow-Management-System entwickelt. Es wird für die Entwicklung von Prozessmodellen, welche auf deklarativen Sprachen basieren, benutzt. Declare bietet die folgenden Funktionen an \cite{pesic2007declare}:
\begin {itemize}
\item Modellentwicklung
\item Modellüberprüfung (Suche nach Fehlern in Modellen)
\item automatisierte Modellausführung
\item Modelle können zur Laufzeit geändert werden
\item Analyse der bereits ausgeführten Prozesse
\item Prozess Dekomposition
\end {itemize}

Abbildung \ref{fig:Declare} zeigt die Systemarchitektur von Declare.

\begin{figure}[H]
\begin{center}
  \includegraphics[scale=0.8]{Declare} %pdf, jpg, png...
  \caption{Declare Systemarchitektur nach \cite{pesic2007declare}}
  \label{fig:Declare}
\end{center}
\end{figure} 

Hieraus wird ersichtlich, dass \textit{Declare} mit den beiden Systemen \textit{YAWL} und \textit{ProM} kooperiert. Bei \textit{YAWL} handelt es sich um ein Workflow-Management System, welches auf strukturierte Workflows spezialisiert ist. Dies wirkt sich auf die Zusammenarbeit mit \textit{Declare} in der Art aus, dass die strukturierten Teile des Prozesses von \textit{YAWL} abgehandelt werden, während die unstrukturierten Teile von \textit{Declare} übernommen werden. Bei \textit{ProM} handelt es sich um ein Prozess-Mining-Tool. Hier werden bereits ausgeführte Prozesse von \textit{Declare} analysiert und darauf aufbauend werden dem Nutzer während der Prozessausführung Empfehlungen gegeben \cite{pesic2007declare}. \newline
Weiterhin besteht \textit{Declare} selbst aus drei Komponenten \textit{Framework}, \textit{Designer} und \textit{Worklist}.  Beim \textit{Designer} handelt es sich um ein Modellierungstool, welches für Systemeinstellungen und die Prozessmodell-Entwicklung verwendet wird (Abbildung \ref{fig:Designer}). Das \textit{Framework} ist für das Prozess-Enactment (Prozessbereitstellung) zuständig. Außerdem übernimmt es die Kommunikation mit \textit{YAWL} und \textit{ProM} und das Ändern der Prozessmodelle zur Laufzeit (Abbildung \ref{fig:Framework}). Die Prozessausführung wird von \textit{Worklist} durchgeführt. Hier können die Nutzer ihre zuvor erstellten Prozesse ausführen und können die von \textit{ProM} erstellten Empfehlungen sehen (Abbildung \ref{fig:Worklist}). Alle Modelle können als Bilddateien exportiert werden \cite{pesic2007declare}.



\begin{figure}[H]
\begin{center}
  \includegraphics[scale=0.4]{Designer} %pdf, jpg, png...
  \caption{Declare Designer (Screenshot aus Declare)}
  \label{fig:Designer}
\end{center}
\end{figure} 


\begin{figure}[H]
\begin{center}
  \includegraphics[scale=0.4]{Framework} %pdf, jpg, png...
  \caption{Declare Framework (Screenshot aus Declare)}
  \label{fig:Framework}
\end{center}
\end{figure} 

\begin{figure}[H]
\begin{center}
  \includegraphics[scale=0.4]{Worklist} %pdf, jpg, png...
  \caption{Declare Worklist (Screenshot aus Declare)}
  \label{fig:Worklist}
\end{center}
\end{figure} 





 









