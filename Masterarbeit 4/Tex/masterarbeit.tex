\documentclass{thesis}

% hier namen etc. einsetzen
\fullname{Bianka Hampp}
\email{bianka.hampp@uni-ulm.de}
\headline{Vergleich der Anwendbarkeit von\\deklarativen und imperativen\\Prozessmodellierungsansätzen \\ im Kontext von \\  Softwareentwicklungsprozessen}
\titel{Thema}
\jahr{JAHR}
\matnr{MATRIKEL NR}
\gutachterA{Gutachter 1}
%\gutachterB{Gutachter 2}
%\betreuer{Betreuer}
\typ{Masterarbeit }
\fakultaet{Ingenieurwissenschaften\\und Informatik}
\institut{Institut für Datenbanken und Informationssysteme}

% Falls keine Lizenz gewünscht wird bitte den folgenden Text entfernen.
% Die Lizenz erlaubt es zu nichtkommerziellen Zwecken die Arbeit zu
% vervielfältigen und Kopien zu machen. Dabei muss aber immer der Autor
% angegeben werden. Eine kommerzielle Verwertung ist für den Autor
% weiter möglich.
\license{
This work is licensed under the Creative Commons.
Attribution-NonCommercial-ShareAlike 3.0 License. To view a copy of this
license, visit http://creativecommons.org/licenses/by-nc-sa/3.0/de/ or send a
letter to Creative Commons, 543 Howard Street, 5th Floor, San Francisco,
California, 94105, USA. \\ Satz: PDF-\LaTeXe
}

\hypersetup{%
	pdftitle=\pdfescapestring{\thetitel},
	pdfauthor={\thefullname},
 	pdfsubject={\thetyp},
}


% trennungsregeln
\hyphenation{Sil-ben-trenn-ung}

\usepackage{rotating}

\begin{document}
\frontmatter
\maketitle
% impressum
\clearpage
\impressum

\cleardoublepage
% ab hier zeilenabstand 1,4 fach 10pt/14pt
\setstretch{1.4}

\section*{Kurzfassung}

Softwaresysteme sind heutzutage aus dem täglichen Leben nicht mehr weg zu denken. Sie finden sich in jedem elektronischen Gerät und kaum jemand kann n der heutigen Zeit auf Softwaresysteme verzichten. Da die Herstellung von Softwaresystemen mit einer sehr hohen Komplexität einhergeht, ist es wichtig, 
bei deren Erstellung einem Softwareentwicklungsprozess zu folgen. Denn Softwareentwicklungsprozesse geben Aktivitäten vor, welche zur Herstellung von Software notwendig. Dadurch helfen sie, die Entwicklung von Software zu strukturieren. Drei sehr bekannte Vertreter von Softwareentwicklungsprozessen sind Scrum, Open UP und das V-Modell XT.\newline
An der Entwicklung von Software sind oftmals eine Reihe verschiedener Personen mit unterschiedlichen fachlichen Hintergründen beteiligt. Diese müssen den vorgegeben Softwareentwicklungsprozess alle verstehen. Da eine rein textuelle Beschreibung der selbigen oftmals sehr umfangreich ist, bieten sich Prozessmodellierungssprachen zur Beschreibung von Softwareentwicklungsprozessen an da diese sowohl eine formale Korrektheit aufweisen aber auch intuitiv verständlich sind.\newline
Es existieren inzwischen eine ganze Reihe verschiedener Prozessmodellierungssprachen. Deren Vor- und Nachteile werden intensiv diskutiert. Hierbei wird auch der Unterschied zwischen deklarativen und imperativen Prozessmodellierungssprachen diskutiert.\newline
Aus diesem Grund werden in dieser Arbeit Teile der drei Softwareentwicklungsprozesse Scrum, Open UP und V-Modell XT sowohl in imperativer als auch in deklarativer Prozessmodellierungssprache modelliert und anhand der daraus entstehenden Modelle wird die Anwendbarkeit der imperativen und deklarativen Prozessmodellierungsansätzen verglichen.



\cleardoublepage
%\section*{Danksagung}

An dieser Stelle bedanke ich mich bei allen, die mich während meines Studiums und bei der Erstellung meiner Masterarbeit unterstützt haben.

Ich möchte Herrn Prof. Dr. Manfred Reichert danken, der es mir ermöglicht hat, diese Arbeit am Institut für Datenbanken und Informationssysteme zu schreiben.

Ein besonderer Dank gilt meinem Betreuer Herrn Gregor Grambow, der mir während des Verfassens meiner Masterarbeit unterstützend zur Seite stand und mir mit großer Hilfsbereitschaft, Korrekturlesen und Hinweisen bei der Erstellung dieser Arbeit geholfen hat.

Zudem danke ich allen Studenten und Doktoranden, die an meiner Studie teilgenommen haben und sich die Zeit genommen haben, meinen Fragebogen auszufüllen.

Außerdem möchte ich mich bei meinen Betreuern bei den Unternehmen Airbus Group, Nokia und Daimler AG bedanken, welche mir die letzten fünf Jahre durch Werkstudententätigkeiten die Möglichkeit gegeben haben, meine theoretischen Kenntnisse durch praktische Erfahrungen zu vertiefen und zu erweitern. 
An dieser Stelle danke ich auch meinen Arbeitskollegen in diesen Unternehmen für die gute Zusammenarbeit.

Des Weiteren bedanke ich mich bei Freunden und Kommilitonen der Universität Ulm für ihre moralische Unterstützung während des Studiums und beim Verfassen dieser Arbeit bedanken. Zudem möchte ich meinen Kommilitonen der Universität Ulm für die angenehme Studienzeit danken. Aus Platzgründen können diese hier nicht alle namentlich erwähnt werden.

Weiterhin danke ich meinen Eltern Vera und Rolf Hampp, meiner Schwester Frau Dr. Gabriele Hampp und meiner Patin Frau Ingeborg Grumer, die mich während des Studiums stets unterstützt haben.

% inhaltsverzeichnis einfügen
\tableofcontents

\mainmatter
% hier kommen die kapitel der arbeit
\chapter{Einleitung}\label{sec:chapter1}

In der heutigen Zeit sind Softwaresysteme aus dem täglichen Leben nicht mehr weg zu denken \cite{Puntambekar2007}. Diese zeigen bei deren Entwicklung oftmals ein hohes Maß an Komplexität und Umfang auf. Sie müssen nicht nur eine hohe Qualität aufweisen, sondern auch in einer vorgegebenen Zeit zu festgelegten Kosten fertig gestellt werden \cite{Grechenig2010}. Deshalb muss die Entwicklung von Software systematisch durchgeführt werden \cite{gumm2012einfuhrung}. Aus diesem Grund ist es wichtig, bei der Entwicklung eines Systems einem effizienten Softwareentwicklungsprozess zu folgen, da Softwareentwicklungsprozesse den Entwicklungsprozess strukturieren und dadurch beherrschbar machen, indem sie eine Menge von Aktivitäten vorgeben, welche zur Fertigstellung der Software notwendig sind \cite{richling2011autonomie}. Hierbei werden die grundlegenden Aktivitäten bei der Entwicklung eines Softwaresystems wie Planung, Spezifikation, Design, Implementierung und Test strukturiert \cite{gumm2012einfuhrung, Hanser2010}. \newline

Inzwischen existiert eine Reihe verschiedener Softwareentwicklungsprozesse. Diese unterscheiden sich in leichtgewichtige (weniger formal, kaum Dokumentation) und schwergewichtige (sehr formale, dokumentenlastige Vorgehensweise) Prozessmodelle. Scrum ist ein Beispiel für ein leichtgewichtiges Prozessmodell, beim V-Modell XT handelt es sich um ein schwergewichtiges Prozessmodell und der Open Rational Unified Process (Open UP) befindet sich an einer Schnittstelle zwischen schwergewichtigen und leichtgewichtigen Prozessmodellen \cite{Hanser2010}.\newline

Um Softwareentwicklungsprozesse richtig anzuwenden, müssen diese auch verstanden werden. Eine rein textuelle Beschreibung der selbigen ist oftmals sehr umfangreich. Daher sollten diese in einer vereinfachten Art dargestellt werden. Hierfür bieten sich Prozessmodellierungssprachen an, da diese einerseits eine gewisse formale Exaktheit aufweisen und andererseits oftmals auch intuitiv verständlich sind \cite{thomas2009,kircher2006}. \newline

\section{Motivation}



Heutzutage existiert eine Vielzahl unterschiedlicher Prozessmodellierungssprachen. Über deren Vor- und Nachteile wird intensiv diskutiert. Hierbei werden auch sehr häufig die Vorzüge und Nachteile von imperativen und deklarativen Prozessmodellierungssprachen beleuchtet \cite{fahland2010}. \newline

Es gibt bereits Arbeiten und Studien, welche den Vergleich von imperativen und deklarativen Prozessmodellierungssprachen untersuchen. Jedoch gibt es noch kaum Arbeiten, welche sich intensiv mit dem Vergleich der Anwendbarkeit der beiden Prozessmodellierungssprachen bei der Modellierung beschäftigen.\newline

Aus diesem Grund wird die Anwendbarkeit von imperativen und deklarativen Prozessmodellierungsansätzen in dieser Arbeit im Kontext von Softwareentwicklungsprozessen eingehend untersucht werden. Hierfür werden Teile der Softwareentwicklungsprozesse Scrum, Open UP und V-Modell XT sowohl in imperativer als auch in deklarativer Prozessmodellierungssprache modelliert und anschließend wird deren Anwendbarkeit in diesem Kontext analysiert und diskutiert.\newline



\section{Zielstellung}
Die vorliegende Arbeit verfolgt das Ziel, die Anwendbarkeit von deklarativen und imperativen Prozessmodellierungssprachen zu vergleichen. Hierfür soll dem Leser der vorliegenden Arbeit ein grundlegendes Wissen über Software Engineering und Softwareentwicklungsmodelle vermittelt werden. Weiterhin sollen ihm auch grundlegende Kenntnisse in den Bereichen Prozessmodellierung und Prozessmodellierungssprachen, insbesondere imperative und deklarative Prozessmodellierungssprachen beigebracht werden. \newline

Zudem soll eine Einführung des Lesers in die Softwareentwicklungsprozesse Scrum, Open UP und V-Modell XT erfolgen. Das Ziel ist es hier, diese drei Modelle zu analysieren und somit für die nachfolgende Modellierung aufzubereiten.\newline

Ein wichtiges Ziel dieser Arbeit ist es, Teile der Softwareentwicklungsprozesse Scrum, Open UP und V-Modell XT in der imperativen Prozessmodellierungssprache BPMN und in der deklarativen Prozessmodellierungssprache ConDec zu modellieren. Hierdurch wird der Grundstein für den nachfolgenden Vergleich dieser Modelle gelegt.\newline

Das Hauptziel dieser Arbeit ist der Vergleich der Anwendbarkeit der deklarativen und imperativen Prozessmodellierungssprachen. Hierfür werden die zuvor erstellten Modelle genauestens analysiert und die beiden Prozessmodellierungssprachen BPMN und ConDec werden dann bezüglich ihrer Eignung zum Modellieren miteinander verglichen.\newline

Ein weiteres Ziel ist die Validierung des Vergleiches. Die Validierung wird mit Hilfe einer Studie durchgeführt, bei welcher mehrere Studierende und Doktoranden der Informatik befragt werden.\newline





\section{Aufbau der Arbeit}

Eine Übersicht über den Aufbau dieser Arbeit gibt Abbildung \ref{fig:Aufbau}.
Zunächst werden in Kapitel 2 und 3 grundlegende Begriffe erläutert, welche für das Verständnis der vorliegenden Arbeit notwendig sind.\newline

Kapitel 2 liefert einen Einblick in Prozessmodelle. In Kapitel 2.1 wird der Begriff Software Engineering eingeführt. Anschließend werden in Kapitel 2.2 Softwareprozesse erläutert. Hierfür werden Software-Projekttypen sowie schwergewichtige und leichtgewichtige Prozessmodelle definiert.\newline

In Kapitel 3 erfolgt eine Einführung in die Grundlagen der Modellierung. Zum einen werden in Kapitel 3.1 Prozessmodellierung, die Ziele der Prozessmodellierung sowie die Grundsätze ordnungsgemäßer Modellierung vorgestellt. Zum anderen erfolgt in Kapitel 3.2 eine allgemeine Einführung in Prozessmodellierungssprachen und vor allem werden die in dieser Arbeit verwendete imperative und deklarative Modellierung erklärt. Des Weiteren werden noch in Kapitel 3.3 die in der vorliegenden Arbeit verwendeten Modellierungswerkzeuge beschrieben.\newline

Die Anforderungserhebung erfolgt in Kapitel 4. Hier werden in Kapitel 4.1 die Vergleichskriterien für die beiden Prozessmodellierungssprachen erläutert.\newline

Die imperative und deklarative Modellierung für Software Engineering Prozessmodelle erfolgt in \textbf{Kapitel 5}. Es wird das Software Engineering Prozessmodell Scrum in Kapitel 5.1 zunächst eingeführt, anschließend analysiert, imperativ und deklarativ modelliert und die imperativen und deklarativen Modellierungen werden miteinander verglichen. In Kapitel 5.2 wird dann der Open Unified Process (Open UP) vorgestellt, es folgt eine Analyse des selbigen und es werden imperative und deklarative Modelle des Open UP erstellt und miteinander verglichen. Das V-Modell XT wird in Kapitel 5.3 erläutert, analysiert, imperativ und deklarativ modelliert und die jeweiligen Modelle werden einander gegenüber gestellt. In Kapitel 5.4 erfolgt ein übergreifender Vergleich der imperativen und deklarativen Modelle.\newline

Die Validierung der Ergebnisse aus Kapitel 5 wird in Kapitel 6 durchgeführt. Zunächst werden in Kapitel 6.1 die Forschungsfragen vorgestellt. Anschließend wird in Kapitel 6.2 das Design der Studie erklärt und es wird in Kapitel 6.3 die Durchführung der Studie erläutert. Abschließend erfolgt in Kapitel 6.4 ein Fazit der Studie.  \newline

Kapitel 7 widmet sich verwandten Arbeiten zur vorliegenden Arbeit. Zunächst werden in Kapitel 7.1 verwandte Arbeiten zur Modellierung von Software Engineering Prozessmodellen gegenüber der vorliegenden Arbeit abgegrenzt. Weiterhin werden in Kapitel 7.2 Arbeiten über die Verständlichkeit von Prozessmodellierungssprachen und in Kapitel 7.3 Arbeiten über den Vergleich von Prozessmodellierungssprachen dargelegt und der Thematik dieser Arbeit gegenüber gestellt.\newline

Kapitel 8 fasst die gesamte Arbeit nochmals zusammen und gibt einen Ausblick auf zukünftige Forschung in dieser Thematik.

\begin{figure}[htp]
\begin{center}
  \includegraphics[scale=0.8]{Aufbau} %pdf, jpg, png...
  \caption{Aufbau der Arbeit}
  \label{fig:Aufbau}
\end{center}
\end{figure}


\chapter{Prozessmodelle}\label{sec:chapter2}

In Kapitel 2 werden grundlegende Konzepte des Software Engineering vorgestellt, welche notwenidg sind, um den Inhalt dieser Arbeit besser zu verstehen. Zunächst wird in Kapitel 2.1 der Begriff Software Engineering definiert und die Ziele, der Prozess und die Prinzipien des Software Engineering werden erläutert. Weiterhin wird in Kapitel 2.2 der Begriff Softwareprozessmodell erklärt. Hierbei werden Software-Projekttypen sowie schwergewichtige und leichtgewichtige Prozessmodelle beschrieben. Anschließend gibt es eine Einführung in die drei Softwareprozessmodelle Scrum, Open Unified Process und V-Modell-XT.

\section{Software Engineering}\label{sec:chapter2: Software Engineering}
Heutzutage werden immer mehr Systeme von Software kontrolliert. Dies macht Software Engineering zu einer der bedeutendsten Technologien \cite{Puntambekar2007}.
Unter Software versteht man laut Duden die "Gesamtheit aller Programme, die auf einem Computer eingesetzt werden können". Das Wort Engineering, welches sich laut Duden von dem lateinischen Wort Ingenium (=[schöpferische] Begabung; Erfindungsgabe) ableitet, wird heutzutage mit Ingenieurwesen, bzw. technische Entwicklung übersetzt. Software Engineering umfasst somit die Gesamtheit der Aktivitäten zur Analyse, Konzeption, Entwicklung und Implementierung einer softwaretechnischen Lösung \cite{Specker1998}.
Software Engineering besteht aus mehreren Schichten (Abbildung \ref{fig:SchichtenSE}):

\begin{figure}[htp]
\begin{center}
  \includegraphics[width=.5\linewidth]{SELayer} %pdf, jpg, png...
  \caption{Schichten des Software Engineering \cite{Puntambekar2007}}
  \label{fig:SchichtenSE}
\end{center}
\end{figure}

Somit sind für Software Engineering ein diszipliniertes Qualitätsmanagement sowie eine Prozessschicht vorhanden, um die termingerechte Ablieferung von Software zu gewährleisten. In der Methoden-Schicht wird sodann die Implementierung unter Zuhilfenahme von Requirementanalysen, Design und Programmierung durchgeführt. Hierbei werden Werkzeuge zur Automatisierung in SoftwareDokumenteprozessen benutzt \cite{Puntambekar2007}. 

\subsection{Ziele, Prozess und Prinzipien des Software Engineering}

Das Hauptziel in der Software Entwicklung ist, dass die Lösungen mit den Anforderungen übereinstimmen. Vollständige und konsistente Anforderungserhebungen sind, insbesondere für große Systeme, selten. Sowohl die Nutzer, als auch die Entwickler haben ein oftmals unvollständiges Verständnis des eigentlichen Problems und erheben somit ihre Anforderungen erst während der Entwicklung. Somit muss man mit Änderungen der Anforderungen an ein System während dessen Entwicklung rechnen. Aus diesem Grund ist es wichtig, Ziele beim Software Engineering zu haben, um die Auswirkungen solcher Änderungen einzudämmen \cite{Booch1993}.
Abbildung  \ref{fig:Wuerfel} zeigt die von \cite{ross1975software} definierten Ziele, Prinzipien und den Prozess des Software Engineering, welche nachfolgend genauer erläutert werden: 

\begin{figure}[htp]
\begin{center}
  \includegraphics[width=0.6\linewidth]{Wuerfel} %pdf, jpg, png...
  \caption{Ziele, Prozess und Prinzipien des Software Engineering  \cite{ross1975software}}
  \label{fig:Wuerfel}
\end{center}
\end{figure}

\subsubsection{Prinzipien des Software Engineering}


 Das \textit{Modluarisierungsprinzip} gibt eine geeignete Strukturierung für Softwaresysteme an. Das \textit{Abstraktionsprinzip} soll dabei helfen, sich von unwichtigen Details, welche für die zu entwickelnde Lösung irrelevant sind, zu lösen. Das \textit{Geheimnisprinzip} bezieht sich auf das Definieren und Durchsetzen von Zugriffsbeschränkungen. Das \textit{Lokalitätsprinzip} verlangt das räumlich zusammenhängende Ablegen von zusammengehörenden Informationen. Konsistenz wird durch das \textit{Gleichförmigkeitsprinzip} gewährleistet. Durch das \textit{Vollständigkeitsprinzip} wird sichergestellt, dass nichts vergessen wurde. Das Prinzip der \textit{Nachvollziehbarkeit} stellt sicher, dass Informationen, welche zur Überprüfung der Korrektheit benötigt werden, detailliert dargelegt werden \cite{ross1975software}.
 
 \subsubsection{Prozess des Software Engineering}

Wie Abbildung  \ref{fig:Wuerfel} entnommen werden kann, besteht der Prozess des Software Engineering aus 5 Schritten: Im ersten Schritt \textit{Zielsetzung} werden die Anforderungen an ein System festgelegt. Anschließend erfolgt im Schritt \textit{Konzept} die Ableitung der Software-Architektur, um die zuvor erhobenen Anforderungen zu erfüllen. Des Weiteren werden die Komponenten des Software-Systems festgelegt. Im dritten Schritt \textit {Mechanismus} erfolgt sodann die Implementierung des Software-Systems. Im darauffolgenden Schritt \textit{Notation} wird die Kommandosprache definiert, die ein Benutzer verwendet, um die Funktionalitäten des Software-Systems aufzurufen. Im letzten Schritt \textit{Gebrauch} muss noch die Bedienung des Systems, z.B. in Form eines Benutzerhandbuches, beschrieben werden \cite{ross1975software}.
  \subsubsection{Ziele des Software Engineering}

\textit{Modifizierbarkeit} ist das wohl schwierigste Ziel des Software Engineering. Hierbei geht es darum, dass es manchmal notwendig ist, Teile des zu entwickelnden Systems zu ändern, während andere Teile unverändert bleiben, aber dennoch das gewünschte neue Ergebnis erreicht wird. Auf die \textit{Effizienz} der jeweiligen Aktivitäten sollte immer geachtet werden, da dieses Ziel des Software Engineering häufig vernachlässigt wird. Bei dem Ziel \textit{Zuverlässigkeit} geht es darum, einerseits Fehler bei der Konzeption, im Design und der Implementierung zu vermeiden, andererseits muss auch Fehlverhalten bei der Ausführung und der Leistung verhindert werden.
 
\section{Softwareprozessmodelle}\label{sec:chapter.2: Softwareprozessmodelle}

Für das Verständnis, die Schaffung oder Unternehmung von etwas Großem, fertigen Menschen in der Regel ein vereinfachtes Bild davon an, bzw. nehmen Maß, fertigen eine Skizze oder einen Plan an oder orientieren sich an einem Vorbild, bzw. bauen sich eines. Dies geschieht normalerweise mit Papier und Schreibzeug, anderen Materialien oder einem Computer. Besonders für die Lösung von komplexen wissenschaftlichen Problemen oder für die Erfüllung großer Führungs- und Konstruktionsaufgaben ist dies unumgänglich \cite{Hesse2008}. \newline
Hierbei stützten sie sich auf Modelle, welche als Stellvertreter für die Sache, die verstanden, geschaffen, unternommen oder betrieben werden soll, angesehen werden kann \cite{Hesse2008}. \newline
Insbesondere die heutzutage von Softwareentwicklern zu erstellenden Softwareprodukte zeichnen sich durch ein hohes Maß an Komplexität und Umfang aus. Neben den Erwartungen von Kunden hinsichtlich Qualität müssen Softwaresysteme ebenfalls termingerecht und innerhalb eines vorgegebenen Budgetrahmens erstellt werden. Effektive und effiziente Softwareprozessmodelle gewinnen somit immer mehr Bedeutung \cite{Grechenig2010}.
Modell leitet sich von dem lateinischen Begriff  \glqq modelus \grqq
ab und kann mit  \glqq Regel, Form, Muster, Vorbild \grqq übersetzt werden \cite{Hesse2008}. 
Der Begriff Prozess stammt von dem lateinischen Wort "processus" ' ab und lässt sich mit "Fortgang oder Verlauf" ' übersetzen \cite{koch2011, Staud2006}. \newline 
Ein Softwareprozess stellt eine Abfolge von Schritten dar, welche zur Herstellung von Software notwendig sind \cite{Mishra2012, Stoerrle2005}. Mit Hilfe eines Softwareprozessmodelles lässt sich der organisatorische Rahmen zur Herstellung von Software beschreiben \cite{Koelmel2000}. Ein Softwareprozessmodell stellt somit ein Modell für die Entwicklung eines Software-Systems dar \cite{Hanser2010}. Die einzelnen Abschnitte eines Softwareprozesses werden hierbei als Phasen bezeichnet \cite{Stoerrle2005}. Diese werden unterscheiden in (Abbildung \ref{fig:SEProzess}):

\begin{figure}[htp]
\begin{center}
  \includegraphics[width= 0.8\linewidth]{Softwareprozess} %pdf, jpg, png...
  \caption{Phasen Softwareprozess nach \cite{Hanser2010}}
  \label{fig:SEProzess}
\end{center}
\end{figure}

In einem Softwareprozessmodell werden nicht nur die durchzuführenden Aktivitäten definiert, sondern auch die Rollen und Qualifikationen der Mitarbeiter, welche die jeweiligen Aktivitäten durchführen sollen, bzw. für diese verantwortlich sind. Des Weiteren werden die während des Entwicklungsprozesses zu erstellenden Dokumente und Unterlagen festgelegt \cite{Hanser2010}.

\subsection{Software-Projekttypen}

Software-Projekte lassen sich in drei Gruppen einteilen (Abbildung \ref{fig:Projekttypen}):

\begin{figure}[htp]
\begin{center}
  \includegraphics[width= 0.8\linewidth]{Projekttypen} %pdf, jpg, png...
  \caption{Software-Projekttypen nach \cite{Boehm81}}
  \label{fig:Projekttypen}
\end{center}
\end{figure}

Bei den \textit{Einfachen Projekten} sind relativ kleine Teams am Entwicklungsprozess beteiligt und bei den Teammitgliedern besteht räumliche Nähe. Jedes Teammitglied weist eine hohe methodische und fachliche Erfahrenheit auf und kennt sich in dem späteren Einsatzgebiet der Software gut aus. Die Anzahl der Code-Zeilen bei der zu entwickelnden Software ist meist gering \cite{Boehm81, Hanser2010}. \newline
Bei den \textit{Komplexen Projekten} handelt es sich um Software-Projekte, welche in den meisten Fällen stark durch behördliche Auflagen reguliert sind. Die Software muss einerseits eine hohe Zuverlässigkeit aufweisen und andererseits sind nachträgliche Änderungen fast nicht mehr möglich. Im Gegensatz zu den \textit{Einfachen Projekten} ist das Entwicklungsteam hier groß, besteht sowohl aus erfahrenen, als auch aus unerfahrenen Entwicklern und die Anzahl der Code-Zeilen ist ebenfalls groß \cite{Boehm81, Hanser2010}. \newline
Eine Schnittstelle zwischen diesen beiden Projekttypen bilden die \textit{Mittelschweren Projekte}. Hier sind die Software-Entwicklungsteams mittelgroß und bestehen aus erfahrenen und unerfahrenen Mitgliedern. Teilweise sind nicht alle Aspekte des Produktes schon im Vornherein bekannt und die Anzahl der Code-Zeilen ist groß \cite{Boehm81, Hanser2010}.

\subsection{Schwergewichtige und leichtgewichtige Prozessmodelle}

Aus der eben erfolgten Einteilung von Software-Projekten lässt sich eine Einteilung von Software-Prozessmodellen in \textit{leichtgewichtige} und \textit{schwergewichtige Prozessmodelle} ableiten \cite{Hanser2010}. \newline
\textit{Leichtgewichtige Prozessmodelle} eignen sich eher für kleine Teams, bei denen keine detaillierte Anforderungserhebung stattfindet, da die Kommunikation sowohl innerhalb des Teams, als auch mit dem Kunden auf Grund der kleinen Teamgröße gut funktioniert. Da viele Informationen hier informell über kurze Kommunikationswege weitergegeben werden, ist eine ausführliche Dokumentation derer nicht notwendig. Der Einsatz von \textit{leichtgewichtigen Prozessmodellen} eignet sich sehr gut für \textit{einfache Projekte} und teilweise auch für \textit{mittelschwere Projekte}\cite{Hanser2010}. \newline
Eine sehr formale und dokumentenlastige Vorgehensweise kommt bei den \textit{schwergewichtigen Prozessmodellen} zum Einsatz. Es findet eine ausführliche Dokumentation in allen Entwicklungsphasen statt und der Ablauf des Prozesses ist genau vorgegeben. Bei Software-Produkten, welche bei einer möglichen Fehlfunktion Menschenleben in Gefahr bringen, ist beispielsweise eine Vorgehensweise mit einem \textit{schwergewichtigen Prozessmodell} sinnvoll. Ihr Einsatz ist besonders in \textit{schweren Projekten} vorzuziehen, aber auch in \textit{mittelschweren Projekten} \cite{Hanser2010}. \newline















\chapter{Modellierung}\label{sec:chapter3}
Kapitel 3 liefert einen Überblick über die Grundlagen der Modellierung. Zunächst werden in Kapitel 3.1 die Grundlagen der Prozessmodellierung erläutert. Hierbei wird auf die Grundsätze ordnungsgemäßer Modellierung eingegangen. Anschließend werden in Kapitel 3.2 Prozessmodellierungssprachen diskutiert. Einerseits werden imperative Modellierungssprachen erklärt und es wird ein kurzer Einblick in die Prozessmodellierungssprache BPMN gegeben. Andererseits werden deklarative Prozessmodellierungssprachen beschrieben und es erfolgt ein Einblick in die deklarative Prozessmodellierungssprache ConDec. In Kapitel 3.3 werden die in dieser Arbeit für die imperative und deklarative Modellierung verwendeten Modellierungswerkzeuge vorgestellt.

\section{Prozessmodellierung}\label{sec:chapter3:Prozessmodellierung}

Prozessmodellierung hat den Zweck, Prozesse zu beschreiben \cite{fahland2010}. Ein Prozessmodell ist eine vereinfachte Darstellung eines Prozesses und besteht aus einer Abfolge von Tätigkeiten, welche chronologisch-sachlogisch angeordnet sind. Der Umfang und Detaillierungsgrad der Prozessmodelle kann sich je nach Zweck und Zielsetzung unterscheiden \cite{koch2011}.

\subsection{Ziele der Prozessmodellierung}
Mit der Modellierung von Prozessen werden verschiedene Ziele verfolgt. Eine erste Übersicht über die Ziele der Prozessmodellierung gibt Abbildung \ref{fig:ZieleProzess} \cite{koch2011}.
\begin{figure}[htp]
\begin{center}
  \includegraphics[width=\textwidth]{ZieleProzess} %pdf, jpg, png...
  \caption{Ziele der Prozessmodellierung nach \cite{koch2011}}
  \label{fig:ZieleProzess}
\end{center}
\end{figure}

Bei der \textit{Transparenz} geht es darum, dass alle Beteiligten am Prozess einsehen können, von wem welche Aufgaben durchgeführt werden. Weiterhin verfolgt die Prozessmodellierung das Ziel, durch \textit{Fehlervermeidung} die Qualität, Termintreue und Kundenzufriedenheit zu erhöhen. Durch die Modellierung eines Prozesses kann dieser genau analysiert werden und hierdurch können Einsparungspotenziale von \textit{Kosten} aufgedeckt werden. Indem die Abläufe in einem Unternehmen als Prozesse dargestellt werden, ist es möglich, eine \textit{personenunabhängige Verfügbarkeit des Wissens} zu erreichen, da das Wissen hierdurch allen Personen zugänglich gemacht wird, unabhängig davon, ob sie am Prozess beteiligt sind oder nicht. Die Prozessmodellierung führt zu einer \textit{erleichterten Einarbeitung neuer Mitarbeiter}. Durch die Darstellung der Tätigkeiten der einzelnen Mitarbeiter in Prozessmodellen wird ihnen ihr Beitrag zum Erfolg des Unternehmens vor Augen geführt was eine \textit{erhöhte Mitarbeitermotivation} zur Folge hat. Nach deren Erstellung gibt es verschiedene \textit{Auswertungsmöglichkeiten} für die Prozessmodelle. Durch die Modellierung von Prozessen werden etwaige Schwachstellen, wie z.B. Doppelarbeiten und Prozessverzögerungen offengelegt, wodurch eine \textit{Prozessoptimierung} möglich ist. Mit Hilfe von \textit{Simulationen} der Prozessmodelle lassen sich eventuelle Engpässe rechtzeitig erkennen. Die Voraussetzung für die \textit{Zertifizierung} nach DIN EN ISO 9000:1000 sind Prozessmodelle als Dokumentation. Basis für die Entwicklung von Softwaresystemen bilden Prozessmodelle, weshalb sie als \textit{Basis für die informationstechnische Unterstützung} dienen \cite{koch2011}.

\subsection{Grundsätze ordnungsgemäßer Modellierung}

Bei der Gestaltung eines Modelles sollten grundlegende Prinzipien beachtet werden, um die Qualität eines Modelles zu sichern. Hierfür gibt es die Grundsätze ordnungsgemäßer Modellierung, über deren Prinzipien Abbildung \ref{fig:Prinzipien} einen Überblick gibt  \cite{freund2007}.

\begin{figure}[htp]
\begin{center}
  \includegraphics[scale=0.6]{Prinzipien} %pdf, jpg, png...
  \caption{Grundsatz ordnungsgemäßer Modellierung nach \cite{journals95}}
  \label{fig:Prinzipien}
\end{center}
\end{figure}

Der \textit{Grundsatz der Richtigkeit} besitzt zwei verschiedene Ausprägungen: Eine syntaktische und eine semantische. Die syntaktische \textit{Richtigkeit} eines Modelles wird durch die Einhaltung der Notationsregeln der dem Modell zugrunde liegenden Prozessmodellierungssprache erreicht \cite{journals95, becker2012prozessmanagement}. \newline

Ein Modell wird als semantisch korrekt, oder auch formal korrekt bezeichnet, wenn es dem ihm zugrunde liegenden Metamodell gegenüber vollständig und konsistent ist, d.h. es gibt den abzubildenden Sachverhalt korrekt wieder. Hierbei muss einerseits auf die korrekte Abbildung der Struktur des Metamodelles, als auch des dort beschriebenen Verhaltens geachtet werden \cite{journals95, becker2012prozessmanagement}. \newline


Modelle werden üblicherweise in getrennten Sichten modelliert, um die Komplexität so gering wie möglich zu halten. Beispielsweise werden Prozesse in einem Prozessmodell, die Daten aber in einem Datenmodell modelliert. Werden bei einer Modellierung mehrere Sichten (z.B. Organisationssicht, Datensicht, Funktionssicht) modelliert, müssen diese auch ineinander integriert werden. Beim \textit{Grundsatz des systematischen Aufbau} geht es darum, bei der Modellierung auch auf die anderen Sichten zu achten, um eine spätere konsistente Integration der verschiedenen Sichten zu gewährleisten. Insbesondere ist zu vermeiden, dass die gleichen Informationsobjekte mehrmals mit jeweils verschiedenen Begriffen verwendet werden. Weiterhin sollten die Eingabedaten eines Prozessmodells einen Verweis auf bestehende Datenmodelle enthalten \cite{journals95, freund2007, becker2012prozessmanagement,koch2011}.\newline

Der \textit{Grundsatz der Relevanz} besagt, dass alle Elemente und Verknüpfungen eines Modells, ohne die der Nutzen des Modells sinken würde, für die Modellierung relevant sind \cite{journals95, reinshagen2009}. Auf der anderen Seite sollten aber auch nur diejenigen Teile der Realität in das Modell aufgenommen werden, die wirklich notwendig sind. Es sollte somit darauf geachtet werden, nur so viele Informationen ins Modell zu bringen wie minimal benötigt werden \cite{journals95, freund2007,reinshagen2009}.\newline


Durch den \textit{Grundsatz der Klarheit} soll sichergestellt werden, dass das Modell für den Adressaten verständlich ist. Es muss also bei der Modellierung auf Strukturiertheit, Verständlichkeit und Anschaulichkeit geachtet werden. Insbesondere sollte das Modell ohne besondere methodische Kenntnisse verständlich sein. Somit sollte die Modellierung entweder von links nach rechts oder von oben nach unten verlaufen, wobei darauf zu achten ist, dass sich Flusslinien und Kanten hierbei so wenig wie möglich überkreuzen. Weiterhin sollte die Anzahl der Elemente auf das Nötigste reduziert werden. Vor allem die Anzahl an Verzweigungen innerhalb eines Prozessmodells wirkt sich negativ auf die Verständlichkeit von Prozessmodellen aus. Ebenso hat eine hohe Anzahl von Verbindungen zwischen Aktivitäten einen negativen Einfluss auf das Verständnis \cite{leimeister2012,journals95, freund2007,reinshagen2009, becker2012prozessmanagement,koch2011,bpm07,thesis_maja}.\newline

Der \textit{Grundsatz der Wirtschaftlichkeit} sagt aus, dass die Modellierung kosteneffektiv durchzuführen ist \cite{leimeister2012}. Es gilt also abzuwägen, ob der Aufwand, der für die Modellierung notwendig ist, auch einen entsprechenden Nutzen bringt \cite{freund2007, journals95}.\newline

Wird in unterschiedlichen Modellen der gleiche Sachverhalt abgebildet, so sollten letztendlich auch vergleichbare Modelle entstehen, unabhängig von der verwendeten Modellierungssprache. Dies besagt der \textit{Grundsatz der Vergleichbarkeit}. Insbesondere ist auf einen einheitlichen Abstraktionsgrad der Prozessmodelle zu achten \cite{leimeister2012, journals95, freund2007,reinshagen2009}.\newline


\section{Prozessmodellierungssprachen}\label{sec:chapter3:Prozessmodellierungssprachen}

Die Modellierung eines Prozesses mit natürlicher Sprache bringt einige Nachteile mit sich, wie z.B. fehlende Eindeutigkeit, schwer zu überprüfende Vollständigkeit und teilweise Widersprüche. Mögliche Folgen davon können unterschiedliche Interpretationen, Missverständnisse und falsche Schlussfolgerungen sein. Eine reine Beschreibung der Prozessmodelle mit mathematischen Modellen und Formalismen führt jedoch oftmals zu einer Verminderung der intuitiven Verständlichkeit der Prozessmodelle. Aus diesem Grund ist es sinnvoll den Prozess graphisch als Diagramm mit einer Prozessmodellierungssprache darzustellen, da diese eine Schnittstelle zwischen formaler Exaktheit und intuitiver Verständlichkeit bilden \cite{thomas2009,kircher2006}.  \newline
Hierfür existieren eine Reihe verschiedener Prozessmodellierungssprachen, deren Vor- und Nachteile intensiv diskutiert werden. Ein viel erörterter Unterschied ist der zwischen imperativen und deklarativen Prozessmodellierungssprachen \cite{fahland2010}. \newline
Die ursprüngliche Unterscheidung zwischen imperativen und deklarativen Sprachen stammt aus der Programmierung. Während imperative Programmierung angibt, "Wie etwas zu tun ist", folgt deklarative Programmierung dem Ansatz \grqq Sag was benötigt wird und lass das System herausfinden, wie es erreicht werden kann\grqq \ \cite{pichler2012}.

\subsection{Imperative Modellierung}
Imperative Programmierung wird als zustandsbehaftete Programmierung bezeichnet, da das  Ergebnis einer Komponente nicht nur von ihren Argumenten abhängt, sondern auch von internen Parametern, was auch als ihr \grqq Zustand\grqq \  bezeichnet wird \cite{fahland2010}.  \newline
Ähnlich wie die imperative Programmierung, folgt auch die imperative Modellierung einem \grqq Inside-Out-Ansatz \grqq. Alle Ausführungsalternativen eines Prozesses sind somit in diesem spezifiziert und alle weiteren Ausführungsalternativen müssen explizit hinzugefügt werden. Bei der imperativen Modellierung werden Prozesse mit Operatoren und elementaren Aktivitäten modelliert. Hierbei können Sequenz, Parallelität und Synchronisation beschrieben werden \cite{kaschek1998}. Bei einer imperativen Modellierungssprache liegt der Fokus auf den ständigen Veränderungen der Prozess-Objekte.

\subsubsection {BPMN}

Die \textit{Business Process Modelling Notation (BPMN)} wurde von der \textit{Business Process Management Initiative} entwickelt und 2004 veröffentlicht. Seit 2005 wird sie von der \textit{Object Management Group} standardisiert und weiterentwickelt \cite{krallmann2013}.
Die BPMN Notationselemente lassen sich anhand der fünf Kategorien \textit{Swimlanes}, \textit{Flussobjekte}, \textit{verbindende Objekte}, \textit{Daten} und \textit{Artefakte} einteilen. Abbildung \ref{fig:BPMN} zeigt die Einteilung und die wichtigsten Prozess-Elemente von BPMN, welche nachfolgend genauer erläutert werden \cite{gpfert2012}. \newline

\begin{figure}[htp]
\begin{center}
  \includegraphics[scale=0.6]{BPMN} %pdf, jpg, png...
  \caption{BPMN-Elemente Übersicht nach \cite{gpfert2012}}
  \label{fig:BPMN}
\end{center}
\end{figure}

In der Kategorie \textbf{Swimlanes} befinden sich \textit{Pools} und \textit{Lanes}. \textit{Pools} stellen eine Art Container für den Prozess dar. Ein \textit{Pool} ist ein Prozessteilnehmer. Ein Prozessteilnehmer ist z.B. eine Organisationseinheit oder eine selbstständige Geschäftseinheit. Werden in einem Prozessmodell mehrere \textit{Pools} verwendet, so können hiermit Kollaborationen zwischen verschiedenen Prozessteilnehmern dargestellt werden. Ein \textit{Pool} kann in mehrere \textit{Lanes} unterteilt werden. \textit{Lanes} können untergeordnete Organisationseinheiten, Partnerrollen (z.B. Vertrieb, Projektleitung, Marketing) oder auch verschiedene Bestandteile eines Systems sein \cite{gpfert2012, pitschke2010, allweyer2013}. \newline
 \textit{Ereignisse}, \textit{Aktivitäten} und \textit{Gateways} befinden sich in der Kategorie \textbf{Flussobjekte}.
Start und Ende von Prozessen werden in BPMN durch \textit{Ereignisse} beschrieben. Diese werden in  \textit{Startereignisse} und \textit{Endereignisse} unterschieden und geben somit den Beginn und das Ende eines Prozesses an. Weiterhin gibt es auch noch \textit{Zwischenereignisse}. Hierdurch können beispielsweise Pausen im Prozess modelliert werden. Der Prozess stoppt in diesem Fall solange, bis ein bestimmtes Ereignis eintritt \cite{allweyer2013}. \newline
\textit{Aktivitäten} stellen Arbeitseinheiten dar und sind ein Oberbegriff für Aufgaben, Unterprozesse und Aufruf-Aktivitäten. Aufgaben sind Tätigkeiten, welche nicht weiter unterteilt werden können, während ein Unterprozess eine Aufgabe darstellt, welche in weitere Tätigkeiten unterteilt werden kann. Beschriftet werden sie mit einer Objekt-Verb-Verbindung (z.B. Lieferung überprüfen) \cite{gpfert2012}. \newline
Mit Hilfe von \textit{Gateways} lässt sich der Prozessablauf kontrollieren und steuern, da durch diese Verzweigungen und Zusammenführungen von Sequenzflüssen dargestellt werden. \cite{gpfert2012, allweyer2013}. Hierbei werden \textit{Exklusive Gateways} zur Modellierung alternativer Pfade, \textit{Parallele Gateways} zur Modellierung parallel ablaufender Pfade, \textit{Inklusive Gateways} zur Modellierung der Auswahl eines oder mehrerer Pfade und \textit{Komplexe Gateways} zur Modellierung komplexer Regeln bei Verzweigungen und Zusammenführungen, unterschieden \cite{allweyer2013}.\newline 

\begin{figure}[H]
\begin{center}
  \includegraphics[scale=0.5]{gates} %pdf, jpg, png...
  \caption{BPMN-Gateways}
  \label{fig:gates}
\end{center}
\end{figure} 


\textit{Nachrichtenfluss}, \textit{Assoziation}, \textit{Datenassoziation} und \textit{Sequenzfluss} bilden zusammen die Kategorie \textbf{Verbindende Objekte}. Ein \textit{Nachrichtenfluss} wird dazu verwendet, den Nachrichtenfluss zwischen zwei getrennten Prozessteilnehmern, z.B. aus zwei verschiedenen Unternehmen, darzustellen. Mit Hilfe einer \textit{Assoziation} können Daten, Text und andere Artefakte mit Flussobjekten verknüpft werden. Hiermit werden die Ein- und Ausgabe von Aktivitäten aufgezeigt. Ein \textit{Sequenzfluss} dient dazu, die Reihenfolge der Aktivitäten im Prozess festzulegen \cite{white2004introduction}. \newline
In der Kategorie \textbf{Daten} gibt es \textit{Datenobjekte}, \textit{Dateninput}, \textit{Datenoutput} und \textit{Datenspeicher}. \textit{Datenobjekte} geben hierbei an, welche Daten von den Aktivitäten benötigt werden, bzw. von diesen erzeugt werden \cite{white2004introduction}. Sie stellen somit Informationen dar, welche durch den Prozess fließen. Bei einem  \textit{Dateninput} handelt es sich um einen externen Input für den ganzen Prozess, der von einer Aktivität gelesen wird. Ein \textit{Datenoutput} hingegen wird als Ergebnis eines ganzen Prozesses erzeugt. Somit handelt es sich bei \textit{Dateninput}, bzw. \textit{Datenoutput} um Eingangs-, bzw. Ausgangsprozessschnittstellen \cite{bpmnposter}. Ein \textit{Datenspeicher} kann für den indirekten Austausch von Daten zwischen zwei verschiedenen Prozessteilnehmern verwendet werden. Hierfür ist es notwendig, dass beide Prozessteilnehmer Zugriff auf den \textit{Datenspeicher} haben \cite{allweyer2013}.\newline
Die Kategorie \textbf{Artefakte} beinhaltet \textit{Gruppierung}, \textit{Anmerkung} und \textit{Assoziation}. Diese ergänzen den Prozess um zusätzliche Informationen, haben jedoch keinerlei Einfluss auf diesen \cite{gpfert2012}. Eine \textit{Gruppierung} kann hierbei zur Dokumentation oder für Analysezwecke benutzt werden. Durch \textit{Anmerkungen} können dem Leser zusätzliche Informationen in Textform bereitgestellt werden \cite{white2004introduction}. Mit Hilfe einer \textit{Assoziation} lassen sich Datenobjekte mit Aktivitäten und Prozessen verknüpfen \cite{bpmnposter}. \newline
 Eine Übersicht über alle Elemente der BPMN Notation kann Anhang A entnommen werden.\newline





\subsection{Deklarative Modellierung}

 Die deklarative Modellierung folgt im Gegensatz zur imperativen Modellierung einem \grqq Outside-In-Ansatz\grqq \ \cite{lichtenegger2012}. Das heißt, deklarative Sprachen legen den Ablauf nicht im Vorhinein fest \cite{pichler2012} und sie sind somit sehr flexibel \cite{reichert2012}. Zu Beginn befinden sich nur die Aktivitäten im Prozessmodell und erlauben jegliches Ausführungsverhalten. Erst wenn Constraints zum Modell hinzugefügt werden, werden schrittweise Ausführungsalternativen verworfen \cite{pichler2012}. Constraints lassen sich hierbei in die beiden verschiedenen Kategorien \textbf{Ausführungsconstraints} und \textbf{Terminierungsconstraints} einteilen. Die Ausführungsconstraints geben Einschränkungen für die Ausführung von Aktivitäten an. Hierbei kann es sich z.B. um die Anzahl möglicher Ausführungen für eine Aktivität oder eine Mindestzeitverzögerung zwischen zwei Aktivitäten handeln. Terminierungsconstraints hingegen führen auf, wann eine korrekte Terminierung (Beendigung) des Prozesses möglich ist. Es kann hier z.B. vorgeschrieben werden, dass eine Aktivität mindestens einmal ausgeführt werden muss oder dass der Aktivität A Aktivität B folgen muss. Bevor dies nicht geschehen ist, kann der Prozess nicht korrekt enden \cite{reichert2012}.  Abbildung \ref{fig:Dec} zeigt ein Beispiel für ein deklaratives Prozessmodell. Es besteht aus den drei Aktivitäten A, B und C sowie aus zwei Constraints:  Das Constraint zwischen A und B legt fest, dass der Aktivität B die Aktivität A vorausgehen muss und das Constraint bei Aktivität C legt fest, dass diese mindestens einmal ausgeführt werden muss, aber beliebig oft durchgeführt werden kann. Abgesehen von diesen Bedingungen, können die Aktivitäten sowohl beliebig oft als auch in beliebiger Reihenfolge ausgeführt werden. Es wäre z.B. [A,B,C,C,A,B,C] eine korrekte Ausführungsreihenfolge. Die Ausführungsreihenfolgen [C,B,C,A] oder [A,B,A,B] wären jedoch inkorrekt, da bei der ersten Ausführungsreihenfolge B vor A ausgeführt wird und somit das Constraint zwischen diesen beiden Aktivitäten verletzt würde. Bei der zweiten Ausführungsreihenfolge wird Aktivität C nicht bearbeitet, wodurch das Constraint bei Aktivität C verletzt wird \cite{pesic2006}. \newline

\begin{figure}[H]
\begin{center}
  \includegraphics[scale=0.8]{DecProcessModel} %pdf, jpg, png...
  \caption{Deklarativer Beispiel-Prozess \cite{pesic2006}}
  \label{fig:Dec}
\end{center}
\end{figure} 

\subsubsection{ConDec}

Die deklarative Modellierungssprache ConDec wurde erstmals unter dem Namen DecSerFlow veröffentlicht \cite{fahland2010}. Mit ConDec lassen sich einerseits sehr strenge Modelle erstellen, welche den gesamten Prozess im Detail vorgeben und andererseits sehr leichtgewichtige Modelle, welche zwar angeben, welche Arbeit getan werden muss, aber nicht wie sie ausgeführt werden muss \cite{pesic2006}. \newline
In ConDec gibt es die vier verschiedenen Arten von Contraints: \textit{Existenz}, \textit{Choice}, \textit{Relation} und \textit{Negation}. Tabelle \ref{tab:tab3} zeigt die Bedeutung der verschiedenen Constraints. \newline

\begin{table}
\begin{tabular}{|p{0.5\textwidth}|p{0.5\textwidth}|}
\hline
\textbf{Constraint} & \textbf{Erläuterung}\\
\hline
Existenz Constraints & Ein-stellige Kardinalitäts-Constraints. Sie geben an, wie oft eine Aktivität ausgeführt werden kann bzw. muss.\\
\hline
Choice Constraints & N-stellige Constraints. Sie geben die Notwendigkeit der Ausführung von Aktivitäten an, die zu einer Reihe möglicher Alternativen gehören, unabhängig von anderen Constraints. \\
\hline
Relation Constraints & Zwei-stellige Constraints. Sie geben vor, dass eine gewisse Aktivität ausgeführt werden muss falls eine andere Aktivität ausgeführt wird. Es können auch qualitative zeitliche Constraints zwischen diesen beiden Aktivitäten verlangt werden.\\
\hline
Negation Constraints & Stellt die negative Version der Relation Constraints dar. Sie verbieten explizit die Ausführung einer gewissen Aktivität, wenn eine andere Aktivität ausgeführt wird.\\
\hline
 \end{tabular}
  \caption{Constraints ConDec \cite{pesic2006}}
\label{tab:tab3}
 \end{table}
 Eine Übersicht über die genaue Notation von ConDec ist in Anhang B verfügbar.


\section{Modellierungswerkzeuge}\label{sec:chapter3:Modellierungswerkzeuge}
Ein Modellierungswerkzeug ist ein Softwaresystem, mit dessen Hilfe sich Prozessmodelle erstellen  lassen. Teilweise bietet ein Modellierungswerkzeug noch weitere Funktionen wie z.B. das Ausführen und Monitoring der Prozesse, Simulationen und die Analyse von Prozessmodellen an. Die Ausführung der Prozessschritte kann hierbei durch die jeweilige Person, welche für die Aktivität zuständig ist, ausgeführt werden. Für die Prozessmodellierung in der vorliegenden Arbeit kommt das Modellierungswerkzeug Signavio für die imperative Modellierung mit BPMN und Declare für die deklarative Modellierung mit ConDec zum Einsatz. Diese beiden Modellierungswerkzeuge werden nachfolgend vorgestellt \cite{gadatsch2012}.

\subsection{Signavio}

Bei Signavio handelt es sich um ein webbasiertes Prozessmodellierungstool, welches auch das kollaborative Modellieren von Prozessen mit den Modellierungsstandards BPMN und EPC zulässt. Ein großer Vorteil von Signavio besteht darin, dass es nicht auf dem Rechner installiert werden muss, sondern direkt im Web-Browser ausgeführt werden kann. Die Prozessmodelle werden in einem zentralen Repository gespeichert und sind für die Benutzer entsprechend ihren Zugriffsrechten aufrufbar. Prozessmodelle besitzen alle eine eigene eindeutige URL und können über diese im Web-Browser aufgerufen werden. Hierbei wird auch gleich die Modellierungsumgebung mitgeladen und kann somit im Web-Browser ausgeführt werden \cite{quteprints}. Abbildung \ref{fig:Signavio} zeigt den \textit{Signavio Process Editor}.

\begin{figure}[H]
\begin{center}
  \includegraphics[scale=0.6]{Signavio} %pdf, jpg, png...
  \caption{Siganvio Process Editor (Screenshot Siganvio)}
  \label{fig:Signavio}
\end{center}
\end{figure} 

Links in Abbildung \ref{fig:Signavio} ist die BPMN Palette zu sehen. Die einzelnen Elemente können per \textit{Drag and Drop} in das Arbeitsdokument gezogen werden. Signavio verfügt über eine automatische Überprüfung von Modellierungskonventionen. Durch diese ist es möglich, das Modell auf die Einhaltung von Modellierungsrichtlinien, wie z.B. Notationsumfang, Benennung, Prozessstruktur und Diagrammlayout zu überprüfen. Die Modelle können alle als PDF exportiert werden. \newline
In Abbildung \ref{fig:Simulation} ist die Simulations-Sicht von Signavio zu sehen. Hier kann der Benutzer den Prozessablauf simulieren. Dies kann einerseits mit Benutzerinteraktion Schritt für Schritt erfolgen oder auch im Ganzen durch den Simulator gesteuert werden, wobei XOR-Verzweigungen nach wie vor vom Benutzer ausgewählt werden müssen.
\begin{figure}[H]
\begin{center}
  \includegraphics[scale=0.6]{Simulation} %pdf, jpg, png...
  \caption{Siganvio Simulation (Screenshot Signavio)}
  \label{fig:Simulation}
\end{center}
\end{figure} 



\subsection{Declare}

Declare wurde als Constraint-basiertes Workflow-Management-System entwickelt. Es wird für die Entwicklung von Prozessmodellen, welche auf deklarativen Sprachen basieren, benutzt. Declare bietet die folgenden Funktionen an \cite{pesic2007declare}:
\begin {itemize}
\item Modellentwicklung
\item Modellüberprüfung (Suche nach Fehlern in Modellen)
\item automatisierte Modellausführung
\item Modelle können zur Laufzeit geändert werden
\item Analyse der bereits ausgeführten Prozesse
\item Prozess Dekomposition
\end {itemize}

Abbildung \ref{fig:Declare} zeigt die Systemarchitektur von Declare.

\begin{figure}[H]
\begin{center}
  \includegraphics[scale=0.8]{Declare} %pdf, jpg, png...
  \caption{Declare Systemarchitektur nach \cite{pesic2007declare}}
  \label{fig:Declare}
\end{center}
\end{figure} 

Hieraus wird ersichtlich, dass \textit{Declare} mit den beiden Systemen \textit{YAWL} und \textit{ProM} kooperiert. Bei \textit{YAWL} handelt es sich um ein Workflow-Management System, welches auf strukturierte Workflows spezialisiert ist. Dies wirkt sich auf die Zusammenarbeit mit \textit{Declare} in der Art aus, dass die strukturierten Teile des Prozesses von \textit{YAWL} abgehandelt werden, während die unstrukturierten Teile von \textit{Declare} übernommen werden. Bei \textit{ProM} handelt es sich um ein Prozess-Mining-Tool. Hier werden bereits ausgeführte Prozesse von \textit{Declare} analysiert und darauf aufbauend werden dem Nutzer während der Prozessausführung Empfehlungen gegeben \cite{pesic2007declare}. \newline
Weiterhin besteht \textit{Declare} selbst aus drei Komponenten \textit{Framework}, \textit{Designer} und \textit{Worklist}.  Beim \textit{Designer} handelt es sich um ein Modellierungstool, welches für Systemeinstellungen und die Prozessmodell-Entwicklung verwendet wird (Abbildung \ref{fig:Designer}). Das \textit{Framework} ist für das Prozess-Enactment (Prozessbereitstellung) zuständig. Außerdem übernimmt es die Kommunikation mit \textit{YAWL} und \textit{ProM} und das Ändern der Prozessmodelle zur Laufzeit (Abbildung \ref{fig:Framework}). Die Prozessausführung wird von \textit{Worklist} durchgeführt. Hier können die Nutzer ihre zuvor erstellten Prozesse ausführen und können die von \textit{ProM} erstellten Empfehlungen sehen (Abbildung \ref{fig:Worklist}). Alle Modelle können als Bilddateien exportiert werden \cite{pesic2007declare}.



\begin{figure}[H]
\begin{center}
  \includegraphics[scale=0.4]{Designer} %pdf, jpg, png...
  \caption{Declare Designer (Screenshot aus Declare)}
  \label{fig:Designer}
\end{center}
\end{figure} 


\begin{figure}[H]
\begin{center}
  \includegraphics[scale=0.4]{Framework} %pdf, jpg, png...
  \caption{Declare Framework (Screenshot aus Declare)}
  \label{fig:Framework}
\end{center}
\end{figure} 

\begin{figure}[H]
\begin{center}
  \includegraphics[scale=0.4]{Worklist} %pdf, jpg, png...
  \caption{Declare Worklist (Screenshot aus Declare)}
  \label{fig:Worklist}
\end{center}
\end{figure} 





 










\chapter{Modellierungswerkzeuge}\label{sec:chapter4}
Dieses Kapitel stellt die in dieser Arbeit für die Prozessmodellierung benutzten Modellierungswerkzeuge vor. Nach einer kurzen allgemeinen Einführung in Modellierungswerkzeuge, wird das Modellierungswerkzeug Signavio beschrieben, welches zur imperativen Modellierung von Prozessen mit BPMN in dieser Arbeit herangezogen wird. Anschließend erfolgt eine Einführung in das Modellierungswerkzeug Declare, mit welchem die deklarativen Prozessmodelle in der Prozessmodellierungssprache ConDec in der vorliegenden Arbeit erstellt werden. 

\section{Modellierungswerkzeuge}\label{sec:chapter4:Modellierungswerkzeuge}
Ein Modellierungswerkzeug ist ein Softwaresystem, mit dessen Hilfe sich Prozessmodelle erstellen, ausführen und monitoren lassen. Teilweise bietet ein Modellierungswerkzeug noch weitere Funktionen wie z.B. Simulationen und die Analyse von Prozessmodellen an. Die Ausführung der Prozessschritte kann hierbei durch die jeweilige Person, welche für die Aktivität zuständig ist, ausgeführt werden. Für die Prozessmodellierung in der vorliegenden Arbeit kommt das Modellierungswerkzeug Signavio für die imperative Modellierung mit BPMN und Declare für die deklarative Modellierung mit ConDec zum Einsatz. Diese beiden Modellierungswerkzeuge werden nachfolgend vorgestellt \cite{gadatsch2012}.

\subsection{Signavio}

Bei Signavio handelt es sich um ein webbasiertes Prozessmodellierungstool, welches auch das kollaborative Modellieren von Prozessen mit den Modellierungsstandards BPMN und EPC zulässt. Der Vorteil von Signavio besteht darin, dass es nicht auf dem Rechner installiert werden muss, sondern direkt im Web-Browser ausgeführt werden kann. Die Prozessmodelle werden in einem zentralen Repository gespeichert und sind für die Benutzer entsprechend ihren Zugriffsrechten aufrufbar. Prozessmodelle besitzen alle eine eigene eindeutige URL und können über diese im Web-Browser aufgerufen werden. Hierbei wird auch gleich das Modellierungswerkzeug Signavio mitgeladen und kann somit im Web-Browser ausgeführt werden \cite{quteprints}. \newline
Abbildung \ref{fig:Signavio} zeigt den \textit{Signavio Process Editor}.

\begin{figure}[H]
\begin{center}
  \includegraphics[scale=0.6]{Signavio} %pdf, jpg, png...
  \caption{Siganvio Process Editor (Screenshot Siganvio)}
  \label{fig:Signavio}
\end{center}
\end{figure} 

Links ist die BPMN Palette zu sehen. Die einzelnen Elemente können per \textit{Drag and Drop} in das Arbeitsdokument gezogen werden. Signavio verfügt über Modellierungskonventionen. Mit diesen ist es  möglich, das Modell auf die Einhaltung von Modellierungsrichtlinien, wie z.B. Notationsumfang, Benennung, Prozessstruktur und Diagrammlayout zu überprüfen. Die Modelle können alle als PDF exportiert werden. \newline
In Abbildung \ref{fig:Simulation} ist die Simulations-Sicht von Signavio zu sehen. Hier kann der Benutzer den Prozessablauf simulieren. Dies kann einerseits mit Benutzerinteraktion Schritt für Schritt erfolgen oder auch im Ganzen durch den Simulator gesteuert werden, wobei XOR-Verzweigungen nach wie vor vom Benutzer ausgewählt werden müssen.
\begin{figure}[H]
\begin{center}
  \includegraphics[scale=0.6]{Simulation} %pdf, jpg, png...
  \caption{Siganvio Simulation (Screenshot Signavio)}
  \label{fig:Simulation}
\end{center}
\end{figure} 



\subsection{Declare}

Declare wurde als Constraint-basiertes Workflow-Management-System entwickelt. Es wird für die Entwicklung von Prozessmodellen, welche auf deklarativen Sprachen basieren, benutzt. Declare bietet die folgenden Funktionen an:
\begin {itemize}
\item Modellentwicklung
\item Modellüberprüfung (Suche nach Fehlern in Modellen)
\item automatisierte Modellausführung
\item wechselnde Modelle zur Laufzeit
\item Analyse der bereits durchgeführten Prozesse
\item Prozess Dekomposition
\end {itemize}

Abbildung \ref{fig:Declare} zeigt die Systemarchitektur von Declare.

\begin{figure}[H]
\begin{center}
  \includegraphics[scale=0.8]{Declare} %pdf, jpg, png...
  \caption{Declare Systemarchitektur nach \cite{pesic2007declare}}
  \label{fig:Declare}
\end{center}
\end{figure} 

Hieraus wird ersichtlich, dass \textit{Declare} mit den beiden Systemen \textit{YAWL} und \textit{ProM} kooperiert. Bei \textit{YAWL} handelt es sich hierbei um ein Workflow-Management System, welches auf strukturierte Workflows spezialisiert ist. Dies wirkt sich auf die Zusammenarbeit mit \textit{Declare} in der Art aus, dass die strukturierten Teile des Prozesses von \textit{YAWL} abgehandelt werden, während die unstrukturierten Teile von \textit{Declare} übernommen werden. Bei \textit{ProM} handelt es sich um ein Prozess-Mining-Tool. Hier werden bereits ausgeführte Prozesse von \textit{Declare} analysiert und darauf aufbauend werden dem Nutzer während der Prozessausführung Empfehlungen gegeben \cite{pesic2007declare}. \newline
Weiterhin besteht \textit{Declare} selbst aus drei Komponenten \textit{Framework}, \textit{Designer} und \textit{Worklist}.  Beim \textit{Designer} handelt es sich um ein Modellierungstool, welches für Systemeinstellungen und die Prozessmodell-Entwicklung verwendet wird (Abbildung \ref{fig:Designer}). Das \textit{Framework} ist für das Prozess-Enactment zuständig. Außerdem übernimmt es die Kommunikation mit \textit{YAWL} und \textit{ProM} und das Ändern der Prozessmodelle zur Laufzeit (Abbildung \ref{fig:Framework}). Die Prozessausführung wird von \textit{Worklist} durchgeführt. Hier können die Nutzer ihre zuvor erstellten Prozesse ausführen und können die von \textit{ProM} erstellten Empfehlungen sehen (Abbildung \ref{fig:Worklist}). Alle Modelle können als Bilddateien exportiert werden.



\begin{figure}[H]
\begin{center}
  \includegraphics[scale=0.4]{Designer} %pdf, jpg, png...
  \caption{Declare Designer (Screenshot aus Declare)}
  \label{fig:Designer}
\end{center}
\end{figure} 


\begin{figure}[H]
\begin{center}
  \includegraphics[scale=0.4]{Framework} %pdf, jpg, png...
  \caption{Declare Framework (Screenshot aus Declare)}
  \label{fig:Framework}
\end{center}
\end{figure} 

\begin{figure}[H]
\begin{center}
  \includegraphics[scale=0.4]{Worklist} %pdf, jpg, png...
  \caption{Declare Worklist (Screenshot aus Declare)}
  \label{fig:Worklist}
\end{center}
\end{figure} 






\chapter{Anforderungserhebung}\label{sec:chapter5}
In diesem Kapitel werden die Anforderungen an den in Kapitel 5 folgenden Vergleich der imperativen und deklarativen Modellierung für SE-Prozessmodelle erhoben. Hierfür werden in Kapitel 4.1 die Vergleichskriterien vorgestellt und erläutert.

\section{Vergleichskriterien}\label{sec:chapter5:Vergleichskriterien}

Bisher gibt es nur wenige Arbeiten, welche sich mit deklarativen Prozessmodellierungssprachen und insbesondere mit dem Vergleich von imperativen und deklarativen Prozessmodellierungssprachen beschäftigen. Aus diesem Grund soll in der vorliegenden Arbeit ein Vergleich der Anwendbarkeit zwischen deklarativen und imperativen Prozessmodellierungssprachen im Kontext von Softwareentwicklungsprozessen durchgeführt werden. Hierbei soll die Eignung der beiden Prozessmodellierungssprachen für die Modellierung von unterschiedlich großen Prozessmodellen beurteilt werden. Weiterhin sollen die Stärken und Grenzen der beiden Modellierungssprachen aufgezeigt werden und es soll dadurch herausgefunden werden, ob eine der beiden Modellierungssprachen über eine bessere Eignung zur Modellierung verfügt als die andere \cite{list2006evaluation}.\newline
Hierfür sollen die imperativen und deklarativen Prozessmodelle, welche für die drei Softwareentwicklungsprozesse Scrum, Open UP und V-Modell-XT erstellt werden im Hinblick auf verschiedene Vergleichskriterien untersucht werden. Da es sich bei Scrum um ein leichtgewichtiges Softwareentwicklungsprozessmodell, beim V-Modell XT um ein schwergewichtiges Softwareentwicklungsprozessmodell und bei Open UP um ein Softwareentwicklungsprozessmodell handelt, welches sich in der Mitte zwischen leichtgewichtig und schwergewichtig befindet, eignen sich diese drei besonders gut, zum Vergleichen der imperativen und deklarativen Modellierung für unterschiedlich große Matamodelle. Außerdem liegen den in imperativer und deklarativer Modellierungssprache zu erstellenden Prozessmodellen so jeweils die gleichen Metamodelle zugrunde, was eine objektive Bewertung für den Vergleich gewährleistet \cite{list2006evaluation}.  

\subsection{Erfüllung der Modellierungsgrundsätze}
Es sollen die imperativen und deklarativen Modellierungssprachen im Hinblick auf deren Erfüllung der in Kapitel 3.1.1 vorgestellten \textit{Grundsätze ordnungsgemäßer Modellierung} untersucht werden, da durch deren Einhaltung die Qualität, Klarheit und Konsistenz der Prozessmodelle gesichert wird \cite{freund2007}. Somit lässt sich hierdurch die Eignung der beiden Prozessmodellierungssprachen sehr gut überprüfen. Denn falls eine von beiden Prozessmodellierungssprachen die Modellierungsgrundsätze wesentlich schlechter einhalten kann, als die andere, so ist sie zum Modellieren deutlich weniger geeignet, da die hierdurch entstandenen Prozessmodelle geringere Qualität, Klarheit und Konsistenz aufweisen. Hierfür werden nachfolgend die erstellten Prozessmodelle im Hinblick auf \textit{Richtigkeit}, \textit{systematischen Aufbau}, \textit{Relevanz}, \textit{Klarheit}, \textit{Wirtschaftlichkeit} und \textit{Vergleichbarkeit} verglichen. Hierfür werden nachfolgend für jeden Modellierungsgrundsatz verschiedene Kriterien festgelegt, mit deren Hilfe die Einhaltung der Modellierungsgrundsätze für die jeweilige Modellierungssprache überprüft wird. \newline

\subsubsection{Richtigkeit}
Die Richtigkeit der Prozessmodelle soll verglichen werden. Hierbei soll die syntaktische und semantische Richtigkeit der Prozessmodelle untersucht werden. \newline
Die sysntaktische Korrektheit wird dahingegend untersucht, ob sich die jeweiligen Modelle unter Einhaltung der Modellierungsregeln der jeweiligen Prozessmodellierungssprache erstellen lassen. Dies wird mit Hilfe der Modellierungstools Signavio und Declare durchgeführt. Beide Programme verfügen über eine automatische Überprüfung der syntaktischen Korrektheit der dort erstellten Modelle. \newline
Bei der semantischen Korrektheit der Prozessmodelle wird verglichen in wie weit die mit deklarativer bzw. imperativer Prozessmodellierungssprache erstellten Prozessmodelle dem zugrunde liegenden Metamodell gegenüber vollständig und konsistent sind. Denn falls wesentliche Aspekte des Metamodells nicht darstellbar sind, leidet der Nutzen des Prozessmodells erheblich. Es wird somit überprüft, ob eine der beiden Prozessmodellierungssprachen die Struktur des Metamodells und das dort beschriebe Verhalten besser abbildet, als die andere. Insbesondere wird hier untersucht, ob es Grenzen in der Darstellbarkeit der abzubildenden Aspekte des Metamodells gibt \cite{journals95, becker2012prozessmanagement}. \newline


\subsubsection{Systematischer Aufbau}
Um den systematischen Aufbau der imperativen und deklarativen Prozessmodelle zu vergleichen, werden die Prozessmodelle dahingehend untersucht, in wie weit sie die Integration anderer Sichten in das Prozessmodell unterstützen und sie Verweise auf bestehende Datenmodelle zulassen. Da nicht alle Informationen, wie z.B. Daten und Funktionen in einem Prozessmodell abgebildet werden können, ist die Integration anderer Sichten in das Prozessmodell sehr wichtig, um wirklich alle Informationen aus dem Metamodell abbilden zu können. Hier können Rückschlüße auf die Eignung zur Modellierung gezogen werden und eventuelle Grenzen der Prozessmodellierungssprache aufgezeigt werden \cite{journals95, freund2007, becker2012prozessmanagement,koch2011}.

\subsubsection{Relevanz}
Beim Vergleich der Relevanz der Prozessmodelle werden die mit BPMN bzw. ConDec modellierten Prozessmodelle dahingehend verglichen in wie weit es möglich ist die Prozessmodelle mit den minimal relevanten Informationen zu erstellen. Es soll also untersucht werden, ob bei einer der beiden Prozessmodellierungssprachen mehr Informationen im Prozessmodell abgebildet werden müssen, als bei der anderen, um die Qualität des Prozessmodells zu sichern. Weiterhin soll auch andersrum untersucht werden, ob bei einer der beiden Prozessmodellierungssprachen alle minimal relevanten Informationen des Metamodells dargestellt werden können. Hier kann widerum die Eignung der beiden Prozessmodellierungssprachen sehr gut verglichen werden, da falls mit einer Prozessmodellierungssprache nicht alle minimal relevanten Informationen des Metamodells abgebildet werden können, ist diese nicht zum Modellieren geeignet.\cite{journals95, freund2007,reinshagen2009}. 

\subsubsection{Klarheit}
Die Prozessmodelle, welche jeweils in imperativer und deklarativer Prozessmodellierungssprache erstellt werden, sollen im Hinblick auf ihre Klarheit untersucht werden. Hierbei soll festgestellt werden, ob es wesentliche Unterschiede bei der Verständlichkeit der Prozessmodelle gibt, wenn diese in imperativer, bzw. deklarativer Prozessmodellierungssprache erstellt wurden. Denn fehlende Verständlichkeit eines Prozessmodells führt dazu, dass das Prozessmodell wenig Nutzen bringt. Insbesondere soll hier bei den imperativen Modellen die Anzahl an Verbindungen bzw. bei den deklarativen Modellen die Anzahl an Constraints zwischen Aktivitäten verglichen werden, da sich diese mit steigender Anzahl negativ auf die Verständlichkeit auswirken\cite{gruhn2006adopting, thesis_maja,haisjackl2014understanding}.\newline
Hier soll auch insbesondere die Anzahl unterschiedlicher Gateways/Constraints betrachtet werden.  Da alle Gateways in BPMN und alle Constraints in ConDec eine unterschiedliche Sematik haben und diese jeweils verstanden werden muss, kann sich eine hohe Anzahl an verschieden Gateways, bzw. Constraints negtiv auf die Verständlichkeit auswirken. Es existieren Studien über den geistigen Aufwand, welcher für das Verständnis von BPMN-Notationselementen notwendig ist. Den einzelnen Notationselementen werden verschiedene geistige Gewichtungen zugewiesen. Ein Sequenzfluss hat auf Grund des weniger hohen geistigen Aufwandes beim Verstehen eine geistige Gewichtung von 1. Das Exklusive Gateway hat eine geistige Gewichtung von 2, falls es nur zwei ausgehende Kanten hat. Bei drei oder mehr ausgehenden Kanten hat es eine geistige Gewichtung von 3. Das parallele Gateway hat eine geistige Gewichtung von 4. Einem Inklusiven Gateway wird sogar eine geistige Gewichtung von 7 zugeschrieben \cite{gruhn2006adopting, thesis_maja}.\newline
Leider existieren derzeit noch keine Metriken über den geistigen Aufwand beim Verstehen der einzelnen Constraints bei ConDec. Jedoch gibt es bereits Studien, welche sich anderweitig mit dem Verstehen von deklarativen Prozessmodellen auseinandergesetzt haben. Da die Existenz-Constraints relativ einfach zu verstehen sind, werden sie beim Vergleich in Kapitel 6 mit den Sequenzflusselemeneten gleichgesetzt und haben somit auch in etwa eine geistige Gewichtung von 1. Da die Constraints in ConDec genau wie die Gateways in BPMN Patterns darstellen, müssten sie ebenfalls alle Werte für die geistige Gewichtung zwischen 2 und 7 annehmen. Da die Constraints jedoch nicht direkt den Gateways in BPMN zugeordnet werden können und sich somit die geistigen Gewichtungen der Gateways nicht auf die Constraints übertragen lassen wird im Vergleich in Kapitel 6 nur die jeweilige Anzahl an Gateways und Constraints im Modell für den Vergleich herangezogen. Zudem wird auch die Anzhal an unterschiedlichen Gateways/Constraints betrachtet. Denn sowohl bei BPMN, als auch bei ConDec wurde in Studien herausgefunden, dass gerade die Kombination von vielen verschiedenen Gateways/Constraints einen sehr großen negativen Einfluß auf das Verstehen hat \cite{gruhn2006adopting, thesis_maja,haisjackl2014understanding}. \newline
Weiterhin soll hier untersucht werden, ob es wesentliche Unterschiede in der Verständlichkeit der imperativen und deklarativen Prozessmodelle gibt, in Abhängigkeit der Größe des zugrunde liegenden Softwareentwicklungsprozessmodells. Hierbei kann die Eignung der jeweiligen Modellierungssprache sehr gut festgestellt werden, da sie im Falle von schwerer/fehlender Verständlichkeit nicht zum Modellieren geeignet ist. Falls es Unterschiede in der Verständlichkeit der Prozessmodelle in Abhängigkeit der Größe des zugrunde liegenden Metamodells gibt, lassen sich hierbei Rückschlüsse auf die Eignung der Prozessmodellierungssprache in Bezug auf große/kleine Metamodelle ziehen \cite{leimeister2012,journals95, freund2007,reinshagen2009, becker2012prozessmanagement,koch2011,bpm07,thesis_maja}.


\subsubsection{Wirtschaftlichkeit der Prozessmodelle}
Hier soll untersucht werden, ob sich der Aufwand für die Modellierung bei den beiden Modellierungssprachen erheblich voneinander unterscheidet, da wenn die Erstellung eines Prozessmodells mit einem zu hohen Aufwand für die Erstellung verbunden ist, obwohl der spätere Nutzen des Prozessmodells erheblich geringer ist, ist die Modellierung nicht sinnvoll. Hier kann die Eignung zur Modellierung der Prozessmodellierungssprachen sehr gut verglichen werden, denn falls der Aufwand für die Modellierung für eine der beiden Prozessmodellierungssprachen weitaus höher ist, als für die andere, eignet sich die Prozessmodellierungssprache mit dem sehr viel höherem Aufwand nicht für die Modellierung.\newline
Hier wird einerseits die Anzahl der Elemente insgesamt, welche zur Modellierung der Prozesse notwendig sind miteinander verglichen. Weiterhin wird die Anzahl von Gateways in BPMN mit der Anzahl von Constraints in ConDec miteinander verglichen und auch die Anzahl unterschiedlicher Gateways/Constraints. Da bei der Verwendung von Gateways/Constraints für den Modellierer ein höherer geistiger Aufwand notwendig ist und somit auch ein größerer Aufwand für das Modellieren, kann sich dies negativ auf die \textit{Wirtschaftlichkeit} auswirken. Weiterhin sind komplexere Modelle auch fehleranfälliger, d.h. wenn der Modellierer einen höheren geistigen Aufwand beim Modellieren leisten muss, ist es auch wahrscheinlicher, dass ihm Fehler beim Modellieren passieren \cite{freund2007, journals95, leimeister2012,mendling2010seven}.


\subsubsection{Vergleichbarkeit}
Bei der Vergleichbarkeit der Prozessmodelle wird untersucht, ob die in imperativer, bzw. deklarativer Prozessmodellierungssprache erstellten Prozessmodelle, welchen die gleichen Metamodelle zugrunde liegen, trotzdem vergleichbare Prozessmodelle darstellen. Es wird hier somit insbesondere untersucht, ob die Abstraktionsgrade der Prozessmodelle sich wesentlich voneinander unterscheiden. Weiterhin wird die Vergleichbarkeit in Bezug auf das Ausführungsverhalten der imperativen und deklarativen Prozessmodelle durch Ausführung der Modelle in den Modellierungstools Siganvio und Declare nach der Modellierung überprüft. Hier werden die möglichen Pfade im jeweiligen Modell durchlaufen und somit wird sichergestellt, dass das Verhalten der Modelle gleich ist. Außerdem wird hier die Größe der jeweiligen Prozessmodelle als Vergleichskriterium herangezogen. Hier wird die Gesamtanzahl der notwendigen Elemente zur Darstellung des Prozessmodells verglichen. Es soll festgestellt werden, ob bei Verwendung einer imperativen oder deklarativen Prozessmodellierungssprache wesentlich mehr Elemente zur Darstellung des gleichen Prozesses notwendig sind indem die Anzahl der verwendeten Aktivitäten und Patterns verglichen wird \cite{leimeister2012, journals95, freund2007,reinshagen2009}.

Eine Übersicht über die Modellierungsgrundsätze und die jeweiligen Vergleichskriterien bietet Abbildung \ref{fig:TabelleKriterien}.

\begin{figure}[!htbp]
\begin{center}
  \includegraphics[width=\textwidth]{TabelleKriterien} %pdf, jpg, png...
  \caption{Übersicht Vergleichskriterien}
  \label{fig:TabelleKriterien}
\end{center}
\end{figure}









\chapter{Imperative und deklarative Modellierung für SE-Prozessmodelle}\label{sec:chapter6}
In diesem Kapitel wird der Vergleich zwischen imperativer und deklarativer Modellierung für SE-Prozessmodelle durchgeführt. Als Erstes wird dieser Vergleich in Kapitel 6.1 für das SE-Prozessmodelle Scrum durchgeführt. Hierfür wird zunächst in Kapitel 6.1 das der Modellierung zugrunde liegende Modell, das SE-Prozessmodell Scrum, vorgestellt und in Kapitel 6.1.1 für die Modellierung analysiert. Danach erfolgt in Kapitel 6.1.2 die imperative Modellierung von in der Prozessmodellierungssprache BPMN und anschließend die deklarative Modellierung in der Prozessmodellierungssprache ConDec in Kapitel 6.1.3. Danach erfolgt  in Kapitel 6.1.4 der Vergleich zwischen den beiden Modellen.\newline
Der zweite SE-Prozess, welcher in diesem Kapitel in imperativer und deklarativer Prozessmodellierungssprache verglichen werden soll, ist der Open Unified Process (Open Up). Auch hier erfolgt zunächst eine kurze Einführung in den Open Up in Kapitel 6.2, bevor dieser in Kapitel 6.2.1 analysiert wird, damit er in den Kapiteln 6.2.2 und 6.2.3 in imperativer, bzw. deklarativer Prozessmodellierungssprache modelliert werden kann. Hiernach erfolgt in Kapitel 6.2.4 der Vergleich zwischen den Prozessmodellen.\newline
Zuletzt wird noch das V-Modell XT modelliert und verglichen. Eine Einführung in das V-Modell XT erfolgt in Kapitel 6.3. In Kapitel 6.3.1 wird dieses als Vorbereitung für die Modellierung analysiert und in den Kapiteln 6.3.2 und 6.3.3 in imperativer und deklarativer Prozessmodellierungssprache modelliert. Der Vergleich hierzu erfolgt in Kapitel 6.3.4.


\section{Scrum}\label{sec:chapter6:Imperative Modellierung}

Der Begriff Scrum stammt aus dem Artikel "The New New Product Development Game", welchen Hirotaka Takeuchi und Ikujiro Nonaka im Harvard Business Review 1986 veröffentlicht haben. Sie beschrieben einen ganzheitlichen Ansatz bei dem kleine, funktionsübergreifende Teams zusammen an einem gemeinsamen Ziel arbeiten. Dies verglichen sie mit der Scrum-Formation beim Rugby \cite{Pham2012,Takeuchi1986}. \newline
Bei Scrum handelt es sich um ein agiles Prozessmodell, welches seit Anfang 1990 für komplexe Entwicklungen verwendet wird. Agile Prozessmodelle werden den leichtgewichtigen Prozessmodellen zugeordnet \cite{Hanser2010, Lacey2012}. Einen ersten Überblick über das Scrum-Prozessmodell gibt Abbildung \ref{fig:Scrum}:

\begin{figure}[htp]
\begin{center}
  \includegraphics[scale=0.6]{Scrum} %pdf, jpg, png...
  \caption{Scrum Überblick nach \cite{scrum2008}}
  \label{fig:Scrum}
\end{center}
\end{figure}

Der genaue Ablauf im Scrum Prozessmodell wird nachfolgend genau analysiert.

\subsection{Analyse Scrum}


Im Scrum-Prozessmodell gibt es nur drei verschiedene Rollen: Den \textit{Product Owner}, das \textit{Team} und den \textit{Scrum Master}. Sämtliche Verantwortlichkeiten innerhalb eines Projektes werden hierbei auf diese drei Rollen aufgeteilt \cite{Schwaber2004}. \newline

Der \textit{Procuct Owner} ist verantwortlich, die Interessen aller am Projekt beteiligten Personen zu vertreten. Neben der Budgetierung des Projektes erstellt er ebenfalls Releasepläne und erstellt den Produkt-Backlog, welcher eine Liste mit funktionalen und nicht-funktionalen Anforderungen darstellt \cite{Schwaber2004, Pichler2010,Schwaber2007}. Weiterhin priorisiert er die Aufgaben, welche von den Entwicklern im Sprint erledigt werden sollen, so dass die aktuell nützlichsten Elemente die höchste Priorität haben. Er erstellt eine Liste dieser Elemente, welche \textit{Sprint-Backlog} genannt wird \cite{Henning2011, Schwaber2007,Pichler2010}. Der \textit{Procuct Owner} ist ebenfalls zuständig für das Annehmen, bzw. Ablehnen der Arbeitsergebnisse \cite{eclipseScrum}. \newline

Die \textit{Teams} bestehen bei Scrum für gewöhnlich aus fünf bis neun Mitgliedern und verwalten sich selbst. Ihre Tätigkeiten müssen erfolgreich sein, liegen aber in ihrer eigenen Verantwortung \cite{Pries2011, Wolf2011}. Alle Teammitglieder sind gemeinsam für den Erfolg eines jeden \textit{Sprints} und des gesamten Projektes verantwortlich \cite{Pichler2010}. \newline

Der \textit{Scrum-Master} ist für den gesamten Scrum-Prozess verantwortlich. Dies schließt die Vermittlung von Scrum-Inhalten (z.B. Schulungen) und die Implementation von Scrum in die Unternehmenskultur ein \cite{Pichler2010}. Er überwacht die Sprint- Tasks, um sicher zu gehen, dass der Sprint erfolgreich verläuft.\newline

Bei Scrum wird die Entwicklung in mehrere kurze Zyklen, also Iterationen eingeteilt. Eine einzelne Iteration wird bei Scrum \textit{Sprint} genannt \cite{Henning2011}. Die Dauer eines Sprints beträgt zwei bis vier Wochen. Am Ende eines jeden Sprints muss das \textit{Team} ein lauffähiges Produkt abliefern \cite{Wolf2011}. Vor jedem Sprint findet ein \textit{Sprint Planning Meeting} statt, welches sich aus zwei Teilen zusammensetzt \cite{Pichler2010}. Im ersten Teil findet eine Planung des nächsten \textit{Sprints} statt \cite{Lacey2012}. Hierfür präsentiert der \textit{Product Owner} dem \textit{Team} eine Liste der Product-Backlog-Elemente mit der aktuell höchsten Priorität \cite{Schwaber2004, Schwaber2007,Pichler2010}. Diese Liste wird \textit{Sprint-Backlog} genannt \cite{Wolf2011}. Das \textit{Team} hat die Möglichkeit Fragen bezüglich Inhalt, Zweck, Bedeutung und Absichten der \textit{Sprint-Backlog}-Elemente zu stellen. Anschließend werden die einzelnen Elemente aus dem \textit{Sprint-Backlog} in sogenannte \textit{Tasks} aufgeteilt, welche jeweils eine ideale Bearbeitungszeit von zwei bis vier Stunden haben, aber niemals länger als zwei Tage dauern sollten \cite{Wolf2011}. Das \textit{Team} kann sich die Aufgaben eigenverantwortlich aufteilen und muss sich anschließend dem \textit{Product  Owner} verpflichten, die \textit{Tasks} bis zum Abschluss des \textit{Sprints} zu erledigen \cite{Wolf2011, Keith2010,Pichler2010}.
Das  \textit{Team} trifft sich während des \textit{Sprints} täglich in einem 15-minütigen Meeting, dem \textit{täglichem Scrum-Meeting}. Hier redet das \textit{Team} über seinen Fortschritt und eventuelle Probleme bei ihrer Arbeit \cite{Keith2010}. Hier muss jedes Teammitglied die nachfolgenden drei Fragen beantworten       \cite{Wolf2011}:
   \begin{enumerate}
      \item Was habe ich seit gestern erreicht?
      \item Was werde ich heute erreichen?
      \item Was blockiert mich?
      \end {enumerate}
      
\subsection{Imperative Modellierung Scrum}

\begin{figure}[htp]
\begin{center}
  \includegraphics[width=\textwidth]{ScrumImperativ} %pdf, jpg, png...
  \caption{Imperative Modellierung Scrum}
  \label{fig:ScrumImperativ}
\end{center}
\end{figure}

\begin{figure}[htp]
\begin{center}
  \includegraphics[width=\textwidth]{ScrumImperativUnter} %pdf, jpg, png...
  \caption{Imperative Modellierung Scrum Unterprozess}
  \label{fig:ScrumImperativUnter}
\end{center}
\end{figure}

\subsection{Deklarative Modellierung Scrum}

\begin{figure}[htp]
\begin{center}
  \includegraphics[width=\textwidth]{ScrumDeklarativ} %pdf, jpg, png...
  \caption{Deklarative Modellierung Scrum}
  \label{fig:ScrumDeklarativ}
\end{center}
\end{figure}

\subsection{Vergleich}



\section{Open Unified Process (Open Up)}


Der Open Unified Process, kurz Open Up ist eine frei zugängliche Variante des Rational Unified Process, welcher ein sehr bekannter Entwicklungsprozess ist \cite{hauber2010}.  Er ist Teil des Eclipse Process Frameworks. Open Up ist ein iterativer, inkrementeller und minimaler Prozess, aber dennoch vollständig und erweiterbar \cite{Gau2006, Basem2010}. Der Prozess ist minimal gehalten, da er nur die wesentlichen Inhalte einbezieht. Trotzdem ist er vollständig, da er als Prozess benutzt werden kann, um ein Softwaresystem zu entwickeln. Er ist außerdem auch erweiterbar, da er als Grundlage herangezogen werden kann und mit weiteren Prozessfragmenten aufgestockt und nach Belieben zugeschnitten werden kann \cite{Wang2007}. Das Konzept des Open Up ist es den Prozess zu vergrößern, sich aber auf das Minimum, welches für das Projekt benötigt wird zu beschränken, anstatt zu versuchen große, überladene Prozesse zu verstehen und diese dann zu verkleinern \cite{ambler2012}.  \newline



Open Up ist auf kleine Teams ausgerichtet, bei welchen bei der Zusammenarbeit räumliche Nähe besteht. Die Teammitglieder haben hierbei die Freiheit, ihre eigenen Entscheidungen bezüglich ihren aktuellen Aufgaben und Prioritäten zu treffen, um die Anforderungen der Stakeholder zu erfüllen. Das Team trifft sich täglich, um über den aktuellen Status zu reden \cite{OpenUPProcess}.\newline
Es werden Rollen, Aufgaben, Artefakte und Ebenen in Open Up definiert. Dies soll ermöglichen, dass verschiedene Sichten, die sich in ihrem Detaillierungsgrad unterscheiden auf das Projekt möglich sind \cite{freudenreichevaluierung}. Einen ersten Überblick über Open Up gibt Abbildung \ref{fig:openup}.


\begin{figure}[htp]
\begin{center}
  \includegraphics[width=\linewidth]{openup} %pdf, jpg, png...
  \caption{Open Up Überblick nach \cite{eclipseopenup}}
  \label{fig:openup}
\end{center}
\end{figure}

Der Open Up wird im Folgenden genauestens analysiert.

\subsection{Analyse Open Up}


Auf der persönlichen Ebene teilen sich die Teammitglieder ihre Arbeit in \textit{Mikro-Inkremente} ein. Diese stellen das Ergebnis von Stunden, bzw. wenigen Tagen Arbeit dar. Die Arbeit entwickelt sich somit ein Mikro-Inkrement weiter und der Fortschritt kann Tag für Tag nachvollzogen werden. Die Teammitglieder teilen ihre Fortschritte täglich miteinander, was die Arbeitstransparenz und das Vertrauen erhöht und die Teamarbeit fördert \cite{eclipseopenup}. \newline

Auf der Team-Ebene wird das Projekt in Iterationen unterteilt, welche einen Zeitraum von mehreren Wochen umfassen, mit dem Ziel am Ende eines Iterationszyklus ein funktionierendes Softwareinkrement zu haben. Dieses Inkrement stellt eine Version des Softwaresystems dar welche zusätzliche oder verbesserte Funktionalitäten besitzt als die vorherige Version \cite{Basem2010}.  In jeder Iteration wird ein Iterationsplan angefertigt,  der vorgibt, was in dieser Iteration geliefert werden muss und auf welchen sich das Team verpflichten muss \cite{freudenreichevaluierung}.

Auf Stakeholder-Ebene wird diesen durch den \textit{Projektlebenszyklus} die Möglichkeit gegeben, die Projektfinanzierung, den Umfang, das Risiko und andere Aspekte des Prozesses zu kontrollieren.
Der Open Up teilt den \textit{Projektlebenszyklus} in vier Phasen ein, über welche Abbildung \ref{fig:Phasen} einen Überblick gibt \cite{eclipseopenup}.

\begin{figure}[htp]
\begin{center}
  \includegraphics[width=\linewidth]{Lebenszyklusopenup} %pdf, jpg, png...
  \caption{Phasen Open Up nach \cite{eclipseopenup}}
  \label{fig:Phasen}
\end{center}
\end{figure}


In jeder Phase finden eine oder mehrere Iterationen statt und werden mit einem Meilenstein abgeschlossen \cite{Basem2010}. Tabelle \ref{tab:tab1} zeigt die Abläufe in den Iterationen in den einzelnen Phasen und die zugehörigen Zielstellungen.
\begin{longtable}{|p{7cm}|p{8cm}|}
\hline
Vorlagenmodell Iterationen & Zielsetzung der Phase \\
\hline
Inception Phase Iteration 
\begin {itemize}
\item Iteration starten 
 \item  Iteration planen und verwalten
 \item  Anforderungen festlegen und verfeinern 
  \end{itemize}
   &
  
  \begin {itemize}
\item Verstehen, was zu bauen ist
 \item Die wichtigsten Systemfunktionen verstehen 
\item Mindestens eine mögliche Lösung bestimmen
\item Kosten, Zeitplan und Risiken verstehen, welche mit dem Projekt verbunden sind
  \end{itemize}

 \\
\hline
 Elaboration Phase Iteration 
   \begin {itemize}
   \item Iteration planen und verwalten
   \item Anforderungen erheben und verfeinern
   \item Architektur definieren
   \item Lösung entwickeln
   \item Testlösung
   \item Laufende Aufgaben
   
  \end{itemize}

  & 
     \begin {itemize}
   \item Ein detaillierteres Verständnis der Anforderungen einholen
   \item Architektur designen, implementieren und validieren
   \item  Wesentliche Risiken mindern und genauen Zeitplan und Kostenschätzungen erstellen
    \end{itemize}
 \\
\hline
\hline
Construction Phase Iteration 
   \begin {itemize}
   \item Iteration planen und verwalten
   \item Anforderungen erheben und verfeinern
     \item Lösung entwickeln
   \item Testlösung
   \item Laufende Aufgaben

\end{itemize}

&
   \begin {itemize}
\item Komplettes Produkt iterativ entwickeln, welches am Ende bereit ist an seine Nutzer ausgeliefert zu werden
\item Entwicklungskosten minimieren und einen gewissen Grad an Parallelität erzielen 
\end{itemize}

 \\
\hline
Transition Phase Iteration 

   \begin {itemize}

 \item Iteration planen und verwalten
     \item Lösung entwickeln
   \item Testlösung
   \item Laufende Aufgaben

 \end{itemize}

 
  &
  
     \begin {itemize}
      \item Beta-Test, um zu überprüfen, dass die Erwartungen der Benutzer erfüllt sind 
     \item Zustimmung der Stakeholder einholen, dass Bereitstellung abgeschlossen ist
      \end{itemize}

\\

\hline

\caption{Iterationen und Zielstellungen der Phasen in Open Up \cite{eclipseopenup}}
\label{tab:tab1}
\end{longtable}






Abbildung \ref{fig:RollenOpenUp} gibt einen Überblick über die verschiedenen Rollen in Open Up. Die Rolle \textit{Analyst} stellt den Kunden und Endnutzer dar. Die Aufgaben des \textit{Analysten} bestehen aus dem Sammeln von Informationen von den Stakeholdern, um das Problem, welches es zu lösen gilt, zu verstehen. Weiterhin erstellt er Anforderungen und setzt Prioritäten für diese.\newline
Der \textit{Architekt} ist für die Definition der Software-Architektur verantwortlich. D.h. er trifft alle wichtigen technischen Entscheidungen, die die gesamte Entwicklung und Umsetzung des Systems betreffen \cite{OpenUPProcess}.\newline
Der \textit{Entwickler} entwickelt einen Teil des Systems und muss hierbei sicherstellen, dass dieser in die Gesamtarchitektur passt. Er muss eventuell Prototypen des User-Interface anfertigen und anschließend die Komponenten implementieren, testen und integrieren.\newline
Der \textit{Projekt Manager} führt die Planung des Projektes durch, koordiniert die Zusammenarbeit zwischen allen Beteiligten und achtet darauf, dass das Projektteam die Erfüllung der Projektziele stets im Auge behält \cite{OpenUPProcess}.\newline
Die Rolle des \textit{Stakeholders} schließt alle Interessengruppen ein, deren Ansprüche durch das Projekt erfüllt werden müssen. \newline
Der \textit{Tester} ist für sämtliche Testaktivitäten verantwortlich. Diese umfassen die Ermittlung, Festlegung, Umsetzung und Durchführung der erforderlichen Tests sowie die Protokollierung und Analyse der Ergebnisse \cite{OpenUPProcess}.
\begin{figure}[htp]
\begin{center}
  \includegraphics[scale=0.6]{RollenOpenUp} %pdf, jpg, png...
  \caption{Rollen in Open Up nach \cite{openup}}
  \label{fig:RollenOpenUp}
\end{center}
\end{figure}

Eine Task bezeichnet in Open Up die Arbeitseinheit einer Rolle, welche von dieser durchgeführt werden soll. Insgesamt gibt es 18 Tasks, welche von den verschiedenen Rollen entweder als Primär-Darsteller(der Verantwortliche für die Durchführung der Aufgabe) oder als zusätzlicher Darsteller(Unterstützung und Bereitstellung von Informationen, die in der Task- Ausführung verwendet werden) durchgeführt werden. Hierdurch wird der kollaborative Charakter von Open Up gefestigt \cite{eclipseopenup}.

Ein Artefakt ist etwas, das hergestellt, modifiziert oder durch eine Task verwendet wird. Rollen sind 
für die Erstellung und Aktualisierung von Artefakten verantwortlich. Artefakte stellen eine Versionskontrolle während des gesamten Projektlebenszyklus dar. Die 17 Artefakte in OpenUP gelten als die wesentlichen Artefakte, welche ein Projekt verwenden sollte, um produkt- und projektbezogene Informationen zu erfassen. Die Informationen müssen hierbei nicht mit formalen Artefakten festgehalten werden, dies kann auch informell, z.B durch White-Boards oder Meeting-Notizen geschehen. Es können die Open Up Artefakte oder eigene Artefakte verwendet werden \cite{eclipseopenup}.

\subsection{Imperative Modellierung Open Up}

\subsubsection{Develop Solution Increment}

 Bei der Aktivität \textit{Develop Solution Increment} geht es um das Design, die Implementierung, das Testen und die Integration der Lösung für eine Anforderung in einem bestimmten Kontext. Sie tritt genauso viele Male auf, wie es Arbeitsaufgaben gibt, die in einer Iteration entwickelt werden müssen.
 Handelt es sich um eine typische Veränderung wird zunächst eine Lösung designt und anschließend ein Entwickeltest implementiert. Bei einer trivialen Änderung an der bestehenden Implementierung kann diese auch direkt in der bestehenden Architektur vorgenommen werden. \newline
 Sobald die Fragen der technischen Umsetzung geklärt sind, werden Entwicklertests implementiert, um die Implementierung zu verifizieren. Anschließend werden diese Entwicklertests ausgeführt.\newline
 Falls bei der Ausführung der Tests Fehler ersichtlich werden, muss eine Lösung für diesen Fehler implementiert werden und die Entwicklertests müssen erneut ausgeführt werden. Dies wird solange wiederholt, bis alle Tests bestanden sind.\newline
 Auch wenn alle Tests bestanden werden, sollte der Entwurf an dieser Stelle nochmals überdacht werden. Falls hier beschlossen wird, dass der Code überarbeitet werden muss, muss im Prozess zurückgegangen werden und erneut eine Lösung designt werden, da eine Änderung des Codes die Implementation und die Entwicklertests beeinflussen könnte.\newline
 Da es am Besten ist die Implementierungsteile so klein wie möglich zu halten, sollte zunächst eine kleine Design-Lösung für einen Teil der Arbeitsaufgabe entwickelt werden. Anschließend sollte dies für weitere kleine Teile solange wiederholt werden, bis die gesamte Arbeitsaufgabe implementiert ist. \newline
 
 
\begin{figure}[htp]
\begin{center}
  \includegraphics[width=\linewidth]{DevelopSolution} %pdf, jpg, png...
  \caption{Develop Solution Increment imperativ}
  \label{fig:Develop}
\end{center}
\end{figure}

\subsubsection{Plan and manage iteration- Inception}

Die Aktivität \textit{Plan and manage iteration} wird während des gesamten Projektlebenszyklus ausgeführt. Ihr Ziel ist es, Risiken und Probleme früh genug zu identifizieren, damit diese entschärft werden können, um die Ziele für die Iteration festzulegen und das Team dabei zu unterstützen, diese zu erreichen.\newline
Die iteration wird durch den Projektmanager und das Team gestartet. Hier findet die Priorisierung der Arbeit für eine gegebene Iteration statt. Der Projektmanager, die Stakeholder und die Teammitglieder einigen sich darauf, was während der Iteration zu entwickeln ist.\newline
Die Teammitglieder melden sich für die Arbeitsaufgaben, die während der Iteration entwickelt werden müssen. Anschließend teilt sich jedes Teammitglied seine Arbeitsaufgaben selbstständig in Arbeitseinheiten ein und schätzt den Aufwand hierfür ab.\newline
Während der Iteration trifft sich das Team regelmäßig, um den aktuellen Stand der Arbeit und eventuelle Probleme zu besprechen.


\begin{figure}[htp]
\begin{center}
  \includegraphics[width=\linewidth]{PlanAndManageIterationInception-2} %pdf, jpg, png...
  \caption{Plan and manage iteration imperativ -Inception}
  \label{fig:PlanAndManageIterationInception-2}
\end{center}
\end{figure}

\begin{figure}[htp]
\begin{center}
  \includegraphics[width=\linewidth]{PlanAndManageIterationInception-2Unter} %pdf, jpg, png...
  \caption{Plan and manage iteration imperativ -Inception Unterprozess Umgebung vorbereiten} 
  \label{fig:PlanAndManageIterationInception-2}
\end{center}
\end{figure}


\begin{figure}[htp]
\begin{center}
  \includegraphics[width=\linewidth]{PlanAndManageIterationInception-2UnterUnter} %pdf, jpg, png...
  \caption{Plan and manage iteration imperativ-Inception Unterprozess Prozess maßschneidern}
  \label{fig:PlanAndManageIterationInception-2UnterUnter}
\end{center}
\end{figure}

\subsubsection{Deploy Release-Transition}


Das Ergebnis dieser Aktivität ist die Release eines Sets von integrierten Komponenten in die Integrationsumgebung. 
\begin{figure}[htp]
\begin{center}
  \includegraphics[width=\linewidth]{DeployReleaseTransition} %pdf, jpg, png...
  \caption{Deploy Release-Transition}
  \label{fig:DeployRelease}
\end{center}
\end{figure}

\begin{figure}[htp]
\begin{center}
  \includegraphics[width=\linewidth]{Releasezusammenstellen} %pdf, jpg, png...
  \caption{Deploy Release-Transition Unterprozess Release zusammen stellen}
  \label{fig:Releasezusammenstellen}
\end{center}
\end{figure}

\begin{figure}[htp]
\begin{center}
  \includegraphics[width=\linewidth]{Deploymentplanausfuhren} %pdf, jpg, png...
  \caption{Deploy Release-Transition Unterprozess Deploymentplan ausführen}
  \label{fig:Deploymentplanausfuhren}
\end{center}
\end{figure}

\begin{figure}[htp]
\begin{center}
  \includegraphics[width=\linewidth]{erfolgreichesDeploymentsicherstellen} %pdf, jpg, png...
  \caption{Deploy Release-Transition Unterprozess Erfolgreiches Deployment sicherstellen}
  \label{fig:erfolgreichesDeploymentsicherstellen}
\end{center}
\end{figure}

\begin{figure}[htp]
\begin{center}
  \includegraphics[width=\linewidth]{Backoutplanausfuhren} %pdf, jpg, png...
  \caption{Deploy Release-Transition Unterprozess Backoutplan ausführen}
  \label{fig:Backoutplanausfuhren}
\end{center}
\end{figure}


\begin{figure}[htp]
\begin{center}
  \includegraphics[width=\linewidth]{ReleaseMitteilungenubermitteln} %pdf, jpg, png...
  \caption{Deploy Release-Transition Unterprozess Release Mitteilungen übermitteln}
  \label{fig:Backoutplanausfuhren}
\end{center}
\end{figure}

\subsubsection{Develop Product Documentation-Construction}

\begin{figure}[htp]
\begin{center}
  \includegraphics[width=\linewidth]{DevelopProductDocumentation} %pdf, jpg, png...
  \caption{Develop Product Documentation-Construction}
  \label{fig:DevelopProductDocumentation}
\end{center}
\end{figure}

\begin{figure}[htp]
\begin{center}
  \includegraphics[width=\linewidth]{ProduktdokumentationErstellen} %pdf, jpg, png...
  \caption{Develop Product Documentation-Construction Unterprozess Produktdokumentation erstellen}
  \label{fig:ProduktdokumentationErstellen}
\end{center}
\end{figure}

\begin{figure}[htp]
\begin{center}
  \includegraphics[width=\linewidth]{BenutzerDokumentationErstellen} %pdf, jpg, png...
  \caption{Develop Product Documentation-Construction Unterprozess BenutzerdokumentationErstellen}
  \label{fig:BenutzerDokumentationErstellen}
\end{center}
\end{figure}


\begin{figure}[htp]
\begin{center}
  \includegraphics[width=\linewidth]{UnterstuetzungsDokumentationErstellen} %pdf, jpg, png...
  \caption{Develop Product Documentation-Construction Unterprozess UnterstützungsdokumentationErstellen}
  \label{fig:UnterstuetzungsdokumentationErstellen}
\end{center}
\end{figure}

\begin{figure}[htp]
\begin{center}
  \includegraphics[width=\linewidth]{TrainingsmaterialErstellen} %pdf, jpg, png...
  \caption{Develop Product Documentation-Construction Unterprozess Trainingsmaterial erstellen}
  \label{fig:TrainingsmaterialErstellen}
\end{center}
\end{figure}

\subsection{Deklarative Modellierung Open Up}


\begin{figure}[htp]
\begin{center}
  \includegraphics[width=\linewidth]{DevelopSolutionIncrement} %pdf, jpg, png...
  \caption{Develop Solution Increment deklarativ}
  \label{fig:Develop}
\end{center}
\end{figure}

\begin{figure}[htp]
\begin{center}
  \includegraphics[width=\linewidth]{PlanIterationdeclare} %pdf, jpg, png...
  \caption{Plan and manage iteration deklarativ}
  \label{fig:PlanIterationdeclare}
\end{center}
\end{figure}
\subsection{Vergleich}







\section{V-Modell XT}


Das V-Modell XT zählt zu den schwergewichtigen Prozessmodellen \cite{Hanser2010}. Es wird als Entwicklungsstandard für die Durchführung von IT-Vorhaben in der öffentlichen Verwaltung in Deutschland herangezogen \cite{Kuhrmann2011}. Beschrieben werden im V-Modell XT die Abläufe im Verlauf eines Entwicklungsprojektes über Produkte, Rollen und Aktivitäten \cite{Friedrich2008}. Es wird somit ganz genau geregelt, \textit{Wer}, \textit{Wann}, \textit{Was} in einem Projekt zu tun hat \cite{2004vmodell}. Die Vorgehensbausteine ermöglichen neben einer Modularisierung der Abläufe auch eine flexible Zusammenstellung, wodurch das V-Modell XT auf die jeweils eigene Situation angepasst werden kann. \cite{Friedrich2008,Zoerner2012}. \newline
\subsection{Analyse V-Modell XT}

Abbildung \ref{fig:grundstruktur} zeigt die Grundstruktur des V-Modell XT, welche im Folgenden detailliert erläutert wird.
\begin{figure}[htp]
\begin{center}
  \includegraphics[width=\linewidth]{grundstruktur} %pdf, jpg, png...
  \caption{Grundstruktur V-Modell XT nach \cite{2004vmodell}}
  \label{fig:grundstruktur}
\end{center}
\end{figure}

\subsubsection{Projekttypen}
Nicht alle V-Modell-Projekttypen laufen nach exakt demselben Schema ab. Auf Grund ihrer charakteristischen Eigenschaften lassen sie sich demnach in unterschiedliche Projekttypen einteilen. Abbildung \ref{fig:Projekttypen} gibt einen ersten Überblick über die verschiedenen Projekttypen im V-Modell XT \cite{2004vmodell}.
\begin{figure}[htp]
\begin{center}
  \includegraphics[width=\linewidth]{v-modell-rollen} %pdf, jpg, png...
  \caption{Projekttypen V-Modell XT nach \cite{2004vmodell}}
  \label{fig:Projekttypen}
\end{center}
\end{figure}

Es existieren somit drei verschiedene Projekttypen: \textit{Systementwicklungsprojekt eines Auftraggebers}, \textit{Systementwicklungsprojekt eines Auftragnehmers} und \textit{Einführung und Pflege eines organisationsspezifischen Vorgehensmodells} \cite{reinhard2008}. \newline

Es werden drei verschiedene Projektrollen unterschieden, welche dem jeweiligen Projekttyp entsprechen: In der Rolle \textit{Auftragnehmer} wird ein vom \textit{Auftraggeber} spezifiziertes System entwickelt. Die Systementwicklung wird an einen oder mehrere \textit{Arbeitnehmer} weiter gegeben, wenn man sich in der Rolle \textit{Arbeitgeber} befindet. Das System  wird selbst entwickelt in der Rolle \textit{Auftraggeber/Auftragnehmer} \cite{brack2010,2004vmodell}.\newline


Beim \textit{Systementwicklungsprojekt eines Auftraggebers} wird die Entwicklung des Projektgegenstandes im Projektverlauf  ausgeschrieben und der Auftragnehmer trifft eine Auswahl anhand der eingehenden Angebote. Der Auftragnehmer, welcher für die Entwicklung des Projektgegenstandes ausgewählt wurde, entwickelt den Projektgegenstand, welcher dann vom Auftragnehmer abgenommen wird \cite{reinhard2008,2004vmodell}.\newline
Umgekehrt wird beim \textit{Systementwicklungsprojekt eines Auftragnehmers} im Laufe des Projektes ein Angebot erstellt und bei Auswahl durch den Auftraggeber ein Projektgegenstand entwickelt, welcher abschließend an den Auftraggeber ausgeliefert und von diesem abgenommen wird \cite{reinhard2008,2004vmodell}.\newline
Bei \textit{Einführung und Pflege eines organisationsspezifischen Vorgehensmodells} geht es um Projekte, welche Prozessmodelle z.B. das V-Modell einführen und verbessern wollen. Für diesen Zweck ist eine Analyse des vorherigen Prozessmodelles notwendig und etwaige Verbesserungsmöglichkeiten sind zu erfassen und durchzuführen \cite{reinhard2008,2004vmodell}.\newline
Wie aus Abbildung \ref{fig:Projekttypen} ersichtlich ist, kann es sich im V-Modell XT beim Projektgegenstand um ein Hardware (HW)-System, ein Software (SW)-System, ein eingebettetes System oder eine Systemintegration handeln \cite{brack2010,2004vmodell}. \newline

\subsubsection{Projekttypvarianten}
Für jeden der Projekttypen, gibt es im V-Modell XT mindestens eine passende Projekttypvariante. Diese bestimmt die Rahmenbedingungen für mögliche Abläufe eines Projektes. In Abbildung \ref{fig:Projekttypen} sind die verschiedenen Projekttypvarianten des V-Modell XT aufgelistet und es wird gezeigt, mit welchen Merkmalen die zugehörigen Projekttypvarianten ausgewählt werden können \cite{2004vmodell}.   
\begin{figure}[htp]
\begin{center}
  \includegraphics[width=\linewidth]{Projekttypvarianten-v-modell} %pdf, jpg, png...
  \caption{Zuordnung der Projekttypvarianten zu den Projekttypen des V-Modell XT \cite{2004vmodell}}
  \label{fig:Projekttypen}
\end{center}
\end{figure}
 Hieraus ist ersichtlich, dass für den Projekttyp \textit{Einführung und Pflege eines organisationsspezifischen Vorgehensmodells} nur eine einzige Projekttypvariante existiert \cite{2004vmodell}.\newline
Für den Projekttyp \textit{Systementwicklungsprojekt (AG)} existieren zwei verschiedene Projekttypvarianten. Falls der Auftraggeber mit nur einem Auftragnehmer zusammen arbeitet, ergibt sich die Projekttypvariante \textit{Systementwicklungsprojekt (AG)- Projekt mit einem Auftragnehmer}. Arbeitet der Auftraggeber mit mehreren Auftragnehmern zusammen ergibt sich die Projekttypvariante \textit{Systementwicklungsprojekt (AG)- Projekt mit mehreren Auftragnehmern}\cite{2004vmodell}.\newline
Bei den Projekttypen \textit{Systementwicklungsprojekt (AN)} und \textit{Systementwicklungsprojekt (AG/AN)} wird die Unterscheidung anhand des Systemlebenszyklusausschnitt des Projektes durchgeführt. Somit in wird den Systemlebenszyklusausschnitten Entwicklung, Weiterentwicklung und Migration eine andere Projekttypvariante gewählt, als in Wartung und Pflege \cite{2004vmodell}.\newline 
  
 \subsubsection{Vorgehensbausteine}
Modulare, aufeinander aufbauende Vorgehensbausteine bilden den Kern des V-Modell XT. Vorgehensbausteine sind selbständig entwickelbare und änderbare Einheiten und bestehen aus Aktivitäten, Produkten und Rollen. Sie geben einerseits vor, \grqq Was\grqq{}  in einem Projekt zu tun ist, also welche Produkte zu erstellen sind und andererseits \grqq Wer\grqq, also welche konkrete Rolle für das jeweilige Produkt verantwortlich ist. Abbildung \ref{fig:vorgehensbausteine} gibt einen Überblick über diese \cite{ruf2008, 2004vmodell}.\newline

\begin{figure}[htp]
\begin{center}
  \includegraphics[width=\linewidth]{vorgehensbaustein} %pdf, jpg, png...
  \caption{Vorgehensbausteine V-Modell XT nach \cite{2004vmodell}}
  \label{fig:vorgehensbausteine}
\end{center}
\end{figure}

Ergebnisse und Zwischenergebnisse werden Produkte genannt. Komplexe Produkte können in ein oder mehrere Themen gegliedert werden und inhaltlich zusammengehörende Produkte können zu einer Disziplin zusammengefasst werden. Produkte können hierbei auch voneinander abhängig sein, sowohl innerhalb eines Vorgehensbausteins, als auch zwischen verschiedenen Vorgehensbausteinen \cite{2004vmodell}.\newline
Jedes Produkt wird von genau einer Aktivität fertig gestellt. Aktivitäten legen auch fest, wie die einzelnen Produkte zu bearbeiten sind. Sie bestehen aus einer oder mehreren Teilaktivitäten, sogenannten Arbeitsschritten. Diese stellen eine Art Arbeitsanleitung dar und bearbeiten eine oder mehrere Themen \cite{2004vmodell}.\newline
Durch Rollen werden eine Menge von Aufgaben und Verantwortlichkeiten gekapselt, wodurch das V-Modell XT unabhängig von organisatorischen Rahmenbedingungen bleibt. Eine Zuordnung von Personen, bzw. Organisationseinheiten zu einer Rolle erfolgt erst zu Beginn eines Projektes. Es wird jedem Produkt genau eine Rolle als Verantwortlicher zugewiesen, weitere Rollen können am Produkt als Mitwirkende mitarbeiten \cite{2004vmodell}. \newline



\subsubsection{V-Modell-Kern und Vorgehensbausteinlandkarte}

Um ein spezifisches Projekt an ein V-Modell-Projekt anzupassen, ist für jeden Projekttyp und jede Projekttypvariante genau vorgegeben, welche Vorgehensbausteine jeweils anzuwenden sind \cite{2004vmodell}. Hierdurch kann also ein individuelles V-Modell für ein Projekt erstellt werden \cite{heinrich2007}. Hierfür ist es notwendig, die Vorgehensbausteine für ein V-Modell-Projekt nach den Vorgaben des Projekttyps auszuwählen und festzulegen \cite{2004vmodell}. \newline

\begin{figure}[htp]
\begin{center}
  \includegraphics[width=\linewidth]{landkarte} %pdf, jpg, png...
  \caption{V-Modell-Kern und Vorgehensbausteinlandkarte nach \cite{2004vmodell}}
  \label{fig:landkarte}
\end{center}
\end{figure}

 Wie Abbildung \ref{fig:landkarte} zeigt, können die Vorgehensbausteine in die vier Bereiche \textit{Alle V-Modell-Projekte}, \textit{Organisationsspezifisches Vorgehensmodell}, \textit{AG/AN-Schnittstelle} und \textit{Systementwicklung} eingeteilt werden \cite{2004vmodell}.\newline
 Im Bereich \textit{Alle V-Modell-Projekte} finden sich diejenigen Vorgehensbausteine, welche in jedem V-Modell-Projekt herangezogen werden können. Zudem gibt es den V-Modell-Kern, in welchem sich die Vorgehensmodelle finden, die in jedem V-Modell-Projekt unerlässlich sind: \textit{Projektmanagement}, \textit{Konfigurationsmanagement}, \textit{Problem- und Änderungsmanagement} und \textit{Qualitätssicherung}. Zusätzlich zu diesen verpflichtenden Vorgehensbausteinen können in jedem Projekt noch \textit{Kaufmännisches Projektmanagement}, welches bei der Integration des Projektmanagements in das kaufmännische Management hilft und \textit{Messung und Analyse}, welches Verfahren für die organisationsweite und projektübergreifende Erfassung und Auswertung von Kennzahlen bereitstellt, verwendet werden \cite{2004vmodell}.\newline
 Ist der Zweck eines Projektes die Entwicklung eines \textit{Organisationsspezifischen Vorgehensmodells}, so muss der Vorgehensbaustein \textit{Einführung und Pflege eines organisationsspezifischen Vorgehensmodells} hinzugenommen werden. In diesem finden sich Verfahren und Richtlinien für die Einführung eines Vorgehensmodells innerhalb einer Organisation sowie die damit einhergehende Etablierung eines stetigen Verbesserungsprozesses \cite{2004vmodell}.\newline
 Wenn ein Projekt die Entwicklung eines Systems zum Ziel hat so wird der Bereich \textit{Systementwicklung} herangezogen. In diesem befinden sich die Vorgehensbausteine \textit{Anforderungsfestlegung}, \textit{Systemerstellung}, \textit{HW-Entwicklung}, \textit{SW-Entwicklung}, \textit{Logistikkonzeption}, \textit{Weiterentwicklung und Migration von Altsystemen}, \textit{Evaluierung von Fertigprodukten}, \textit{Benutzbarkeit und Ergonomie}, \textit{Sicherheit} sowie \textit{Sicherheit (AN)} und \textit{Multi-Projektmanagement} \cite{2004vmodell}. \newline
 Im Bereich \textit{AG/AN-Schnittstelle} befinden sich die Vorgehensbausteine für die Kommunikation zwischen Arbeitgeber und Arbeitnehmer: \textit{Lieferung und Abnahme (AG)}, \textit{Lieferung und Abnahme (AN)}, \textit{Vertragsschluss (AG)} und \textit{Vertragsschluss (AN)}. Hier finden sich Regelungen über den Vertrag zwischen Arbeitgeber und Arbeitnehmer sowie über Lieferung und Abnahme des Entwicklungsgegenstandes \cite{2004vmodell}. \newline
 
 \subsubsection{Projektdurchführungsstrategie}
 Die Vorgehensbausteine im V-Modell XT geben zwar an, welche Produkte jeweils zu erstellen und welche Aktivitäten durchzuführen sind, sie geben jedoch hierbei nicht vor, in welcher Reihenfolge dies geschehen soll. Damit das Projekt trotzdem geplant und gesteuert werden kann, gibt es im V-Modell eine Projektdurchführungsstrategie, welche auf den jeweiligen Projekttyp und die Projekttypvariante abgestimmt ist. Hier wird somit die Reihenfolge der Produkte und Aktivitäten festgelegt, also das  \grqq Wann\grqq {}. festgelegt. Außerdem werden hier zu erreichende Projektfortschrittsstufen vorgegeben \cite{2004vmodell}. \newline
 
 \subsubsection{Entscheidungspunkte}
Abbildung \ref{fig:entscheidungspunkte} zeigt, dass die in der Projektdurchführungsstrategie vorgegebenen Projektfortschrittsstufen bei Erreichen durch Entscheidungspunkte markiert werden, welche einen Meilenstein im Projektablauf darstellen. Um den Entscheidungspunkt zu erreichen muss eine vorgegebene Menge an Produkten fertig gestellt werden. Hier entscheidet das Projektmanagement über das Erreichen der Projektfortschrittsstufe und das Freigeben des nächsten Projektabschnitts. Die Entscheidungspunkte, welche im V-Modell XT erreicht werden müssen, können Abbildung \ref{fig:v-modell} entnommen werden. Diese werden wie im V-Modell-Kern in die vier Bereiche \textit{Alle V-Modell-Projekte}, \textit{Organisationsspezifisches Vorgehensmodell}, \textit{AG/AN-Schnittstelle} und \textit{Systementwicklung} unterschieden \cite{2004vmodell}. \newline
Demnach gelten die Entscheidungspunkte \textit{Projekt genehmigt}, \textit{Projekt definiert}, \textit{Iteration geplant} und \textit{Projekt abgeschlossen} für alle Projekttypen und Projektdurchführungsstrategien \cite{2004vmodell}. \newline
Bei der Systementwicklung werden die Entscheidungspunkte \textit{Anforderungen festgelegt}, \textit{System spezifiziert}, \textit{System entworfen}, \textit{Feinentwurf abgeschlossen}, \textit{Systemelemente realisiert} und \textit{System integriert} verwendet. Falls das Projekt vor der Anforderungserhebung in mehrere Teilmodelle aufgeteilt werden soll, werden zusätzlich die Entscheidungspunkte \textit{Gesamtprojekt aufgeteilt} und \textit{Gesamtprojektfortschritt überprüft} hinzugenommen \cite{2004vmodell}. \newline
Die Entscheidungspunkte für die Arbeitgeber/Arbeitnehmer Schnittstelle setzen sich aus \textit{Projekt ausgeschrieben}, \textit{Angebot abgegeben}, \textit{Projekt beauftragt}, \textit{Lieferung durchgeführt}, \textit{Abnahme erfolgt} und \textit{Projektfortschritt überprüft} zusammen \cite{2004vmodell}. \newline
 Bei der Entwicklung eines organisationsspezifischen Vorgehensmodells kommen die Entscheidungspunkte \textit{Vorgehensmodell analysiert}, \textit{Verbesserung Vorgehensmodell konzipiert} und \textit{Verbesserung Vorgehensmodell realisiert} zum Einsatz \cite{2004vmodell}. \newline
 Die Entscheidungspunkte legen das \grqq Wann\grqq {} und \grqq Was\grqq {} fest, d.h. wann welche Produkte fertig gestellt sein müssen.

 
 
 \begin{figure}[htp]
\begin{center}
  \includegraphics[width=\linewidth]{Entscheidungspunkte} %pdf, jpg, png...
  \caption{Entscheidungspunkte V-Modell XT nach \cite{2004vmodell}}
  \label{fig:entscheidungspunkte}
\end{center}
\end{figure}
 
\begin{sidewaysfigure}[htp]
\begin{center}
  \includegraphics[scale=0.7]{v-modell} %pdf, jpg, png...
  \caption{Entscheidungspunkte für die Projektdurchführungsstrategie nach \cite{2004vmodell}}
  \label{fig:v-modell}
\end{center}
\end{sidewaysfigure}
\subsection{Imperative Modellierung V-Modell}
\subsection{Deklarative Modellierung V-Modell}
\subsection{Vergleich}




\chapter{Validierung}\label{sec:chapter7}

In diesem Kapitel werden die Ergbnisse des Vergleichs der ConDec - und BPMN-Modelle aus Kapitel 5 mit Hilfe einer Studie validiert. Im Rahmen der Studie ist es nur möglich die Grundsätze der \textit{Klarheit} und der \textit{Vergleichbarkeit} zu testen.

\section{Forschungsfragen}

\textbf{Forschungsfrage 1a}: 

Ist die Punktsumme bei den Verständnisfragen insgesamt bei BPMN oder bei ConDec höher?\newline



\textbf{Forschungsfrage 1b}: 

Ist die Punktsumme bei den Verständnisfragen bei den kleinen Modellen bei BPMN oder bei ConDec höher?\newline

\textbf{Forschungsfrage 1c}: 

Ist die Punktsumme bei den Verständnisfragen bei den großen Modellen bei BPMN oder bei ConDec höher? \newline

\textbf{Forschungsfrage 1d}: 

Gibt es Unterschiede im Ergebnis in Abhängigkeit des Hintergrundwissesns der Versuchsobjekte über Prozessmodellierung?  \newline

\textbf{Forschungsfrage 2a}: 

Werden bei den Meinungsfragen insgesamt die Modelle von BPMN oder von ConDec präferiert? \newline

\textbf{Forschungsfrage 2b}: 

Werden bei den Meinungsfragen bei den kleinen Modellen die von BPMN oder von ConDec präferiert?\newline

\textbf{Forschungsfrage 2c}: 

Werden bei den Meinungsfragen bei den großen Modellen die von BPMN oder von ConDec präferiert?\newline

\section{Design der Studie}

Abbildung \ref{fig:UmfrageStruktur} kann die Struktur der Umfrage entnommen werden.
Bei der Studie wurden zwei Fragebögen eingesetzt. Die 32 Teilnhemer der Studie wurden deswegen zufallsbedingt in zwei Gruppen (je 16 Personen) eingeteilt. Zunächst wurden den Probanden allgemeine demographische Fragen gestellt (Geschlecht, Alter, Hintergrundwissen zu imperativer und deklarativer Modellierung, Hintergrundwissen zu den Software Engineering Prozessmodellen Scrum, Open UP und V-Modell XT).\newline
Anschließend wurden den Teilnehmern Verständnisfragen zu ausgewählten Modellen gestellt.
Es wurden die vier  Modellpaare \textit{Lösungsinkrement entwickeln (Open UP)}, \textit{Scrum}, \textit{Systementwicklungsprojekt AG/AN} und \textit{Phasen Open UP -Inception}  aus Kapitel 5 ausgewählt. Hierbei wurde darauf geachtet, dass es sich um zwei kleine Modelle (<= 5 Aktivitäten) und zwei große Modelle (> 5 Aktivitäten) handelt. Gruppe 1 startete mit einem deklarativen Prozess und Gruppe 2 mit dem entsprechenden imperativen Prozess. Somit wurden von jeder Gruppe zwei imperative und zwei deklarative Prozesse bearbeitet.  \newline
Im letzten Teil des Fragebogens wurden den Probanden noch vier Modellpaare direkt gegenüber gestellt und sie wurden nach Ihrem präferierten Modell (deklarativ oder imperativ) gefragt und mussten in einem Freitextfeld den Grund für Ihre Entscheidung angeben. Auch hier wurden den Teilnehmern wiederum zwei kleine (<= 5 Aktivitäten) und zwei große (> 5 Aktivitäten)  Modelle gezeigt. Hier wurden die Modelle \textit{System spezifizieren (V-Modell XT)}, \textit{Phasen des Open UP}, \textit{Inkrementelle Entwicklung durchführen (V-Modell XT)} und \textit{Release deployen} ausgewählt. \newline
Die semantische Gleichheit der Modellpaare wurde wie bereits in Kapitel 5 erwähnt durch das Testen von validen Pfaden durch die jeweiligen Modelle sichergestellt.\newline

\begin{figure}[htp]
\begin{center}
  \includegraphics [width=\textwidth]{UmfrageStruktur} %pdf, jpg, png...
  \caption{Struktur der Umfrage}
  \label{fig:UmfrageStruktur}
\end{center}
\end{figure}

\subsubsection{Verständnisfragen}

Zu jedem Modell wurden den Teilnehmern jeweils acht Verständnisfragen gestellt. Diese zielten auf das Verständnis der möglichen Reihenfolge der Aktivitäten, mögliche Start- und Endaktivitäten, allgemeine informationen aus dem Modell sowie parallele Abläufe von Aktivitäten, sich ausschließende Aktivitäten und die Anzahl möglicher Ausführungen von Aktivitäten.\newline
Die Teilnehmer konnten bei der Beantwortung der Fragen zwischen vier Antwortmöglichkeiten wählen: \textit{Ja}, \textit{Nein}, \textit{Geht nicht aus Modell hervor} und \textit{Unentschlossen}.

\subsubsection{Umfragewerkzeug und Durchführung}

Zur Durchführung wurde das Fragebogenwerkzeug \textit{Limesurvey} verwendet. Der entsprechende Link zum Fragebogen sowie eine kleine Legende zur Notationsübersicht von Declare und BPMN wurde den Teilnehmern per E-Mail zugeschickt. Die entsprechenden Antworten der Probanden wurden automatich von \textit{Limesurvey} gespeichert. Weiterhin war es dort möglich die jeweilige Zeit, welche die Probanden zur Bearbeitung Verständnisfragen benötigt haben, mit zu messen. Die gespeicherten Daten können aus \textit{Limesurvey} für verschiedene exterene Anwendungen exportiert werden (z.B als Excel, CSV oder für SPSS).

\subsubsection{Auswertung}

Für jede richtige Antwort wurde ein Punkt vergeben. Für jede falsche Antwort gab es null Punkte. Auch \textit{Unentschlossen} wurde als falsche Antwort gewertet. Die einzelnen Punkte wurden dann pro Frage aufsummiert, so dass ein maximaler wert Pro Frage von 1 möglich ist.

\subsection{Durchführung der Studie}

\subsubsection{Teilnehmer}

Es wurden 32 Studenten und Doktoranden aus dem Bereich Informatik/Medieninformatik befragt. 12 Teilnehmer waren weiblich und 20 männlich (Abbildung \ref{fig:Geschlechterverteilung}). Die allgemeinen demographischen Daten der Probanden können Abbildung \ref{fig:TabelleAllgemeineDaten} entnommen werden. Diese hatten unterschiedliches Hintergrundwissen zum Thema Prozessmodellierung. Wie Abbildung \ref{fig:VerteilungImperativDeklarative} entnommen werden kann, hatten sieben Studienteilnehmer weder in imperativer noch in deklarativer Modellierung Erfahrung. 18 Probanden hatten nur in imperativer Modellierung Erfahrung, jedoch nicht in deklarativer und sieben weitere Teilnehmer hatten in beiden Modellierungssprachen Erfahrung. Die Versuchsobjekte wurden bewusst nach unterschiedlichem Hintergrundwissen zum Thema Prozessmodellierung ausgewählt, um zu prüfen, inwiefern sich die Ergebnisse bei den Verständnisfragen zwischen Personen mit viel und wenig Hintergrundwissen zum Thema Prozessmodellierung unterscheiden.\newline

\begin{figure}[htp]
\begin{center}
  \includegraphics{Geschlechterverteilung} %pdf, jpg, png...
  \caption{Geschlechterverteilung}
  \label{fig:Geschlechterverteilung}
\end{center}
\end{figure}

\begin{figure}[htp]
\begin{center}
  \includegraphics{VerteilungImperativDeklarative} %pdf, jpg, png...
  \caption{Verteilung Erfahrung imperative und deklarative Modellierung}
  \label{fig:VerteilungImperativDeklarative}
\end{center}
\end{figure}

\begin{figure}[htp]
\begin{center}
  \includegraphics[width=\textwidth]{TabelleAllgemeineDaten} %pdf, jpg, png...
  \caption{Allgemeine demographische Daten}
  \label{fig:TabelleAllgemeineDaten}
\end{center}
\end{figure}



\subsubsection{Ergebnisse Verständnisfragen}

Abbildung \ref{fig:Frage1} zeigt die Ergebnisse der Verständnisfragen zum Modell \textit{Open UP: Lösungsinkrement entwickeln}. Die Ergebnisse variieren hier zwischen den deklarativen und imperativen Modellen. Während die Ergebnisse teilweise gleich sind, bzw. nur wenig voneinander abweichen, liegen die Ergebnisse der deklarativen Modelle bei den Fragen 5 und 6 deutlich unter den Ergebnissen der imperativen Modelle. \newline
Frage 5 lautete: \textit{Nach Ausführung der Aktivität \grqq Integrieren\grqq \ endet der Prozess in jedem Fall sofort}. Hier wurde von den Teilnehmern die imperative XOR- Verknüpfung besser verstanden, als die deklarative Darstellung des Ablaufes. Auch bei Frage 6 (\textit{Als erste Aktivität im Prozess kann die Aktivität \grqq Entwickeltest implementieren\grqq \ ausgeführt werden}) war den Probanden die imperative XOR-Darstellung, wohl in Verbindung mit dem BPMN Startsymbol als klaren Einstiegspunkt klarer, als die entsprechende deklarative Darstellung.\newline

\begin{figure}[htp]
\begin{center}
  \includegraphics[scale=0.8]{Frage1} %pdf, jpg, png...
  \caption{Ergebnisse Verständnisfrage 1 aller Teilnehmer}
  \label{fig:Frage1}
\end{center}
\end{figure}


Die Ergebnisse der Verständnisfrage 2 zum Modell \textit{Scrum} von allen Teilnehmern kann Abbildung \ref{fig:Frage2} entnommen werden.  Bei Scrum handelte es sich um ein großes Modell (>5 Aktivitäten), welches sowohl viele Verzweigungen, als auch viele parallele Aktivitäten aufweist. Auch hier weichen die Ergebnisse zwischen den imperativen und deklarativen Modellen voneinander ab.\newline
Nur bei der ersten Frage (\textit{Ein Scrum Meeting dauert 15 Minuten}) schneidet der deklarative Prozess besser ab, als der imperative. Diese allgemeine Information aus dem Prozess befand sich bei beiden Prozessen in der Beschriftung der Aufgabe \textit{15-minütiges Scrum Meeting durchführen}. Der imperative Scrum-Prozess weist insgesamt mehr Elemente auf, als der deklarative. Daher fiel es wohl den Probanden hier einfacher die Übersicht über allgemeine Informationen zu behalten.\newline
Bei den Fragen 2 und 8 hat das deklarative Modell eine sehr schlechte Punktzahl erreicht. Bei Frage 2 (\textit{Die Aktivität \grqq Task abarbeiten\grqq \ kann beliebig ausgeführt werden}) konnten die Teilnehmer der XOR-Verknüpfung im imperativen Modell, welche eine Rückschleife auf die Aktivität \textit{Task abarbeiten} mehr folgen, als der Darstellung der Aktivität im deklarativen Modell. \newline
Das gleiche gilt für Frage 8 (\textit{Nach Beendigung der Aufgabe \grqq Task abarbeiten\grqq \ endet der Prozess sofort}). Auch hier wurde die Verzweigung und das damit mögliche Zurückkehren zur Aufgabe \textit{Sprint-Planning-Meeting durchführen} durch eine XOR-Verknüpfung im imperativen Modell dargestellt und war somit für die Teilnehmer klarer verständlich.\newline


\begin{figure}[htp]
\begin{center}
  \includegraphics[scale=0.8]{Frage2} %pdf, jpg, png...
  \caption{Ergebnisse Verständnisfrage 2 aller Teilnehmer}
  \label{fig:Frage2}
\end{center}
\end{figure}

Beim Modell \textit{V-Modell: Systementwicklungsprojekt AG/AN} der Verständnisfrage 3 lagen die Ergebnisse des deklarativen Modell bis auf Frage 3 immer unter denen des imperativen Modells. Starke Abweichung gab es bei den Fragen 2, 4, 5, 7. \newline
Bei Frage 2 (\textit{Die Aktivitäten \grqq Prototypische Entwicklung durchführen\grqq, \grqq Komponentenbasierte Entwicklung durchführen\grqq \ und \grqq Inkrementelle Entwicklung durchführen\grqq \ können parallel zueinander ausgeführt werden}) war den Teilnehmern, welche das deklarative Modell bearbeiten mussten, die Notation des Constraints \textit {Exclusive Choice 1 of 3} nicht ganz klar. Sechs der 16 Probanden kreuzten hier entweder \textit{Ja} oder \textit{Unentschlossen} an.\newline
Sowohl Frage 4 (\textit{Nach Ausführung der Aktivität \grqq System abnehmen\grqq \ kann die Aktivität \grqq Anforderungen festlegen\grqq \ ausgeführt werden}, als auch Frage 5 (\textit{Nach Ausführung der Aktivität \grqq System abnehmen\grqq \ kann die Aktivität \grqq Projekt ausschreiben\grqq \ ausgeführt werden}) zielten wieder auf Verzweigungen des Prozesses ab und waren den Teilnehmern mit dem imperativen Prozess klarer. \newline
Frage 7 (\textit{Nach Ausführung der Aktivität \grqq Projekt abschließen\grqq \ endet der Prozess}) wurde von den Probanden, welchen das imperative Modell gezeigt wurde richtiger beantwortet. Hier war im imperativen Modell durch das BPMN-Ende-Symbol den Teilnehmern das Ende des Prozesses wohl klarer, als die Darstellung durch das Constraint \textit{not succession} im deklarativen Modell.\newline

\begin{figure}[htp]
\begin{center}
  \includegraphics[scale=0.8]{Frage3} %pdf, jpg, png...
  \caption{Ergebnisse Verständnisfrage 3 aller Teilnehmer}
  \label{fig:Frage3}
\end{center}
\end{figure}

Die Ergebnisse des Prozesses \textit{Open UP: Inception} weichen zwischen den deklarativen und imperativen Modellen nicht stark voneinander ab. \newline
Lediglich bei Frage 6 (\textit{Die Aktivität \grqq Iteration planen und managen\grqq \ kann beliebig oft ausgeführt werden}) und Frage 7 (\textit{Die Aktivitäten \grqq Anforderungen identifizieren und verfeinern\grqq \ und \grqq auf technisches Vorgehen einigen\grqq \ können beliebig oft ausgeführt werden}) war den Teilnehmern wohl teilweise die funktion des Existenz (1) Constraints nicht ganz klar oder wurde übersehen.\newline

\begin{figure}[htp]
\begin{center}
  \includegraphics[scale=0.8]{Frage4} %pdf, jpg, png...
  \caption{Ergebnisse Verständnisfrage 4 aller Teilnehmer}
  \label{fig:Frage4}
\end{center}
\end{figure}

\clearpage


\subsubsection{Ergebnisse Meinungsfragen}

Abbildung \ref{fig:Meinungsfrage1} zeigt, dass 31 der 32 Befragten beim Modell \textit{System spezifizieren} das imperative Modell bevorzugen. Nur eine Person zog das deklarative Modell vor. Hierbei handelt es sich um einen Teilnehmer, welcher weder in imperativer, noch in deklarativer Prozessmodellierung Erfahrung aufweist. Als Begründung für den Vorzug des deklarativen Modells gab der Befragte an, das Modell sei kompakter, jedoch sei auch mehr Verständnis notwendig.\newline
Die Probanden, welche das imperative Modell bevorzugten gaben verschiedene Gründe hierfür an. Unter anderem nannten sie als Grund die klarere Struktur des BPMN-Modells, die vielen unterschiedlichen/komplexen Elemente im deklarativen Modell oder auch die klare Rollenverteilung durch die Swimlanes. Einige Befragte gaben auch ihre bessere Kenntniss in imperativen Prozessmodellierungssprachen als Grund an.\newline

\begin{figure}[htp]
\begin{center}
  \includegraphics[scale=0.8]{Meinungsfrage1} %pdf, jpg, png...
  \caption{Ergebnisse Meinungsfrage 1 aller Teilnehmer}
  \label{fig:Meinungsfrage1}
\end{center}
\end{figure}

Ebenfalls beim Modell \textit{Phasen Open UP} bevorzugt eine deutliche Mehrheit (26 von 32 Befragten) das imperative Modell, wie Abbildung \ref{fig:Meinungsfrage2} entnommen werden kann. \newline
Hierbei verfügte nur einer der sechs Personen, welche das deklarative Modell bevorzugte auch über Erfahrung in deklarativer Modellierung. Die anderen Probanden verfügten entweder über keine Erfahrungen in beiden Modellierungssprachen (zwei) oder nur über Erfahrungen in imperativer Modellierung (drei). Als Grund für ihre Wahl gaben die Probanden beispielsweise an, dass das deklarative Modell kompakter sei und dass es klarer sei, dass nur bei Erfolg die nächste Aktivität ausgeführt wird.\newline
Die Personen, welche das imperative Modell präferierten gaben an, dass sie den Ablauf mit den Schleifen im imperativen Modell klarer finden und dass sie dem Sequenzfluss besser folgen könnten.\newline


\begin{figure}[htp]
\begin{center}
  \includegraphics[scale=0.8]{Meinungsfrage2} %pdf, jpg, png...
  \caption{Ergebnisse Meinungsfrage 2 aller Teilnehmer}
  \label{fig:Meinungsfrage2}
\end{center}
\end{figure}

Die Ergebnisse des dritten Modellpaares \textit{Inkrementelle Entwicklung} zeigt Abbildung \ref{fig:Meinungsfrage3}. Demanch präferierten nur zwei Personen (eine Person mit Erfahrung sowohl in imperativer, als auch in deklarativer Modellierung, eine Person ohne imperative und deklarative Modellierungserfahrung) das deklarative Modell und 30 Probanden ziehen das imperative Modell vor.\newline
Als Grund für den Vorzug des imperativen Modells wurde die Menge an unterschiedlichen Symbolen beim imperativen Modell genannt. \newline
Die 30 Personen, welchen das imperative Modell besser gefiel gaben an, dass sie die imperative Notation verständlicher finden, der Ablauf im imperativen Modell klarer erkennbar sei, sie keinen Anhaltspunkt haben, wo im deklarativen Modell gestartet bzw. geendet wird und die vielen verschiedenen Constraints im deklarativen Modell es erschweren den Ablauf nachzuvollziehen.\newline

\begin{figure}[htp]
\begin{center}
  \includegraphics[scale=0.8]{Meinungsfrage3} %pdf, jpg, png...
  \caption{Ergebnisse Meinungsfrage 3 aller Teilnehmer}
  \label{fig:Meinungsfrage3}
\end{center}
\end{figure}

Beim Modell \textit{Open UP: Release deployen} präferierten neun Personen das deklarative Modell und 23 Teilnehmer das imperative Modell. Von den neun Teilnehmern, welche das deklarative Modell präferierten hatte nur einer Erfahrung in deklarativer Modellierung, einer hatte weder in deklarativer noch in imperativer Modellierung Erfahrung und sieben hatten nur in imperativer Modellierung Erfahrung. \newline
Als Begründung für die Wahl des deklarativen Modelles wurde die Übersichtlichkeit desselbigen genannt und zwar auf Grund der fehlenden Artefakte im Modell. Es wurde bemängelt, die vielen Artefakte würden das imperative Modell unübersichtlich machen.\newline
Die Probanden, welche das imperative Modell besser fanden gaben als Gründe den klaren Anfang und das klare Ende des Prozesses an, die bessere Verständlichkeit der Optionalität der Aktivität \textit{Backoutplan ausführen} und die fehlenden Artefakte im deklarativen Modell. \newline



\begin{figure}[htp]
\begin{center}
  \includegraphics[scale=0.8]{Meinungsfrage4} %pdf, jpg, png...
  \caption{Ergebnisse Meinungsfrage 4 aller Teilnehmer}
  \label{fig:Meinungsfrage4}
\end{center}
\end{figure}

\clearpage

\subsection{Fazit der Studie}

Durch die Studie konnten die Ergebnisse des Vergleichs aus Kapitel 5 größtenteils belegt werden.\newline
Bei Modellen, welche viele Verzweigungen/Schleife beinhalten war für die Teilnehmer BPMN verständlicher. Dies zeigte sich beim kleinen Prozess \textit{Open UP: Lösungsinkrement entwickeln} und noch deutlicher beim großen Modell \textit{V-Modell: Systementwicklungsprojekt AG/AN}. Bei diesen beiden Prozessen sind in ConDec deutlich mehr Constraints, vor allem viele verschiedene Constraints zur korrekten Darstellung des Ablaufs notwendig, als Gateways in BPMN. Dadurch haben die ConDec-Modelle eine deutlich höhere Komplexität, als die BPMN-Modelle.\newline
Hier fällt besonders auf, dass Fragen bezüglich der Reihenfolge der Aktivitäten bei ConDec bei direkt aufeinander folgenden Aktivitäten größtenteils richtig beantwortet wurden. Jedoch bei Abläufen, bei denen es durch Verzweigungen im Prozessablauf zu einem Sprung kommt, wurden nur die Fragen zu den BPMN-Modellen größtenteils richtig beantwortet. Bei ConDec wurden diese Fragen häufig falsch beantwortet.
 








\chapter{Verwandte Arbeiten}\label{sec:chapter9}

In diesem Kapitel werden verwandte Arbeiten vorgestellt. Es werden Arbeiten zu den Themen Modellierung von Softwareentwicklungsprozessen, Verständlichkeit von Prozessmodellierungssprachen und Vergleich von Prozessmodellierungssprachen beschrieben und gegenüber der Thematik der vorliegenden Arbeit abgegrenzt.

\section{Modellierung von Softwareentwicklungsprozessen}

Es gibt schon einige Arbeiten, in welchen es um die Modellierung von Software Engineering Prozessmodellen in BPMN geht.
\cite{Menhorn2014} beschäftigt sich mit der Analyse und der Überführung von Softwareentwicklungsprozessen in die Prozessmodellierungssprache BPMN und der Erweiterung von BPMN bei eventuellen Grenzen der Darstellbarkeit.\newline
Weiterhin gibt es drei Arbeiten bzw. Blogeinträge, bei welchen es um die Modellierung von Scrum \cite{software}, Open UP \cite{brunner2007fallstudie} und V-Modell XT \cite{Bregenzer2014} in BPMN geht. Hier werden jeweils Teile der drei Software-Engineering Prozessmodelle analysiert und anschließend in BPMN modelliert.\newline
In \cite{sabrina734, sabrina758, sabrina795} wurden Softwareentwicklungsprozesse in einem System umgesetzt. Das Ziel war es, dass die Softwareentwicklungsprozesse dort komplett ausführbar sind. Die Softwareentwicklungsprozesse wurden imperativ umgesetzt. Eine deklarative Umsetzung erfolgte für dynamische Abläufe, welche im laufenden Tagesgeschehen außerhalb der Modelle abgearbeitet wurden.
 Im Gegensatz zur vorliegenden Arbeit wurde in den Arbeiten \cite{software}, \cite{brunner2007fallstudie}, \cite{Bregenzer2014} jeweils nur ein Softwareentwicklungsprozess modelliert und auch nur in der imperativen Prozessmodellierungssprache BPMN. In \cite{Menhorn2014} wurden ebenfalls Teile von Scrum, Open UP und V-Modell XT modelliert, jedoch nur in der imperativen Prozessmodellierungssprache BPMN. Der Fokus der Arbeit \cite{Menhorn2014} lag auf der Erweiterung von BPMN um zusätzliche Notationselemente.\newline
 In der vorliegenden Arbeit hingegen werden Teile von allen drei Softwareentwicklungsprozessmodellen in der imperativen Prozessmodellierungssprache BPMN und der deklarativen Prozessmodellierungssprache ConDec modelliert. Der Fokus der vorliegenden Arbeit liegt zudem auf der Beurteilung der Eignung zur Modellierung der beiden Prozessmodellierungssprachen und auf deren Vergleich und nicht auf der Modellierung der Softwareentwicklungsprozesse an sich oder der Erweiterung der Notationsumfänge der beiden Prozessmodellierungssprachen.\newline
 In \cite{sabrina795, sabrina734, sabrina758} wurden ebenfalls Softwareentwicklungsprozesse imperativ und deklarativ umgesetzt. Jedoch erfolgte dort im Gegensatz zur vorliegenden Arbeit kein Vergleich zwischen imperativen und deklarativen Modellierungsansätzen, sondern es wurde je nachdem, ob es sich um einen statischen oder dynamischen Prozess handelt, ein imperativer oder deklarativer Ansatz gewählt.\newline
Keine der hier vorgestellten Arbeiten beschäftigt sich mit dem Vergleich der Anwendbarkeit von imperativen und deklarativen Prozessmodellierungsansätzen im Kontext von Softwareentwicklungsprozessen.


\section{Verständlichkeit von Prozessmodellierungssprachen}

Verwandte Arbeiten zur Thematik Verständlichkeit von Prozessmodellierungssprachen werden im Folgenden vorgestellt.

Es existieren bereits einige Arbeiten, welche die Verständlichkeit von Prozessmodellen untersuchen. \cite{bpm07} z.B. beleuchtet die Verständlichkeit von Ereignisgesteuerten Prozessketten (EPK), während sich \cite{gruhn2006complexity} der Komplexität und Verständlichkeit von BPMN und UML-Diagrammen widmet. \cite{reijers2011study} untersucht die Verständlichkeit von EPK und BPMN im Hinblick auf die Komplexität der XOR-, OR- und UND-Verzweigungen und in \cite{gruhn2006adopting} wird die Verständlichkeit von BPMN durch Zuordnung von kognitiven Werten zu den einzelnen Notationselementen bei BPMN beurteilt. \newline
In \cite{pinggera2012tracing, forster2012collaborative, pinggera2010investigating} wird der Prozess der Prozessmodellierung aus individueller und kollaborativer Sicht untersucht. Es wird beleuchtet, wie Prozessmodellierer beim Erstellen von Prozessmodellen vorgehen. Zum Erforschen dieses Vorgehens wird auch ein System entwickelt, welches den Prozess des Prozessmodellierens des Anwenders analysieren kann. \newline
In \cite{sabrina942} wird die Verständlichkeit von deklarativen Prozessmodellen im Hinblick auf die Verwendung von hierarchischen Unterprozessen untersucht. Eine weitere Arbeit, welche sich mit der Verständlichkeit von deklarativen Prozessmodellen beschäftigt, ist \cite{haisjackl2014understanding}. Hier wird das Vorgehen von Systemanalysten beim Verstehen von deklarativen Modellen untersucht. In \cite{sabrina933} wird ebenfalls das Verständnis von deklarativen Prozessmodellen im Hinblick auf das Lesen von deklarativen Modellen, die Kombination von Constraints und Unterschiede zwischen flachen und hierarchischen Prozessmodellen beleuchtet. Diese Arbeiten stützen sich bei ihrer Untersuchung hauptsächlich auf empirische Daten, welche durch das Durchführen von Studien zustande gekommen sind. Die vorliegende Arbeit verwendet bei der Untersuchung der Verständlichkeit der deklarativen Prozessmodelle in erster Linie theoretische Erkenntnisse. Diese werden anschließend durch empirische Daten gestützt. \newline
In der vorliegenden Arbeit werden einige Erkenntnisse der vorgestellten Arbeiten beim Modellieren und beim Vergleich der Prozessmodelle beachtet. Beispielsweise wurden Unterprozesse beim  Modellieren verwendet um komplexe Prozessmodelle übersichtlicher darzustellen. Diese wurden aber beim Modellieren von sowohl BPMN als auch ConDec bei den gleichen Sachverhalten angewendet.\newline
Außerdem wurden die Ergebnisse von \cite{gruhn2006adopting} zum schwierigeren Verständnis von Patterns in BPMN bei der Durchführung des Vergleiches herangezogen.\newline
\cite{pinggera2012tracing, forster2012collaborative, pinggera2010investigating} untersuchen im Gegensatz zur vorliegenden Arbeit den Prozess des Prozessmodellierens an sich, aber sie vergleichen keine Prozessmodellierungssprachen im Hinblick auf eine geeignetere Anwendbarkeit bei der Modellierung für den Modellierer und den späteren Leser der Modelle.\newline
Im Gegensatz zu den hier vorgestellten Arbeiten liegt der Fokus des Vergleiches in dieser Arbeit nicht vor allem auf der Eignung zur Modellierung der beiden Prozessmodellierungssprachen. Die hier vorgestellten Arbeiten untersuchen alle die Verständlichkeit von Prozessmodellierungssprachen für den Leser der Prozessmodelle. Keiner dieser Arbeiten untersucht die Anwendbarkeit der Prozessmodellierungssprache bei der Modellierung. \newline


\section{Vergleich von Prozessmodellierungssprachen}

Dieser Abschnitt widmet sich Arbeiten, welche sich mit dem Vergleich von Prozessmodellierungssprachen beschäftigen.\newline
Die Arbeit \cite{recker2007does} untersucht Unterschiede in der Verständlichkeit zwischen Ereignisgesteuerten Prozessketten (EPK) und BPMN.\newline
Der Artikel \cite{fahland2010} beschäftigt sich mit dem Unterschied zwischen imperativen und deklarativen Prozessmodellierungssprachen und arbeitet deren Stärken und Schwächen heraus. Der Vergleich baut auf den Unterschieden von imperativen und deklarativen Programmiersprachen auf. \newline
\cite{pichler2012} untersucht aufbauend auf den Erkenntnissen von \cite{fahland2010} die Verständlichkeit von imperativen und deklarativen Prozessmodellierungssprachen anhand einer Studie. Als imperative Prozessmodellierungssprache dient in diesem Artikel ebenfalls BPMN und als deklarative Prozessmodellierungssprache ebenso ConDec. \newline 
\cite{fahland2010} und \cite{pichler2012} untersuchen, bei welchen abzubildenden Informationen entweder imperative oder deklarative Prozessmodellierungssprachen vorzuziehen sind. In der vorliegenden Arbeit wird beim Vergleich der Anwendbarkeit der beiden Prozessmodellierungssprachen nicht unter bestimmten abzubildenden Informationen unterschieden. Der Vergleich wird im Gegensatz zu \cite{fahland2010} und \cite{pichler2012}  ganz allgemein für alle möglichen abzubildenden Informationen durchgeführt. \newline
In der durchgeführten Studie in \cite{pichler2012} wurden die Teilnehmer vorher sowohl in der imperativen als auch in der deklarativen Prozessmodellierung geschult. In der Studie der vorliegenden Arbeit wurde darauf geachtet, Probanden mit unterschiedlichem Hintergrundwissen zu imperativen und deklarativen Prozessmodellen zu befragen. Das Wissen der Teilnehmer der Studie reichte hier von sehr großem Wissen bis überhaupt kein Wissen.\newline
Die vorgestellten Arbeiten vergleichen alle Prozessmodellierungssprachen anhand der Verständlichkeit aus Sicht des Lesers der Prozessmodelle. Keine dieser Arbeiten vergleicht Prozessmodellierungssprachen aus Sicht des Modellierers, wie es in dieser Arbeit ebenfalls der Fall ist. Zudem stützen sich die Vergleiche der hier vorgestellten Arbeiten größtenteils auf empirische Daten, während sich der Vergleich der vorliegenden Arbeit sowohl auf theoretische als auch empirische Daten stützt.




\chapter{Zusammenfassung und Ausblick}\label{sec:chapter8}






% anhänge
\appendix
\chapter{BPMN Notation}


\begin{figure}[!htbp]
\begin{center}
  \includegraphics[width=\linewidth]{BPM} %pdf, jpg, png...
  \caption{BPMN Ereignisse}
  \label{fig:BPM}
\end{center}
\end{figure}

\begin{figure}[!htbp]
\begin{center}
  \includegraphics[width=\linewidth]{BPM_2} %pdf, jpg, png...
  \caption{BPMN Übersicht}
  \label{fig:BPM_2}
\end{center}
\end{figure}

\begin{figure}[!htbp]
\begin{center}
  \includegraphics[width=\linewidth]{gatewa} %pdf, jpg, png...
  \caption{BPMN Gateways}
  \label{fig:gateways}
\end{center}
\end{figure}

\chapter{ConDec Notation}

 
 
 \begin{longtable}{|p{0.45\textwidth}|p{0.55\textwidth}|}
\hline
\textbf{Constraint} & \textbf{Erläuterung}\\
\hline
\begin{center}
  \includegraphics[scale=0.5]{absence} %pdf, jpg, png...
  \end{center}
& \textbf{absence (A)} \newline
Aktivität A darf nicht ausgeführt werden\\



\hline
\begin{center}

  \includegraphics[scale=0.5]{absenceN} %pdf, jpg, png...
    \end{center}

& \textbf{absence (n+1, A)} \newline
Aktivität A kann höchstens n-mal ausgeführt werden, aber nicht n+1-mal\\
\hline
\begin{center}

  \includegraphics[scale=0.5]{existenceN} %pdf, jpg, png...
    \end{center}

& \textbf{existence (n, A)}\newline
Aktivität A muss mindestens n-mal ausgeführt werden\\
\hline
\begin{center}

  \includegraphics[scale=0.5]{ExactlyN} %pdf, jpg, png...
    \end{center}

& \textbf{exactly (n, A)}\newline
Aktivität A muss genau n-mal ausgeführt werden\\
\hline

\begin{center}

  \includegraphics[scale=0.5]{Init} %pdf, jpg, png...
    \end{center}

& \textbf{init (A)}\newline
Aktivität A muss als erste Aktivität ausgeführt werden\\
\hline


\begin{center}

  \includegraphics[scale=0.5]{RespondedExistence} %pdf, jpg, png...
    \end{center}&

\textbf{responded existence} \newline  Falls A ausgeführt wird, muss B entweder davor oder danach ebenfalls ausgeführt werden. \newline
Beispiel: Korrekt: [A,B]; [B,A]; Inkorrekt:[A];\\
\hline
\begin{center}

  \includegraphics[scale=0.5]{CoExistence} %pdf, jpg, png...
    \end{center} &
\textbf{co-existence} \newline A und B kommen in einem Pfad immer zusammen vor.\newline
Beispiel: Korrekt: [A,B]; Inkorrekt: [A]; [B];
 \\
\hline

\begin{center}

  \includegraphics[scale=0.5]{response} %pdf, jpg, png...
    \end{center} &
\textbf{response} \newline Falls A ausgeführt wird, muss B danach ebenfalls ausgeführt werden. \newline
Beispiel: Korrekt: [B,A,A,A,C,B]; Inkorrekt: [B,A,A,A,C]
\\
\hline
\begin{center}

  \includegraphics[scale=0.5]{Precedence} %pdf, jpg, png...
    \end{center} &
    \textbf{precedence}\newline
Falls B  ausgeführt wird, muss vorher A ausgeführt werden. \newline
Beispiel: Korrekt: [A,C,B,B,A]; Inkorrekt:[C,B,B,A]
\\
\hline
\begin{center}

  \includegraphics[scale=0.5]{Succession} %pdf, jpg, png...
    \end{center}&
\textbf{succession} \newline Verlangt, dass die beiden Constraints precedence und response zwischen den Aktivitäten A und B eingehalten werden. Somit muss jede Aktivität A von Aktivität B gefolgt werden und für jede Aktivität B muss eine Aktivität A vorhanden sein. \newline
Beispiel: Korrekt: [A,C,A,B,B]; Inkorrekt: [A,C]\\

\hline

 \begin{center}

  \includegraphics[scale=0.5]{AlternateResponse} %pdf, jpg, png...
    \end{center} &
\textbf{alternate response} \newline  Verlangt, dass nach einer Aktivität A Aktivität B ausgeführt wird, jedoch darf vor Aktivität B nicht eine weitere Aktivität A ausgeführt werden. \newline
Beispiel: Korrekt: [B,A,C,B,A,B] ; Inkorrekt: [B,A,C,A,B,A,B] \\

\hline

\begin{center}

  \includegraphics[scale=0.5]{AlternatePrecedence} %pdf, jpg, png...
    \end{center} &
\textbf{alternate precedence}\newline
  Verlangt, dass jeder Instanz von Aktivität B eine Instanz der Aktivität A vorausgeht. Die nächste Instanz einer Aktivität B kann somit nicht vor der nächsten Instanz von Aktivität A ausgeführt werden.
  \newline
  Beispiel: Korrekt: [A,C,B,A,B,A]; Inkorrekt: [A,C,B,B,A]\\
\hline
\begin{center}

  \includegraphics[scale=0.5]{AlternateSuccession} %pdf, jpg, png...
    \end{center}&
\textbf{alternate succession} \newline
 Stellt eine Kombination aus alternate response und alternate precedence dar. \newline
  Beispiel:  Korrekt: [A,C,B,A,B,A,B]; Inkorrekt: [C,B,A,A,B] \\
\hline
\begin{center}

  \includegraphics[scale=0.5]{ChainResponse} %pdf, jpg, png...
    \end{center}&
 \textbf{chain response}\newline
  Verlangt, dass die nächste Aktivität, welche nach Aktivität A ausgeführt wird, immer Aktivität B ist. \newline
  Beispiel: Korrekt: [B,A,B,C,A,B]; Inkorrekt: [B,A,C,A,B]\\
\hline
\begin{center}

  \includegraphics[scale=0.5]{ChainPrecedence} %pdf, jpg, png...
    \end{center}&
\textbf{chain precedence} \newline
 Verlangt, dass Aktivität A immer unmittelbar bevor Aktivität B ausgeführt wird.\newline
 Beispiel: Korrekt: [A,B,C,A,B,A]; Inkorrekt: [A,B,C,B,A] \\
\hline
\begin{center}

  \includegraphics[scale=0.5]{ChainSuccession} %pdf, jpg, png...
    \end{center} &
\textbf{chain succession} \newline  Stellt eine Kombination aus chain response und chain precedence dar und verlangt, dass Aktivität A und Aktivität B jeweils nebeneinander ausgeführt werden. \newline
Beispiel: Korrekt: [A,B,C,A,B,A,B]; Inkorrekt: [A,B,C,A,B,A,B,A,C]\\
\hline
 
\begin{center}

  \includegraphics[scale=0.5]{Choice} %pdf, jpg, png...
    \end{center}&

\textbf{choice} \newline  Mindestens eine der beiden Aktivitäten A oder B muss ausgeführt werden.  \newline
Beispiel: Korrekt: [A]; [B]; Inkorrekt:[ ];
\\
\hline

\begin{center}
  \includegraphics[scale=0.5]{ChoiceOneOfThree} %pdf, jpg, png...
    \end{center} &
    \textbf{choice 1 of 3}\newline
Mindestens eine der drei Aktivitäten A,B oder C muss ausgeführt werden.. \newline
Beispiel: Korrekt: [A]; [B];[C]; Inkorrekt:[]
\\
\hline
\begin{center}

  \includegraphics[scale=0.5]{ChoiceOneOfTwo} %pdf, jpg, png...
    \end{center}&
\textbf{1 of 2} \newline Entweder A oder B muss mindestens einmal ausgeführt werden.\newline
Beispiel: Korrekt: [A,C,A,B,B]; Inkorrekt: [C]\\
\hline

\begin{center}

  \includegraphics[scale=0.5]{ExclusiveChoice} %pdf, jpg, png...
    \end{center} &
\textbf{exclusive choice}\newline
  Entweder A oder B kann ausgeführt werden. \newline
  Beispiel:  Korrekt: [A]; [B] Inkorrekt: [A,B] \\
\hline
\begin{center}

  \includegraphics[scale=0.5]{ExclusiveChoiceOneOfThree} %pdf, jpg, png...
    \end{center}&
\textbf{exclusive choice 1 of 3} \newline
 Entweder A oder B oder C kann ausgeführt werden. \newline
  Beispiel:  Korrekt: [A]; [B]; [C]; Inkorrekt: [A,B]; [A,C]; [B,C] \\
\hline
\begin{center}

  \includegraphics[scale=0.5]{NotCoExistence} %pdf, jpg, png...
    \end{center} &
\textbf{not co-existence}\newline
  Verlangt, dass falls Aktivität A ausgeführt wird, darf Aktivität B nicht mehr ausgeführt werden und umgekehrt.
  \newline
  Beispiel: Korrekt: [A,C,A,A] ; Inkorrekt: [A,C,A,A,B] \\
\hline

\hline
\begin{center}

  \includegraphics[scale=0.5]{NotSuccession} %pdf, jpg, png...
    \end{center} &
\textbf{not succession}\newline
  Verlangt, dass falls Aktivität A ausgeführt wird, darf Aktivität B nicht danach ausgeführt werden.
  \newline
  Beispiel: Korrekt: [B,C,A,C,A] ; Inkorrekt: [A,C,B]\\
\hline

\hline
\begin{center}

  \includegraphics[scale=0.5]{NotChainSuccession} %pdf, jpg, png...
    \end{center} &
\textbf{negation chain succession}\newline
  Verlangt, dass die Aktivitäten A und B nicht nebeneinander ausgeführt werden.
  \newline
  Beispiel: Korrekt: [A,C,B,C] ; Inkorrekt: [B,A,B,A] \\
\hline

 \caption{Constraints ConDec \cite{Montali2010, Pesic200}}
 \end{longtable}

\chapter{Fragebögen}

\includepdf[pages={1-7,9-12}]{Fragebogen1.pdf}
\includepdf[pages={1-11}]{Fragebogen2.pdf}




% hier kommen die anhänge

\backmatter			% abtrennung für verzeichnisse

% hier die verzeichnisse
\listoffigures
\listoftables

% Bibliograhpy
\bibliographystyle{alphadin} 	% buchstaben + jahr und sortiert
\bibliography{Literature}

\clearpage
\erklaerung

\end{document}
