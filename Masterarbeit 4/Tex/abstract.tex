\section*{Kurzfassung}

Softwaresysteme sind heutzutage aus dem täglichen Leben nicht mehr weg zu denken. Sie finden sich in jedem elektronischen Gerät und kaum jemand kann in der jetzigen Zeit auf Softwaresysteme verzichten. Da die Herstellung von Softwaresystemen mit einer sehr hohen Komplexität einhergeht, ist es wichtig, 
bei deren Erstellung einem Softwareentwicklungsprozess zu folgen. Softwareentwicklungsprozesse geben Aktivitäten vor, welche zur Herstellung von Software notwendig sind. Dadurch helfen sie, die Entwicklung von Software zu strukturieren. Drei sehr bekannte Vertreter von Softwareentwicklungsprozessen sind Scrum, Open UP und das V-Modell XT.\newline

An der Entwicklung von Software sind oftmals eine Reihe verschiedener Personen mit unterschiedlichen fachlichen Hintergründen beteiligt. Diese müssen alle den vorgegeben Softwareentwicklungsprozess verstehen. Da eine rein textuelle Beschreibung der selbigen oftmals sehr umfangreich ist, bieten sich Prozessmodellierungssprachen zur Beschreibung von Softwareentwicklungsprozessen an da diese sowohl eine formale Korrektheit aufweisen aber auch intuitiv verständlich sind.\newline

Es existieren inzwischen eine ganze Reihe verschiedener Prozessmodellierungssprachen. Deren Vor- und Nachteile werden intensiv diskutiert. Hierbei wird auch der Unterschied zwischen deklarativen und imperativen Prozessmodellierungssprachen erörtert.\newline

Aus diesem Grund werden in dieser Arbeit Teile der drei Softwareentwicklungsprozesse Scrum, Open UP und V-Modell XT sowohl in imperativer als auch in deklarativer Prozessmodellierungssprache modelliert und anhand der daraus entstehenden Modelle wird die Anwendbarkeit der imperativen und deklarativen Prozessmodellierungsansätze verglichen.



